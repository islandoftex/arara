\documentclass[a4paper,twoside,12pt]{memoir}

\usepackage[T1]{fontenc}
\usepackage[utf8]{inputenc}
\usepackage[margin=2.5cm]{geometry}
\usepackage{arara}

\addbibresource{references.bib}
\newcommand{\araraversion}{4.0}
\newcommand{\todo}[1]{\fbox{\em#1}}

\begin{document}

\begin{titlingpage}
\vspace*{2em}

\begin{center}
\includegraphics[scale=0.7]{../logos/logo2.pdf}

\vspace{4em}

\begin{tcolorbox}[
  boxrule=0pt,
  colback=araracolour,
  top=1em,
  bottom=1em
]
  \color{white}
  \centering
  \Huge
  \sffamily
  \bfseries User manual
\end{tcolorbox}

\vspace{6em}

{\large\em Paulo Cereda, Marco Daniel,\\
Brent Longborough, and Nicola Talbot\par}

\vspace{2em}

\url{https://github.com/cereda/arara}

\vfill

{\color{araracolour}
\LARGE
\sffamily
\bfseries
Version \araraversion}

\end{center}
\end{titlingpage}

\chapterstyle{araraheadings}
\pagestyle{headings}
\frontmatter
\nouppercaseheads

\cleardoublepage

\vspace*{25em}

\begin{flushright}
\em No birds were harmed in the making of this manual.
\end{flushright}

% !TeX root = ../arara-manual.tex
\chapter*{Foreword}
\label{chap:foreword}

\epigraph{That deserves no less than a ``Holy guacamole!''.}{\textsc{Gonzalo Medina}}

\emph{Foreword here.}

\vfill

\begin{flushright}
Nicola Louise Cecilia Talbot\\
\emph{on behalf of the \arara\ team}
\end{flushright}

% !TeX root = ../arara-manual.tex
\chapter*{Prologue}
\label{chap:prologue}

\epigraph{Moral of the story: never read the
documentation, bad things happen.}{\textsc{David Carlisle}}

\emph{Prologue here.}

\vfill

\begin{flushright}
Paulo Roberto Massa Cereda\\
\emph{on behalf of the \arara\ team}
\end{flushright}

\chapter*{Release information}
\label{chap:releaseinformation}

\epigraph{Are there programming languages other
than \TeX?}{\textsc{Enrico Gregorio}}

\emph{Release information here}

% !TeX root = ../arara-manual.tex
\chapter*{License}
\label{chap:license}

\epigraph{Anything that prevents you from being friendly, a good neighbour, is a terror tactic.}{\textsc{Richard Stallman}}

\arara\ is licensed under the \href{http://www.opensource.org/licenses/bsd-license.php}{New BSD License}. It is important to observe that the New BSD License has been verified as a GPL-compatible free software license by the \href{http://www.fsf.org/}{Free Software Foundation}, and has been vetted as an open source license by the \href{http://www.opensource.org/}{Open Source Initiative}.

\vfill

\begin{messagebox}{New BSD License}{araracolour}{\icinfo}{white}
\footnotesize
\includegraphics[scale=0.25]{logos/logo1.pdf}

Copyright \textcopyright\ 2012--2018, Paulo Roberto Massa Cereda\\
All rights reserved.

\vspace{1em}

Redistribution and use in source and binary forms, with or without modification, are permitted provided that the following conditions are met:

\begin{itemize}
\item Redistributions of source code must retain the above copyright notice, this list of conditions and the following disclaimer.

\item Redistributions in binary form must reproduce the above copyright notice, this list of conditions and the following disclaimer in the documentation and/or other materials provided with the distribution.
\end{itemize}

This software is provided by the copyright holders and contributors ``as is'' and any express or implied warranties, including, but not limited to, the implied warranties of merchantability and fitness for a particular purpose are disclaimed. In no event shall the copyright holder or contributors be liable for any direct, indirect, incidental, special, exemplary, or consequential damages (including, but not limited to, procurement of substitute goods or services; loss of use, data, or profits; or business interruption) however caused and on any theory of liability, whether in contract, strict liability, or tort (including negligence or otherwise) arising in any way out of the use of this software, even if advised of the possibility of such damage.
\end{messagebox}


\cleardoublepage

\vspace*{25em}

\begin{flushright}
\em To Marco's son Niclas.
\end{flushright}

\cleardoublepage

\tableofcontents*

\cleardoublepage

\listoffigures*

\cleardoublepage

\listoftables*

\mainmatter

\part[A primer on formats and scripting]{A primer on\\ formats and scripting}
\label{part:primer}

% !TeX root = ../arara-manual.tex
\chapter{YAML}
\label{chap:yaml}

According to the \href{http://yaml.org/spec/1.2/spec.html}{specification}, YAML (a recursive acronym for \emph{YAML Ain't Markup Language}) is a human-friendly, cross language, Unicode-based data serialization language designed around the common native data type of programming languages. \arara\ uses this format in three circumstances:

\begin{enumerate}
\item\emph{Parametrized directives}, as the set of attribute/value pairs (namely, argument name and corresponding value) is represented by a map. This particular type of directive is formally introduced in Section~\ref{sec:directives}, on page~\pageref{sec:directives}.

\item\emph{Rules}, as their entire structure is represented by a set of specific keys and their corresponding values (a proper YAML document). A rule follows a very strict model, detailed in Section~\ref{sec:rule}, on page~\pageref{sec:rule}.

\item\emph{Configuration files}, as the general settings are represented by a set of specific keys and their corresponding values (a proper YAML document). Configuration files are covered in Chapter~\ref{chap:configurationfile}, on page~\pageref{chap:configurationfile}.
\end{enumerate}

This chapter only covers the relevant parts of the YAML format for a consistent use with \arara. For advanced topics, I highly recommend the complete format specification, available online.

\section{Collections}
\label{sec:yamlcollections}

According to the specification, YAML's block collections use indentation for scope and begin each entry on its own line. Block sequences indicate each entry with a dash and space. Mappings use a colon and space to mark each \emph{key: value} pair. Comments begin with an octothorpe \rbox{\#}. \arara\ relies solely on mappings and a few scalars to sequences at some point. Let us see an example of a sequence:

\begin{codebox}{A sequence of scalars in YAML}{teal}{\icnote}{white}
team:
- Paulo Cereda
- Marco Daniel
- Brent Longborough
- Nicola Talbot
- Ben Frank
\end{codebox}

It is quite straightforward: \abox{team} holds a sequence of four scalars. YAML also has flow styles, using explicit indicators rather than indentation to denote scope. The flow sequence is written as a comma-separated list within square brackets:

\begin{codebox}{A sequence of scalars in YAML}{teal}{\icnote}{white}
primes: [ 2, 3, 5, 7, 11 ]
\end{codebox}

Attribute maps are easily represented by nesting entries, respecting indentation. For instance, consider a map \abox{developer} containing two keys, \abox{name} and \abox{country}. The YAML representation is presented as follows:

\begin{codebox}{An attribute map in YAML}{teal}{\icnote}{white}
developer:
 name: Paulo
 country: Brazil
\end{codebox}

Similarly, the flow mapping uses curly braces. Observe that this is the form adopted by a parametrized directive (see syntax in Section~\ref{sec:directives}, on page~\pageref{sec:directives}):

\begin{codebox}{An attribute map in YAML (flow mapping)}{teal}{\icnote}{white}
developer: { name: Paulo, country: Brazil }
\end{codebox}

An attribute map can contain sequences as well. Consider the following code where \abox{developers} holds a list of two developers containing their names and countries:

\begin{codebox}{An attribute map with sequences in YAML}{teal}{\icnote}{white}
developers:
- name: Paulo
  country: Brazil
- name: Marco
  country: Germany
\end{codebox}

The previous code can be easily represented in flow style by using square and curly brackets to represent sequences and attribute maps.

\section{Scalars}
\label{sec:yamlscalars}

Scalar content can be written in block notation, using a literal style, indicated by a vertical bar, where \emph{all line breaks are significant}. Alternatively, they can be written with the folded style, denoted by a greater-than sign, where \emph{each line break is folded to a space} unless it ends an empty or a more-indented line. It is mportant to note that \arara\ intensively uses both styles (as seen in Section~\ref{sec:rule}, on page~\pageref{sec:rule}). Let us see an example:

\begin{codebox}{Scalar content in literal and folded styles}{teal}{\icnote}{white}
logo: |
  This is the arara logo
  in its ASCII glory! 
    __ _ _ __ __ _ _ __ __ _ 
   / _` | '__/ _` | '__/ _` |
  | (_| | | | (_| | | | (_| |
   \__,_|_|  \__,_|_|  \__,_|
slogan: >
  The cool TeX
  automation tool
\end{codebox}

As seen in the previous code, \abox{logo} holds the ASCII logo of our tool, respecting line breaks. Similarly, observe that the \abox{slogan} key holds the text with line breaks replaced by spaces (in the same fashion \TeX\ does with consecutive, non-empty lines).

\begin{messagebox}{Block indentation indicator}{attentioncolour}{\icattention}{black}
\setlength{\parskip}{1em}
According to the YAML specification, the indentation level of a block scalar is typically detected from its first non-empty line. It is an error for any of the leading empty lines to contain more spaces than the first non-empty line, hence the ASCII logo could not be represented, as it starts with a space.

When detection would fail, YAML requires that the indentation level for the content be given using an explicit indentation indicator. This level is specified as the integer number of the additional indentation spaces used for the content, relative to its parent node. It would be the case if we want to represent our logo without the preceding text.
\end{messagebox}

YAML's flow scalars include the plain style and two quoted styles. The double-quoted style provides escape sequences. The single-quoted style is useful when escaping is not needed. All flow scalars can span multiple lines. Note that line breaks are always folded. Since \arara\ uses MVEL as its underlying scripting language (Chapter~\ref{chap:mvel}, on page~\pageref{chap:mvel}), it might be advisable to quote scalars when starting with forbidden symbols in YAML.

\section{Tags}
\label{sec:yamltags}

According to the specification, in YAML, untagged nodes are given a type depending on the application. The examples covered in this primer use the \rbox{seq}, \rbox{map} and \rbox{str} types from the fail safe schema. Explicit typing is denoted with a tag using the exclamation point symbol. Global tags are usually uniform resource identifiers and may be specified in a tag shorthand notation using a handle. Application-specific local tags may also be used. For \arara, there is a special schema used for both rules and configuration files, so in those cases, make sure to add \abox{!config} as global tag:

\begin{codebox}{Global tag for rules and configuration files}{teal}{\icnote}{white}
!config
\end{codebox}

In particular, rules and configuration files of \arara\ are properly covered in Section~\ref{sec:rule} and Chapter~\ref{chap:configurationfile}, on pages~\pageref{sec:rule} and~\pageref{chap:configurationfile}, respectively. For now, it suffices to say that the \abox{!config} global tag is necessary to provide the correct mapping of values inside our tool.

\section{Further reading}
\label{sec:yamlfurtherreading}

This chapter does not cover all features of the YAML format, so further reading is advisable. I highly recommend the \href{http://yaml.org/spec/1.2/spec.html}{official YAML specification}, currently covering the third version of the format.

% !TeX root = ../arara-manual.tex
\chapter{MVEL}
\label{chap:mvel}

According to the \href{https://en.wikipedia.org/wiki/MVEL}{Wikipedia entry}, the MVFLEX Expression Language (hereafter referred as MVEL) is a hybrid, dynamic, statically typed, embeddable expression language and runtime for the Java platform. Originally started as a utility language for an application framework, the project is now developed completely independently. \arara\ relies on such scripting language in two circumstances:

\begin{enumerate}
\item\emph{Rules}, as nominal attributes gathered from directives are used to build complex command invocations and additional computations. A rule follows a very strict model, detailed in Section~\ref{sec:rule}, on page~\pageref{sec:rule}.

\item\emph{Conditionals}, as logical expressions must be evaluated in order to decide whether and how a directive should be interpreted. Conditionals are detailed in Section~\ref{sec:directives}, on page~\pageref{sec:directives}.
\end{enumerate}

This chapter only covers the relevant parts of the MVEL language for a consistent use with \arara. For advanced topics, I highly recommend the official language guide for MVEL 2.0, available online.

\section{Basic usage}
\label{sec:mvelbasicusage}

The following primer is provided by the \href{https://mvel.documentnode.com/}{official language guide}, almost verbatim, with a few modifications to make it more adherent to our needs with \arara. Consider the following expression:

\begin{codebox}{Simple property expression}{teal}{\icnote}{white}
user.name
\end{codebox}

In this expression, we have a single identifier \rbox{user.name}, which by itself is a property expression, in that the only purpose of such expression is to extract a property out of a variable or context object, namely \rbox{user}. Property expressions are widely used by \arara, as directive parameters are converted to a map inside the corresponding rule scope. For instance, a parameter \rbox{foo} in a directive will be mapped as \rbox{parameters.foo} inside a rule during interpretation. This topic is detailed in Section~\ref{sec:directives}, on page~\pageref{sec:directives}. The scripting language can also be used for evaluating a boolean expression:

\begin{codebox}{Boolean expression evaluation}{teal}{\icnote}{white}
user.name == 'John Doe'
\end{codebox}

This expression yields a boolean result, either \rbox{true} or \rbox{false} based on a comparation operation. Like a typical programming language, MVEL supports the full gambit of operator precedence rules, including the ability to use bracketing to control execution order:

\begin{codebox}{Execution order control through bracketing}{teal}{\icnote}{white}
(user.name == 'John Doe') && ((x * 2) - 1) > 20
\end{codebox}

You may write scripts with an arbitrary number of statements using semicolon to denote the termination of a statement. This is required in all cases except in cases where there is only one statement, or for the last statement in a script:

\begin{codebox}{Multiple statements}{teal}{\icnote}{white}
statement1; statement2; statement3
\end{codebox}

It is important to observe that MVEL expressions use a \emph{last value out} principle. This means, that although MVEL supports the \rbox{return} keyword, it can be safely omitted. For example:

\begin{codebox}{Automatic return}{teal}{\icnote}{white}
foo = 10;
bar = (foo = foo * 2) + 10;
foo;
\end{codebox}

In this particular example, the expression automatically returns the value of \rbox{foo} as it is the last value of the expression. It is functionally identical to:

\begin{codebox}{Explicit return}{teal}{\icnote}{white}
foo = 10;
bar = (foo = foo * 2) + 10;
return foo;
\end{codebox}

Personally, I like to explicitly add a \rbox{return} statement, as it provides a visual indication of the expression exit point. All rules released with \arara\ favour this writing style. However, feel free to choose any writing style you want, as long as the resulting code is consistent.

The type coercion system of MVEL is applied in cases where two incomparable types are presented by attempting to coerce the right value to that of the type of the left value, and then vice-versa. For example:

\begin{codebox}{Type coercion}{teal}{\icnote}{white}
"123" == 123;
\end{codebox}

Surprisingly, the evaluation of such expression holds \rbox{true} in MVEL because the underlying type coercion system will coerce the untyped number \rbox{123} to a string \rbox{123} in order to perform the comparison.

\section{Inline lists, maps and arrays}
\label{sec:mvelinlinelistsmapsandarrays}

According to the documentation, MVEL allows you to express lists, maps and arrays using simple elegant syntax. Lists are expressed in the following format:

\begin{codebox}{Creating a list}{teal}{\icnote}{white}
[ "Jim", "Bob", "Smith" ]
\end{codebox}

Note that lists are denoted by comma-separated values delimited by square brackets. Similarly, maps (sets of key/value attributes) are expressed in the following format: 

\begin{codebox}{Creating a map}{teal}{\icnote}{white}
[ "Foo" : "Bar", "Bar" : "Foo" ]
\end{codebox}

Note that attributes are composed by a key, a colon and the corresponding value. A map is denoted by comma-separated attributes delimited  by square brackets. Finally, arrays are expressed in the following format:

\begin{codebox}{Creating an array}{teal}{\icnote}{white}
{ "Jim", "Bob", "Smith" }
\end{codebox}

One important aspect about inline arrays is their special ability to be coerced to other array types. When you declare an inline array, it is untyped at first and later coerced to the type needed in context. For instance, consider the following code, in which \rbox{sum} takes an array of integers:

\begin{codebox}{Array coercion}{teal}{\icnote}{white}
math.sum({ 1, 2, 3, 4 });
\end{codebox}

In this case, the scripting language will see that the target method accepts an integer array and automatically type the provided untyped array as such. This is an important feature exploited by \arara\ when calling methods within the rule or conditional scope.

\section{Property navigation}
\label{sec:propertynavigation}

MVEL provides a single, unified syntax for accessing properties, static fields, maps and other structures. Lists are accessed the same as arrays. For example, these two constructs are equivalent (MVEL and Java access styles for lists and arrays, respectively):

\begin{codebox}{MVEL access style for lists and arrays}{teal}{\icnote}{white}
user[5]
\end{codebox}

\begin{codebox}{Java access style for lists and arrays}{teal}{\icnote}{white}
user.get(5)
\end{codebox}

Observe that MVEL accepts plain Java methods as well. Maps are accessed in the same way as arrays except any object can be passed as the index value. For example, these two constructs are equivalent (MVEL and Java access styles for maps, respectively):

\begin{codebox}{MVEL access style for maps}{teal}{\icnote}{white}
user["foobar"]
user.foobar
\end{codebox}

\begin{codebox}{Java access style for maps}{teal}{\icnote}{white}
user.get("foobar")
\end{codebox}

It is advisable to favour such access styles over their Java counterparts when writing rules and conditionals for \arara. The clean syntax offer subsidies for a more readable code.

\section{Flow control}
\label{sec:mvelflowcontrol}

The expression language goes beyond simple evaluations. In fact, MVEL supports an assortment of control flow operators (namely, conditionals and repetitions) which allows advanced scripting operations. Consider this conditional statement:

\begin{codebox}{Conditional statement}{teal}{\icnote}{white}
if (var > 0) {
   r = "greater than zero";
}
else if (var == 0) { 
   r = "exactly zero";
}
else { 
   r = "less than zero";
}
\end{codebox}

As seen in the previous code, the syntax is very similar to the ones found in typical programming languages. MVEL also provides a shorter version, know as ternary statement:

\begin{codebox}{Ternary statement}{teal}{\icnote}{white}
answer == true ? "yes" : "no";
\end{codebox}

The \rbox{foreach} statement accepts two parameters separated by a colon, being the first the local variable holding the current element, and the second the collection or array to be iterated. For example:

\begin{codebox}{Iteration statement}{teal}{\icnote}{white}
foreach (name : people) {
    System.out.println(name);
}
\end{codebox}

As expected, MVEL also implements the standard C \rbox{for} loop. Observe that newer versions of MVEL allow an abreviation of \rbox{foreach} to the usual \rbox{for} statement, as syntactic sugar. In order to explicitly indicate a collection iteration, we usually use \rbox{foreach} in the default rules for \arara, but both statements behave exactly the same from a semantic point of view. 

\begin{codebox}{Iteration statement}{teal}{\icnote}{white}
for (int i = 0; i < 100; i++) { 
   System.out.println(i);
}
\end{codebox}

The scripting language also provides two versions of the \rbox{do} statement: one with \rbox{while} and one with \rbox{until} (being the latter the exact inverse of the former):

\begin{codebox}{Iteration statement}{teal}{\icnote}{white}
do {
    x = something();
} while (x != null);
\end{codebox}

\begin{codebox}{Iteration statement}{teal}{\icnote}{white}
do {
   x = something();
} until (x == null);
\end{codebox}

At last, MVEL also implements the standard \rbox{while}, with the significant addition of a \rbox{until} counterpart (for inverted logic):

\begin{codebox}{Iteration statement}{teal}{\icnote}{white}
while (isTrue()) {
   doSomething();
}
\end{codebox}

\begin{codebox}{Iteration statement}{teal}{\icnote}{white}
until (isFalse()) {
   doSomething();
}
\end{codebox}

Since \rbox{while} and \rbox{until} are unbounded (i.e, the number of iterations required to solve a problem may be unpredictable), we usually tend to avoid using such statements when writing rules for \arara.

\section{Projections and folds}
\label{sec:mvelprojectionsandfolds}

Projections are a way of representing collections. According to the official documentation, using a very simple syntax, one can inspect very complex object models inside collections in MVEL using the \rbox{in} operator. For example:

\begin{codebox}{Projection and fold}{teal}{\icnote}{white}
names = (user.name in users);
\end{codebox}

As seen in the previous code, \rbox{names} holds all values from the \rbox{name} property of each element, represented locally by a placeholder \rbox{user}, from the collection \rbox{users} being inspected. This feature can even perform nested operations.

\section{Assignments}
\label{sec:mvelassignments}

According to the official documentation, the scripting language allows variable assignment in expressions, either for extraction from the runtime, or for use inside the expression. As MVEL is a dynamically typed language, there is no need to specify a type in order to declare a new variable. However, feel free to explicitly declare the type when desired.

\begin{codebox}{Assignment}{teal}{\icnote}{white}
str = "My string";
String str = "My string";
\end{codebox}

Unlike Java, however, the scripting language provides automatic type conversion (when possible) when assigning a value to a typed variable. In the following example, an integer value is assigned to a string:

\begin{codebox}{Assignment}{teal}{\icnote}{white}
String num = 1;
\end{codebox}

For dynamically typed variables, in order to perform a type conversion, it is just a matter of explicitly casting the value to the desired type. In the following example, an explicit string cast is assigned to the \rbox{num} variable:

\begin{codebox}{Assignment}{teal}{\icnote}{white}
num = (String) 1;
\end{codebox}

When writing rules for \arara, is advisable to keep variables to a minimum in order to avoid unnecessary assignments and a potential performance drop. However, make sure to favour readability over unmaintained code.

\section{Basic templating}
\label{sec:mvelbasictemplating}

MVEL templates are comprised of \emph{orb} tags inside a plaintext document. Orb tags denote dynamic elements of the template which the engine will evaluate at runtime. \arara\ heavily relies on this concept for runtime evaluation of conditionals and rules. For rules, we use orb tags to return either a string from a textual template or a proper command object. The former constituted the basis of command generation in previous versions of our tool; from version 4.0 on, we highly recommend the latter, detailed in Section~\ref{sec:rule}, on page~\ref{sec:rule}. Conditionals are in fact orb tags in desguise, such that the expression (or a sequence of expression) is properly evaluated at runtime. Consider the following example:

\begin{codebox}{Template}{teal}{\icnote}{white}
My favourite team is @{ person.name == 'Enrico'
? 'Juventus' : 'Palmeiras' }!
\end{codebox}

The previous code features a basic form of orb tag named \emph{expression orb}. It contains a expression (or a sequence of expressions) which will be evaluated to a certain value, as seen earlier on, when discussing the \emph{last value out} principle. In the example, the value to be returned will be a string containing a football team name (the result is of course based on the comparison outcome).

\section{Further reading}
\label{sec:mvelfurtherreading}

This chapter does not cover all features of the MVEL expression language, so further reading is advisable. I highly recommend the \href{http://mvel.documentnode.com/}{MVEL language guide} currently covering version 2.0 of the language.

\part{The application}
\label{part:application}

% !TeX root = ../arara-manual.tex
\chapter{Introduction}
\label{chap:introduction}

Hello there, welcome to \arara, the cool \TeX\ automation tool! This chapter is actually a quick introduction to what you can (and cannot) expect from \arara. For now, concepts will be informally presented and will be detailed later on, in the next chapters.

\section{What is this tool?}
\label{sec:whatisthistool}

Good question! \arara\ is a \TeX\ automation tool based on rules and directives. It is, in some aspects, similar to other well-known tools like \rbox{latexmk} and \rbox{rubber}. The key difference (and probably the selling point) might be the fact that \arara\ aims at explicit instructions in the source code (in the form of comments) in order to determine what to do instead of relying on other resources, such as log file analysis. It is a different approach for an automation tool, and we have both advantages and disadvantages of such design. Let us use the following file \rbox{hello.tex} as an example:

\begin{ncodebox}{Source file}{teal}{\icnote}{white}{hello.tex}
\documentclass{article}

\begin{document}
Hello world!
\end{document}
\end{ncodebox}

How would one successfully compile \rbox{hello.tex} with \rbox{latexmk} and \rbox{rubber}, for instance? It is quite straightforward: it is just a matter of providing the file to the tool and letting it do the hard work:

\begin{codebox}{Terminal}{teal}{\icnote}{white}
$ latexmk -pdf mydoc.tex
$ rubber --pdf mydoc.tex
\end{codebox}

The mentioned tools perform an analysis on the file and decide what has to be done. However, if one tries to invoke \rbox{arara} on \rbox{hello.tex}, I am afraid \emph{nothing} will be generated; the truth is, \arara\ does not know what to do with your file, and the tool will even raise an error message complaining about this issue:

\begin{codebox}{Terminal}{teal}{\icnote}{white}
$ arara hello.tex
  __ _ _ __ __ _ _ __ __ _ 
 / _` | '__/ _` | '__/ _` |
| (_| | | | (_| | | | (_| |
 \__,_|_|  \__,_|_|  \__,_|

Processing 'hello.tex' (size: 86 bytes, last modified: 05/03/2018
07:28:30), please wait.

It looks like no directives were found in the provided file. Make
sure to include at least one directive and try again.

Total: 0.00 seconds
\end{codebox}

Quite surprising. However, this behaviour is not wrong at all, it is completely by design: \arara\ needs to know what you want. And for that purpose, you need to tell the tool what to do.

\begin{messagebox}{A very important concept}{attentioncolour}{\icattention}{black}
That is the major difference of \arara\ when compared to other tools: \emph{it is not an automatic process and the tool does not employ any guesswork on its own}. You are in control of your documents; \arara\ will not do anything unless you \emph{teach it how to do a task and explicitly tell it to execute the task}.
\end{messagebox}

Now, how does one tell \arara\ to do a task? That is the actually the easy part, provided that you have everything up and running. We accomplish the task by adding a special comment line, hereafter known as \emph{directive}, somewhere in our \rbox{hello.tex} file (preferably in the first lines):

\begin{ncodebox}{Source file}{teal}{\icnote}{white}{hello.tex}
% arara: pdflatex
\documentclass{article}

\begin{document}
Hello world!
\end{document}
\end{ncodebox}

For now, do not worry too much about the terms, we will come back to them later on, in Chapter~\ref{chap:importantconcepts}, on page~\pageref{chap:importantconcepts}. It suffices to say that \arara\ expects \emph{you} to provide a list of tasks, and this is done by inserting special comments in the source file. Let us see how \arara\ behaves with this updated code:

\begin{codebox}{Terminal}{teal}{\icnote}{white}
$ arara hello.tex 
  __ _ _ __ __ _ _ __ __ _ 
 / _` | '__/ _` | '__/ _` |
| (_| | | | (_| | | | (_| |
 \__,_|_|  \__,_|_|  \__,_|

Processing 'hello.tex' (size: 86 bytes, last modified: 05/03/2018
07:28:30), please wait.

(PDFLaTeX) PDFLaTeX engine .............................. SUCCESS

Total: 0.73 seconds
\end{codebox}

Hurrah, we finally got our document properly compiled with a \TeX\ engine by the inner workings of our beloved tool, resulting in an expected \rbox{hello.pdf} file created using the very same system call that typical automation tools like \rbox{latexmk} and \rbox{rubber} use. Observe that \arara\ works practically on other side of the spectrum: you need to tell it how and when to do a task.

\section{Core concepts}
\label{sec:coreconcepts}

When adding a directive in our source code, we are explicitly telling the tool what we want it to do, but I am afraid that is not sufficient at all. So far, \arara\ knows \emph{what} to do, but now it needs to know \emph{how} the task should be done. If we want \arara\ to run \rbox{pdflatex} on \rbox{hello.tex}, we need to have instructions telling our tool how to run that specific application. This particular sequence of instructions is referred as a \emph{rule} in our context. 

\begin{messagebox}{Note on rules}{attentioncolour}{\icattention}{black}
Although the core team provides a lot of rules shipped with \arara\ out of the box, with the possibility of extending the set by adding more rules, some users might find this decision rather annoying, since other tools have most of their rules hard-coded, making the automation process even more transparent. However, since \arara\ does not rely on a specific automation or compilation scheme, it becomes more extensible. The use of directives in the source code make the automation steps more fluent, which allows the specification of complex workflows very easy.
% "very easy" -> "much easier" perhaps?
\end{messagebox}

Despite the inherited verbosity of automation steps not being suitable for small documents, \arara\ really shines when you have a document which needs full control of the automation process (for instance, a thesis or a manual). Complex workflows are easily tackled by our tool.

Rules and directives are the core concepts of \arara: the first dictates how a task is done, and the latter is the proper instance of the associated rule on the current document, i.e, when and where the commands must be executed.

\begin{messagebox}{The name}{araracolour}{\icok}{white}
\begin{minipage}{0.45\textwidth}
\vspace{.8em}
{\centering\includegraphics[width=0.9\textwidth]{figures/arara.png}\par}

\vspace{.7em}

\em Do you like araras? We do, specially our tool which shares the same name of this colorful bird.
\end{minipage}\hspace{1em}
\begin{minipage}{0.5\textwidth}
The tool name was chosen as an homage to a Brazilian bird of the same name, which is a macaw. The word \emph{arara} comes from the Tupian word \emph{a'rara}, which means \emph{big bird} (much to my chagrin, Sesame Street's iconic character Big Bird is not a macaw; according to some sources, he claims to be a golden condor). Araras are colorful, noisy, naughty and very funny. Everybody loves araras. The name seemed catchy for a tool and, in the blink of an eye, \arara\ was quickly spread to the whole \TeX\ world.
\end{minipage}
\end{messagebox}

Now that we informally introduced rules and directives, let us take a look on how \arara\ actually works given those two elements. The whole idea is pretty straightforward, and I promise to revisit these concepts later on in this manual for a comprehensive explanation (more precisely, in Chapter~\ref{chap:importantconcepts}).

First and foremost, we need to add at least one instruction in the source code to tell \arara\ what to do. This instruction is named a \emph{directive} and it will be parsed during the preparation phase. Observe that \arara\ will tell you if no directive was found in a file, as seen in our first interaction with the tool.

An \arara\ directive is usually defined in a line of its own, started with a comment (denoted by a percent sign in \TeX\ and friends), followed by the word \rbox{arara:} and task name:

\begin{codebox}{A typical directive}{teal}{\icnote}{white}
% arara: pdflatex
\documentclass{article}
...
\end{codebox}

Our example has one directive, referencing \rbox{pdflatex}. It is important to observe that the \rbox{pdflatex} identifier \emph{does not represent the command to be executed}, but \emph{the name of the rule associated with that directive}.

\begin{messagebox}{New feature in version 4.0}{araracolour}{\icinfo}{white}
\textbf{Multiline directives} -- Later on, in Section~\ref{sec:directives}, on page~\pageref{sec:directives}, we will discover that a directive can also span several lines in order to provide a better code organization. For now, let us assume a typical directive occupies only one line.
\end{messagebox}

Once \arara\ finds a directive, it will look for the associated \emph{rule}. In our example, it will look for a rule named \rbox{pdflatex} which will evidently run the \rbox{pdflatex} command line application. Rules are \gls{YAML} files named according to their identifiers followed by the \rbox{yaml} extension and follow a strict structure. This concept is covered in Section~\ref{sec:rule}, on page~\pageref{sec:rule}.

\begin{messagebox}{New feature in version 4.0}{araracolour}{\icattention}{white}
\textbf{\gls{REPL} workflow} -- \arara\ now employs a \gls{REPL} workflow for rules and directives. In previous versions, directives were extracted, their corresponding rules were analyzed, commands were built and added to a queue before any proper execution or evaluation. I decided to change this workflow, so now \arara\ evaluates each rule on demand, i.e, there is no \emph{a priori} checking. A rule will \emph{always} reflect the current state, including potential side effects from previous executed rules.
\end{messagebox}

Now, we have a queue of pairs $(\textit{directive}, \textit{rule})$ to process. For each pair, \arara\ will map the directive to its corresponding rule, evaluate it and run the proper command. The execution chain requires that command $i$ was successfully executed to then proceed to command $i+1$, and so forth. This is also by design: \arara\ will halt the execution if any of the commands in the queue had raised an error. How does one know if a command was successfully executed? \arara\ checks the corresponding \emph{exit status} available after a command execution. In general, a successful execution yields 0 as its exit status.

\begin{messagebox}{New feature in version 4.0}{araracolour}{\icinfo}{white}
\textbf{Custom exit status checking} -- In previous versions, there was no way of customizing the exit status checking of a command. A command was successful if, and only if, its resulting exit status was 0 and no other value. From now on, we can define any value, or even forget about it and make it always return a valid status regardless of execution (for instance, in a rule that always is successful -- see, for instance, the \rbox{clean}  rule).
\end{messagebox}

That is pretty much how \arara\ works: directives in the source code are mapped to rules. These pairs are added to a queue. The queue is then executed and the status is reported. More details about the expansion process are presented in Chapter~\ref{chap:importantconcepts}, on page~\pageref{chap:importantconcepts}. In short, we teach \arara\ to do a task by providing a rule, and tell it to execute it through directives in the source code.

\section{Operating system remarks}
\label{sec:operatingsystemremarks}

The application is written using the Java language, so \arara\ runs on top of a Java virtual machine, available on all the major operating systems~--~in some cases, you might need to install the proper virtual machine. We tried very hard to keep both code and libraries compatible with older virtual machines or from other vendors. Currently, \arara\ is known to run on Oracle's Java 5 to 10, and OpenJDK 5 to 10. We also have reports of users successfully using the tool with virtual machines provided by Azul Systems, so your mileage might vary greatly.

\begin{messagebox}{Outdated Java virtual machines}{attentioncolour}{\icerror}{black}
Dear reader, beware of outdated software, mainly Java virtual machines! Although \arara\ offers support for older virtual machines, try your best to keep your software updated as frequently as possible. The legacy support exists only for historical reasons, and also due to the sheer fact that we know some people that still runs \arara\ on very old hardware. If you are not in this particular scenario, get the latest virtual machine.
\end{messagebox}

In Chapter~\ref{chap:buildingfromsource}, on page~\pageref{chap:buildingfromsource}, we provide instructions on how to build \arara\ from sources using Apache Maven. Even if you use multiple operating systems, \arara\ should behave the same, including the rules. There are helper functions available in order to provide support for system-specific rules based on the underlying operating system.

\section{Support}
\label{sec:support}

If you run into any issue with \arara, please let us know. We all have very active profiles in the \href{https://tex.stackexchange.com/}{\TeX\ community at StackExchange}, so just use the \rbox[araracolour]{arara} tag in your question and we will help you the best we can (also, take a look at their \href{https://tex.meta.stackexchange.com/q/1436}{starter guide}).  We also have a \href{https://gitter.im/cereda/arara}{Gitter chat room}, in which we occasionally hang out. Also, if you think the report is worthy of an issue, open one in our \href{https://github.com/cereda/arara/issues}{GitHub repository}. And last, but not least, feel free to poke us by good old electronic mail (please try the other approaches first).

We really hope you like our humble contribution to the \TeX\ community. Let \arara\ enhance your \TeX\ experience, it will help you when you will need it the most. Enjoy the manual.

% !TeX root = ../arara-manual.tex
\chapter{Important concepts}
\label{chap:importantconcepts}

Time for our first contact with \arara! I must strees that is very important to understand a few concepts in which \arara\ relies before we proceed to the usage itself. Do not worry, these concepts are easy to follow, yet they are vital to the comprehension of the application and the logic behind it.

\section{Rules}
\label{sec:rule}

A \emph{rule} is a formal description of how \arara\ handles a certain task. For instance, if we want to use \abox{pdflatex} with our tool, we should have a rule for that. Directives are mapped to rules, so a call to a nonexistent rule \abox{foo}, for instance, will indeed raise an error:

\begin{codebox}{Terminal}{teal}{\icnote}{white}
  __ _ _ __ __ _ _ __ __ _ 
 / _` | '__/ _` | '__/ _` |
| (_| | | | (_| | | | (_| |
 \__,_|_|  \__,_|_|  \__,_|

Processing 'doc1.tex' (size: 83 bytes, last modified: 05/03/2018
12:10:33), please wait.

I could not find a rule named 'foo' in the provided rule paths.
Perhaps a misspelled word? I was looking for a file named
'foo.yaml' in the following paths in order of priority:
(/opt/paulo/arara/rules)

Total: 0.09 seconds
\end{codebox}

% TODO fix reference
Once a rule is defined, \arara\ automatically provides an access layer to that rule through directives in the source code, a concept to be formally introduced later on, in Section~\ref{foo}. Observe that a directive reflects a particular instance of a rule of the same name (i.e, a \abox{foo} directive in a certain source code is an instance of the \abox{foo} rule).

In short, a rule is a plain text file written in the YAML format, introduced in Chapter~\ref{foo} (page~\pageref{foo}). I opted for this format because back then it was cleaner and more intuitive to use than other markup languages such as XML, besides of course being a data serialization standard for programming languages.

\begin{messagebox}{Animal jokes}{araracolour}{\icok}{white}
As a bonus, the acronym \emph{YAML} rhymes with the word \emph{camel}, so \arara\ is heavily environmentally friendly. Speaking of camels, there is the programming reference as well, since this amusing animal is usually associated with Perl and friends.
\end{messagebox}

% TODO fix reference
The default rules, that is, the rules shipped with \arara, are placed inside a special subdirectory named \abox[araracolour]{rules/} inside another special directory named \abox[araracolour]{ARARA\_HOME} (the place where our tool is installed). We will learn later on, in Section~\ref{foo} (page~\pageref{foo}), that we can add an arbitrary number of paths for storing our own rules, in order of priority, so do not worry too much with the location of the default rules, although it is important to understand and acknowledge their existance.

The following list describes the basic structure of an \arara\ rule by presenting the proper elements (or keys, if we consider the proper YAML nomenclature). Observe that elements marked as \rbox[araracolour]{M} are mandatory (i.e, the rule \emph{has} to have them in order to work). Similarly, elements marked as \rbox[araracolour]{O} are optional, so you can safely ignore them when writing a rule for our tool. A key preceded by \rbox{context$\rightarrow$} indicates a context and should be properly defined inside it.

\begin{description}
\item[\describe{M}{!config}] This keyword is mandatory and must be the first line of any \arara\ rule. It denotes the object mapping metadata to be internally used by the tool. Actually, the tool is not too demanding on using it (in fact, you could suppress it entirely and \arara\ will not complain), but it is considered good practice to start all rules with a \abox{!config} keyword regardless.

\item[\describe{M}{identifier}] This key acts as a unique identifier for the rule (as expected). It is highly recommended to use lowercase letters without spaces, accents or punctuation symbols, as good practice (again). As a convention, if you have an identifier named \abox{pdflatex}, the rule filename must be \abox{pdflatex.yaml} (like our own instance). Please note that, although \abox{.yml} is known to be a valid YAML extension as well, \arara\ only considers files ending with the \abox{.yaml} extension. This is a deliberate decision.

\begin{codebox}{Example}{teal}{\icnote}{white}
identifier: pdflatex
\end{codebox}

\item[\describe{M}{name}] This key holds the name of the task as a plain string. When running \arara, this value will be displayed in the output. We like to call \emph{task} an instantiated rule (through a directive). Task names are displayed enclosed in parenthesis.

\begin{codebox}{Example}{teal}{\icnote}{white}
name: PDFLaTeX
\end{codebox}

\item[\describe{O}{authors}] We do love blaming people, so \arara\ features a special key to name the rule authors (if any) so you can write stern electronic communications to them! This key holds a list of strings. If the rule has just one author, add it as the first (and only) element of the list.

\begin{codebox}{Example}{teal}{\icnote}{white}
authors:
- Marco Daniel
- Paulo Cereda
\end{codebox}

\item[\describe{M}{commands}] This key is introduced in version 4.0 of \arara\ and denotes a potential list of commands. From the user perspective, each command is called \emph{subtask} within a task (rule and directive) context. A task may represent only a single command (a single subtask), as well as a sequence of commands (subtasks). For instance, the \abox{frontespizio} rule requires at least two commands. So, as a means of normalizing the representation, a task composed of a single command (single subtask) is defined as the only element of such list, as opposed to previous versions of \arara, which had an specific key to hold just one command.

\begin{messagebox}{Incompatibility with older versions}{attentioncolour}{\icerror}{black}
Dear reader, note that rules from version 4.0 are incompatible with older versions of \arara. If you are migrating from old versions to version 4.0, we need to replace \abox{command} by \abox{commands} and specifying a contextual element, as seen in the following descriptions. Please refer to Section~\ref{foo} (page~\pageref{foo}) for a comprehensible migration guide.
\end{messagebox}

In order to properly set a subtask, the keys used in this specification are defined inside the \rbox{commands$\rightarrow$} context and presented as follows.

\begin{description}
\item[\describecontext{O}{commands}{name}] This key holds the name of the subtask as a plain string. When running \arara, this value will be displayed in the output. Subtask names are displayed after the main task name. By the way, did you notice that this key is entirely optional? That means that a subtask can simply be unnamed, if you decide so. However, such practice is not recommended, as is always good to have a visual description of what \arara\ is running at the moment, so name your subtasks properly.

\item[\describecontext{M}{commands}{command}] This key holds the action to be performed, typically a system command. In previous versions, \arara\ would rely solely on a string. For this version on, as a means to enhance the user experience (and also fix serious blockers when handling spaces in file names, as seen in \href{https://github.com/cereda/arara/issues}{previous issues} reported in the repository), the tool offers four types of returned values:

\begin{itemize}[label={--}]
\item A plain string: this is the default (and only) behaviour in older versions of \arara. The plain string is processed as it is by the underlying execution engine. However, automatic argument parsing poses as a complex issue, so this approach, although supported, is not recommended anymore.

\begin{codebox}{Example}{teal}{\icnote}{white}
command: 'ls'
\end{codebox}

% TODO fix reference
It is important to observe that you can use either a plain string directly or using an orb tag with an explicit \abox{return} command (as seen in Section~\ref{foo}, page~\pageref{foo}). Personally, I favour the explict indication for a quick understanding.

% TODO fix reference
\item A \abox{Command} object: \arara\ 4.0 features a new approach for handling system commands based on a high level structure with explict argument parsing named \abox{Command} (for our curious users, it is a plain Java object). In order to use this approach, we need to rely on orb tags and use a helper method named \mtbox{getCommand} to obtain the desired result. We will detail this method later on, in Section~\ref{foo} (page~\pageref{foo}). We highly recommend the adoption of this approach for rule writing instead of using plain strings.

\begin{codebox}{Example}{teal}{\icnote}{white}
command: "@{ return getCommand('ls') }"
\end{codebox}

% TODO fix reference
\item A boolean value: it is also possible to exploit the expressive power of the underlying scripting language available in the rule context (see Chapter~\ref{foo}, in page~\pageref{foo}, for more details) for writing complex code. In this particular case, since the computation is being done by \arara\ itself and not the underlying operating system, there will not be a command to be executed, so simply return a boolean value -- either an explicit \abox{true} or \abox{false} value or a logical expression -- to indicate whether the computation was successfull.

\begin{codebox}{Example}{teal}{\icnote}{white}
command: "@{ return 1 == 1 }"
\end{codebox}

\item A \abox{Trigger} object: this is surely the least common type of returned value and it is mentioned here just for documentation purposes. In simple terms, a \abox{Trigger} object constitutes a special command that changes the internal workings of \arara\ at runtime. We have not worked much on this concept, so there is only one trigger available, seen in action in the official \abox{halt} rule. In order to use this approach, we need to rely on orb tags and use a helper method named \mtbox{getTrigger} to obtain the desired result.
\end{itemize}

It is also worth mentioning that \arara\ also supports lists of commands represented as plain strings, \abox{Command} or \abox{Trigger} objects, boolean values or a mix of them. This is useful if your rule has to decide whether more actions are required in order to accomplish a task. In this case, our tool will take care of the list and execute each element in the specified order.

\begin{codebox}{Example}{teal}{\icnote}{white}
command: "@{ return [ 'ls', 'ls', 'ls' ] }"
\end{codebox}

As an example, please refer to the official \abox{clean} rule for a real scenario where a list of commands is successfully employed: for each provided extension, the rule creates a new cleaning command and adds it to a list of removals to be processed later.

\begin{messagebox}{Plain string is deprecated}{attentioncolour}{\icattention}{black}
It took me a lot of effort to find out that handling plain strings and employing guesswork to parse arguments are the root of several issues reported by users. Therefore, this approach is being marked as \emph{deprecated} and will be removed in future versions.
\end{messagebox}

% TODO fix reference
There are at least two variables available in the \abox{command} context and are described as follows (note that MVEL variables and orb tags are discussed in Chapter~\ref{foo}). A variable will be denoted by \varbox{variable} in this list. For each rule argument (defined later on), there will be a corresponding variable in the \abox{command} context, directly accessed through its unique identifier.

\begin{description}
\item[\varbox{file}] This variable holds the file name, without any path reference, as a plain string. It is usually composed by the base name and the extension. This variable is available since the first release of \arara.

\item[\varbox{reference}] This variable is introduced in version 4.0 of \arara\ and holds the canonical, absolute path representation of the \varbox{file} variable as a \abox{File} object. This is useful if there is a need of understanding the hierarchical structure of a project. Since the reference is a Java object, we can use all methods available in the \abox{File} class.
\end{description}

\begin{messagebox}{Quote handling}{araracolour}{\icinfo}{white}
\setlength{\parskip}{1em}
The YAML format disallows key values starting with \abox{@} without proper quoting. This is the reason we had to use double quotes for the value and internally using single quotes for the command string. Also, we could use the other way around, or even using only one type and then escaping them when needed. This is excessively verbose but needed due to the format requirement. Thankfully, \arara\ offers two solutions for removing the quoting verbosity when writing commands.

The first solution is used in previous versions and it still works like a charm in modern days. We need to precede our command with a special keyword \abox{<arara>} which will be removed afterwards. This solution works on virtually every key in the rule context, so it is a bonus. The new code will look like this:

\begin{codebox}{Example}{teal}{\icnote}{white}
command: <arara> @{ return getCommand('ls') }
\end{codebox}

% TODO fix reference
The second approach is more of a YAML feature rather than a tool exclusive, although we have to do a couple of checkings under the hood in order to ensure the correct execution. The idea here is to use the scalar content in folded style, as seen in Section~\ref{foo} (page~\pageref{foo}). The new code will look like this:

\begin{codebox}{Example}{teal}{\icnote}{white}
command: >
  @{
    return getCommand('ls')
  }
\end{codebox}

Mind the indentation, as YAML requires it to properly identify blocks. I personally recommend this approach for longer code, as it provides a better visual representation. You will see the second solution all around the default rules, but feel free to use the one you feel more comfortable.
\end{messagebox}

\item[\describecontext{O}{commands}{exit}] This key holds a special purpose in \arara\ 4.0, as it represents a custom exit status evaluation for the corresponding command. In general, a successful execution has zero as an exit status, but sometimes we end up with tools or situations that we need to override this checking for whatever reason. For this purpose, simply write a MVEL expression \emph{without orb tags} as plain string and use the special variable \varbox{value} if you need the actual exit status returned by the command, available at runtime. For example, if the command returns a non-zero value indicating a successful execution, we can write this key as:

\begin{codebox}{Example}{teal}{\icnote}{white}
exit: value > 0
\end{codebox}

If the execution should be marked as successful by \arara\ regardless of the actual exit status, you can simply write \abox{true} as the key value and this rule will never fail, for obvious reasons.
\end{description}

For instance, consider a full example of the \abox{commands} key, defined with only one command, presented as follows. The hyphen denotes a list element, so mind the indentation for correctly specifying the component keys. Also, note that, in this case, the \abox{exit} key was completely optional, as it does the default checking, and it was included for didactic purposes.

\begin{codebox}{Example}{teal}{\icnote}{white}
commands:
- name: The PDFLaTeX engine
  command: >
    @{
      return getCommand('pdflatex', file)
    }
  exit: value == 0
\end{codebox}

\item[\describe{M}{arguments}] This key holds a list of arguments for the current rule, if any. The arguments specified in this list will be available to the user later on for potential completion through directives. Once instantiated, they will become proper variables in the \abox{command} contexts. This key is mandatory, so even if your rule does not have arguments, you need to specify a list regardless. In this case, use the empty list notation:

\begin{codebox}{Example}{teal}{\icnote}{white}
arguments: []
\end{codebox}

In order to properly set an argument, the keys used in this specification are defined inside the \rbox{arguments$\rightarrow$} context and presented as follows.

\begin{description}
\item[\describecontext{M}{arguments}{identifier}] This key acts as a unique identifier for the argument. It is highly recommended to use lowercase letters without spaces, accents or punctuation symbols, as a good practice. This key will be used later on to set the corresponding value in the directive context.

\begin{codebox}{Example}{teal}{\icnote}{white}
identifier: shell
\end{codebox}

\item[\describecontext{O}{arguments}{flag}] This key holds a plain string and is evaluated when the corresponding argument is defined in the directive context.  After being evaluated, the result will be stored in a variable of the same name to be later accessed in the \abox{command} context. In the scenario where the argument is not defined in the directive, the variable will hold an empty string.

\begin{codebox}{Example}{teal}{\icnote}{white}
flag: >
  @{
      isTrue(parameters.shell, '--shell-escape',
             '--no-shell-escape')
  }
\end{codebox}

% TODO fix reference
There is one variable available in the \abox{flag} context and is described as follows. Note that are also several helper methods available in the rule context (for instance, \mtbox{isTrue} presented in the previous example) which provide interesting features for rule writing. They are detailed later on, in Section~\ref{foo} (page~\pageref{foo}).

\begin{description}
\item[\varbox{parameters}] This variable holds a map of directive parameters available at runtime. For each argument identifier listed in the \abox{arguments} list in the rule context, there will be an entry in this variable. This is useful to get the actual values provided during execution and take proper actions. If a parameter is not set in the directive context, the reference will still exist in the map, but it will be mapped to an empty string.
\end{description}

In the previous example, observe that the MVEL expression defined in the \abox{flag} key checks if the user provided an affirmative value regarding shell escape, through comparing \varbox{parameters.shell} with a set of predefined affirmative values. In any case, the corresponding command flag is defined as result of such evaluation.

\item[\describecontext{O}{arguments}{default}] As default behaviour, if a parameter is not set in the directive context, the reference will be mapped to an empty string. This key exists for the exact purpose of overriding such behaviour and expects a plain string as value.

\begin{codebox}{Example}{teal}{\icnote}{white}
default: ''
\end{codebox}

\item[\describecontext{O}{arguments}{required}] There might be certain scenarios in which a rule could make use of required arguments (for instance, a copy operation in which source and target must be provided). The \abox{required} key acts as a boolean switch to indicate whether the corresponding argument should be mandatory. In this case, set the key value to \abox{true} and the argument turns required. Later on at runtime, \arara\ will throw an error if a required parameter is missing in the directive.

\begin{codebox}{Example}{teal}{\icnote}{white}
required: false
\end{codebox}

Note that setting the \abox{required} key value to \abox{false} corresponds to omitting the key completely in the rule context, which resorts to the default behaviour (i.e, all arguments are optional).
\end{description}

\begin{messagebox}{Note on argument keys}{attentioncolour}{\icattention}{black}
As seen previously, both \abox{flag} and \abox{default} are marked as optional, but at least one of them must occur in the argument specification, otherwise \arara\ will throw an error, as it makes no sense to have no argument handling at all. Please make sure to specify at least one of them!
\end{messagebox}

For instance, consider a full example of the \abox{arguments} key, defined with only one argument, presented as follows. The hyphen denotes a list element, so mind the indentation for correctly specifying the component keys. Also, note that, in this case, keys \abox{required} and \abox{default} were completely optional, and they were included for didactic purposes.

\begin{codebox}{Example}{teal}{\icnote}{white}
arguments:
- identifier: shell
  flag: >
    @{
        isTrue(parameters.shell, '--shell-escape',
               '--no-shell-escape')
    }
  required: false
  default: ''
\end{codebox}
\end{description}

% TODO fix reference
This is the rule structure in the YAML format used by \arara. Keep in mind that all subtasks in a rule are checked against their corresponding exit status. If an abnormal execution is detected, the tool will instantly halt and the rule will fail. Even \arara\ itself will return an exit code different than zero when this situation happens (detailed in Section~\ref{foo}, in page~\pageref{foo}).

\section{Directives}
\label{sec:directives}

%A \emph{directive} is a special comment inserted in the
%source file in which you indicate how \arara\ should
%behave. You can insert as many directives as you
%want and in any position of the file. The tool will
%read the whole file and extract the directives.
%
%\begin{messagebox}{New features in version 4.0}{araracolour}{\icinfo}{white}
%\textbf{Partial directive extraction} -- From version 4.0 on,
%it is now possible to extract directives only available in the
%file preamble, that is, all lines from the beginning that are
%comments until reaching the first line that is not a comment.
%To this end, a new command line flag is introduced. We will
%discuss this feature later on.
%
%\vspace{1em}
%
%\textbf{Predefined preambles} -- It is now possible to set up
%a common preamble to be used with files that require the same
%automation steps, then \arara\ can be invoked based on such
%specifications. We will discuss this feature later on.
%\end{messagebox}
%
%There are two types of directives in \arara. The first one has
%already been mentioned, it has only the rule name (which refers 
%to the \verb|identifier| key from the rule of the same name). It 
%is called \emph{empty directive}:
%
%\begin{codebox}{Empty directive}{teal}{\icnote}{white}
%% arara: pdflatex
%\end{codebox}
%
%Sometimes, however, we need to provide additional information to 
%the rule. That is reason for the second type, the 
%\emph{parametrized directive}, to exist. As the name indicates, 
%we have directive arguments! They are mapped by their identifiers
%and not by their positions. The syntax for a parametrized 
%directive is:
%
%\begin{codebox}{Parametrized directive}{teal}{\icnote}{white}
%% arara: pdflatex: { shell: yes }
%\end{codebox}
%
%Each argument is defined according to the rule mapped by the 
%directive. This means you cannot use an argument \verb|foo| in a 
%directive \verb|bar| if the rule \verb|bar| does not offer 
%support for it (that is, \verb|bar| has to have \verb|foo| 
%defined as argument in its list of arguments inside the rule 
%scope, as seen in the previous section). The syntax for
%an argument is:
%
%\begin{codebox}{Argument syntax}{teal}{\icnote}{white}
%key : value
%\end{codebox}
%
%Suppose we would like to enable shell escape for \verb|pdflatex| 
%when compiling a \verb|hello.tex| file. We can achieve that by 
%providing a parametrized directive, like this one:
%
%\begin{codebox}{\texttt{hello.tex} with a parametrized directive}{teal}{\icnote}{white}
%% arara: pdflatex: { shell: yes }
%
%\documentclass{article}
%\begin{document}
%Hello world!
%\end{document}
%\end{codebox}
%
%Of course, the \verb|shell| argument is defined in the
%\verb|pdflatex| rule scope, otherwise \arara\ would raise
%an error about an invalid argument key. If we try to
%inject a nonexistent \verb|foo| argument in the previous 
%parametrized directive, we will get this message:
%
%\begin{codebox}{Terminal}{teal}{\icnote}{white}
%  __ _ _ __ __ _ _ __ __ _ 
% / _` | '__/ _` | '__/ _` |
%| (_| | | | (_| | | | (_| |
% \__,_|_|  \__,_|_|  \__,_|
%
%Processing 'hello.tex' (size: 103 bytes, last modified:
%05/03/2018 15:40:16), please wait.
%
%I have spotted an error in rule 'pdflatex' located at
%'/opt/paulo/arara/rules'. I found these unknown keys
%in the directive: (foo). This should be an easy fix,
%just remove them from your map.
%
%Total: 0.21 seconds
%\end{codebox}
%
%As the message suggests, we need to remove the unknown argument 
%key from our directive or rewrite the rule in order to include 
%it. The first option is, of course, easier.
%
%\begin{messagebox}{Helpful messages}{araracolour}{\icinfo}{white}
%Make sure to read all messages \arara\ raises, they will help 
%you!
%\end{messagebox}
%
%Sometimes, directives can span several columns of a line, 
%particularly the ones with several arguments. From \arara\ 4.0 
%on, we can split a directive into multiple lines by adding
%\verb|% arara: -->| to each line which should compose the
%directive:
%
%\begin{codebox}{Multiline directive}{teal}{\icnote}{white}
%% arara: pdflatex: {
%% arara: --> shell: yes,
%% arara: --> synctex: yes
%% arara: --> }
%\end{codebox}
%
%It is important to observe that there is no need of them to be
%in contiguous lines, that is, provided that the syntax for
%parametrized directives hold for the line composition, lines can
%be distributed all over the code.
%
%\begin{messagebox}{New feature in version 4.0}{araracolour}{\icinfo}{white}
%\textbf{Conditionals} -- From version 4.0 on, \arara\ provides
%logical expressions processed at runtime to determine whether
%and  how a directive should be processed. This is a huge 
%improvement towards better user experience.
%\end{messagebox}
%
%One of the most awaited features that version 4.0 introduces is 
%the support of conditionals, that is, logical expressions 
%processed at runtime in order to determine whether and how the 
%directive should be processed. The following types are allowed:
%
%\begin{keywords}
%\item[if] evaluated beforehand, the directive is interpreted
%if and only if the result is true.
%
%\item[unless] evaluated beforehand, the directive
%is interpreted if and only if the result is false.
%
%\item[until] directive is interpreted the first time,
%then the evaluation is done; while the result
%is false, the directive is interpreted again and
%again.
%
%\item[while] evaluated beforehand, the directive is
%interpreted if and only if the result is true,
%and the process is repeated while the result
%still holds true.
%\end{keywords}
%
%We will discuss this special feature later on, including methods
%available in the directive scope in order to ease the writing
%of conditionals, as it would be highly advisable to have
%orb tags covered beforehand.
%
%\section{Orb tags}
%\label{sec:orbtags}


%
%Time for our first contact with \arara! I must strees that
%is very important to understand a few concepts in which
%\arara\ relies before we proceed to the usage itself.
%Do not worry, these concepts are easy to follow, yet they
%are vital to the comprehension of the application and the
%logic behind it.
%
%\section{Rules}
%
%A \emph{rule} is a formal description of how \arara\
%handles a certain task. For instance, if we want to use
%\verb|pdflatex| with our tool, we should have a rule for
%that. Directives are mapped to rules, so a call to a
%nonexistent rule \verb|foo|, for instance, will
%indeed raise an error:
%
%\begin{codebox}{Terminal}{teal}{\icnote}{white}
%$ arara doc1.tex
%  __ _ _ __ __ _ _ __ __ _ 
% / _` | '__/ _` | '__/ _` |
%| (_| | | | (_| | | | (_| |
% \__,_|_|  \__,_|_|  \__,_|
%
%Processing 'doc1.tex' (size: 83 bytes, last modified: 05/03/2018
%12:10:33), please wait.
%
%I could not find a rule named 'foo' in the provided rule paths.
%Perhaps a misspelled word? I was looking for a file named
%'foo.yaml' in the following paths in order of priority:
%(/opt/paulo/arara/rules)
%
%Total: 0.09 seconds
%\end{codebox}
%
%Once a rule is defined, \arara\ automatically provides an
%access layer to that rule through directives in the source
%code, a concept to be formally introduced in the following
%section. Observe that a directive reflects a particular
%instance of a rule of the same name.
%
%A rule is a plain text file written in the YAML format.
%I opted for this format because back then it cleaner
%and more intuitive to use than other markup languages,
%besides of course being a data serialization standard for programming languages.
%
%\begin{messagebox}{Animal jokes}{araracolour}{\icok}{white}
%As a bonus, the acronym \emph{YAML} rhymes with the word
%\emph{camel}, so \arara\ is heavily environmentally friendly.
%Speaking of camels, I could rewrite the tool in Perl, but I
%would rather not rage at my editor right now.
%\end{messagebox}
%
%The default rules, that is, the rules shipped with \arara, are
%placed inside a special subdirectory named \verb|rules/| inside
%\verb|ARARA_HOME|. We will learn later on that we can add an
%arbitrary number of paths for storing our own rules, in order
%of priority, so do not worry too much with the location of the
%default rules, although it is important to understand and
%acknowledge their existance. The basic structure of an
%\arara\ rule is:
%
%\begin{codebox}{Basic rule struture}{teal}{\icnote}{white}
%!config
%# Hello, I am a comment
%identifier: pdflatex
%name: PDFLaTeX
%commands:
%- name: PDFLaTeX engine
%  command: pdflatex @{file}
%arguments: []
%\end{codebox}
%
%Let us break down the structure into parts, so it will be easier for us to grasp the elements, hopefully. The indices correspond to the line numbering scheme in the previous code.
%
%\begin{linedescription}
%\item The \verb|!config| keyword is mandatory and must be
%the first line of any \arara\ rule. It denotes the object
%mapping metadata to be internally used by the tool. 
%
%\item A comment line starts with the \verb|#| symbol.
%
%\item The \verb|identifier| key acts as a unique identifier
%for the rule (as expected). It is highly recommended to use
%lowercase letters without spaces, accents or punctuation symbols.
%As a convention, if you have an identifier named \verb|pdflatex|,
%the rule filename must be \verb|pdflatex.yaml| (like our own
%instance).
%
%\item The \verb|name| key holds the name of the task. When
%running \arara, this value will be displayed in the output.
%In our example, the tool will display \verb|PDFLaTeX| as task
%name in the output when dealing with this task. Task names are
%displayed enclosed in parenthesis.
%
%\item The \verb|commands| key is introduced in version 4.0 of
%\arara\ and denotes a potential list of subtasks. A task may
%represent only a single command, as well as a sequence of
%commands (for example, the \verb|frontespizio| rule requires
%at least two commands). So, as a means of normalizing the
%representation, a task composed of a single command (like
%the one in our example) is defined as the only element
%of such list. The keys used inside this list specification
%are defined as follows. A list element is denoted by \verb|-|
%(hyphen).
%
%\item The \verb|name| key has no relation with the previously
%presented key of the same name. In this specific context,
%this key holds the subtask name. When running \arara\, this
%value will be displayed in the output right after the
%task name.
%
%\item The \verb|command| key contains the system command to be
%executed. You can use virtually any type of command,
%interactive or noninteractive. But beware: if \arara\ is
%running in silent mode, which is the default behaviour,
%an interactive command wich might require the user
%input will be halted and the execution will fail.
%Do not despair, you can use a special \verb|--verbose|
%flag with \arara\ in order to interact with such
%commands -- we will talk about flags later on. You probably
%noticed a strange element \verb|@{file}| in the
%\verb|command| line: this element is called \emph{orb tag}.
%For now, just admit these elements exist. We will come back
%to them later on, I promise.
%
%\item The \verb|arguments| key denotes a list of arguments
%for the subtask command. In our example, we have an empty
%list, denoted by \verb|[]|. You can define as many arguments
%as your subtask requires.
%\end{linedescription}
%
%For more complex rules, we might want to use arguments.
%The following code presents a new rule which makes use
%of them instead of an empty list as the previous code:
%
%\begin{codebox}{Adding rule arguments}{teal}{\icnote}{white}
%!config
%identifier: copy
%name: Copy
%commands:
%- name: Copy operation
%  command: copy @{from} @{to}
%arguments:
%- identifier: from
%  flag: '@{parameters.from}'
%- identifier: to
%  flag: '@{parameters.to}'
%\end{codebox}
%
%For every argument in the \verb|arguments| list, we have a
%\verb|-| mark (hyphen) and proper indentation. Let us break down the relevant parts of this new rule:
%
%\begin{linedescription}[start=8]
%\item The \verb|identifier| key acts as a unique identifier
%for the current argument. It is highly recommended to use
%lowercase letters without spaces, accents or punctuation
%symbols.
%
%\item The \verb|flag| key represents the argument value.
%Note that we have other orb tags in the argument
%definitions, \verb|@{parameters.from}| and
%\verb|@{parameters.to}|; we will discuss them later on. Just
%to give some context, \verb|parameters| is a special variable
%which maps the elements available in the directive being
%evaluated. For example, if we have \verb|from: a| in a
%directive, \verb|parameters.from|  will resolve to \verb|a|.
%The argument \verb|flag| value is only applied if, and only
%if, there is an explicit directive argument. Say, if
%\verb|from| is not defined as a directive argument, the
%\verb|flag| value of argument \verb|from| will be
%resolved to an empty string.
%\end{linedescription}
%
%\begin{messagebox}{Overriding the default resolution of nonexistent arguments}{araracolour}{\icinfo}{white}
%If a certain argument does not exist in the directive, its
%rule counterpart will be resolved to an empty string, as the
%default resolution. As a means to overriding this behaviour
%when a directive argument is not specified, use the \verb|default| key within the argument specification.
%\end{messagebox}
%
%When a rule argument just needs a default value regardless of
%a user-specified value, you can safely ignore the \verb|flag|
%key and rely on the \verb|default| one. Similarly, if you need
%to map a directive argument into a rule argument without falling 
%back to a default value different than an empty string, 
%just use the \verb|flag| key. Now, if you need mapping and 
%fallback, stick with both keys. Just keep in mind that at least 
%one of them must exist!
%
%\begin{messagebox}{New feature in version 4.0}{araracolour}{\icinfo}{white}
%\textbf{Required arguments} -- Sometimes, we might end up with
%a rule in need of mandatory arguments. For instance, the
%two arguments of our \verb|copy| should be mandatory, as it makes
%no sense to have optional values for a copy operation. In this
%case, add a \verb|required| key to the relevant argument and set 
%the corresponding value to \verb|true| (a boolean value). Setting
%such value to \verb|false| would fallback to the default
%behaviour.
%\end{messagebox}
%
%All subtasks in a rule are checked against their corresponding
%exit status. If an abnormal execution is detected, \arara\ will
%instantly halt.
%
%For now, we need to keep in mind that \arara\ uses rules to tell
%it how to do a certain task (and subtasks). In the next sections,
%when more concepts are presented, we will come back to this
%subject. Just a taste of things to come, as we mentioned before
%already: directives are mapped to rules through orb tags. Do not
%worry, I will explain how things work.
%
%\begin{messagebox}{There are better ways of writing a rule}{attentioncolour}{\icattention}{black}
%This section covered the simplest way of writing an \arara\ rule,
%solely for didactic (and also historical) reasons. Keep in mind that there are far better and ways to achieve consistent rules. We
%need to discuss more about the basics before entering into
%advanced topics.
%\end{messagebox}
%
%\section{Directives}
%\label{sec:directives}
%
%A \emph{directive} is a special comment inserted in the
%source file in which you indicate how \arara\ should
%behave. You can insert as many directives as you
%want and in any position of the file. The tool will
%read the whole file and extract the directives.
%
%\begin{messagebox}{New features in version 4.0}{araracolour}{\icinfo}{white}
%\textbf{Partial directive extraction} -- From version 4.0 on,
%it is now possible to extract directives only available in the
%file preamble, that is, all lines from the beginning that are
%comments until reaching the first line that is not a comment.
%To this end, a new command line flag is introduced. We will
%discuss this feature later on.
%
%\vspace{1em}
%
%\textbf{Predefined preambles} -- It is now possible to set up
%a common preamble to be used with files that require the same
%automation steps, then \arara\ can be invoked based on such
%specifications. We will discuss this feature later on.
%\end{messagebox}
%
%There are two types of directives in \arara. The first one has
%already been mentioned, it has only the rule name (which refers 
%to the \verb|identifier| key from the rule of the same name). It 
%is called \emph{empty directive}:
%
%\begin{codebox}{Empty directive}{teal}{\icnote}{white}
%% arara: pdflatex
%\end{codebox}
%
%Sometimes, however, we need to provide additional information to 
%the rule. That is reason for the second type, the 
%\emph{parametrized directive}, to exist. As the name indicates, 
%we have directive arguments! They are mapped by their identifiers
%and not by their positions. The syntax for a parametrized 
%directive is:
%
%\begin{codebox}{Parametrized directive}{teal}{\icnote}{white}
%% arara: pdflatex: { shell: yes }
%\end{codebox}
%
%Each argument is defined according to the rule mapped by the 
%directive. This means you cannot use an argument \verb|foo| in a 
%directive \verb|bar| if the rule \verb|bar| does not offer 
%support for it (that is, \verb|bar| has to have \verb|foo| 
%defined as argument in its list of arguments inside the rule 
%scope, as seen in the previous section). The syntax for
%an argument is:
%
%\begin{codebox}{Argument syntax}{teal}{\icnote}{white}
%key : value
%\end{codebox}
%
%Suppose we would like to enable shell escape for \verb|pdflatex| 
%when compiling a \verb|hello.tex| file. We can achieve that by 
%providing a parametrized directive, like this one:
%
%\begin{codebox}{\texttt{hello.tex} with a parametrized directive}{teal}{\icnote}{white}
%% arara: pdflatex: { shell: yes }
%
%\documentclass{article}
%\begin{document}
%Hello world!
%\end{document}
%\end{codebox}
%
%Of course, the \verb|shell| argument is defined in the
%\verb|pdflatex| rule scope, otherwise \arara\ would raise
%an error about an invalid argument key. If we try to
%inject a nonexistent \verb|foo| argument in the previous 
%parametrized directive, we will get this message:
%
%\begin{codebox}{Terminal}{teal}{\icnote}{white}
%  __ _ _ __ __ _ _ __ __ _ 
% / _` | '__/ _` | '__/ _` |
%| (_| | | | (_| | | | (_| |
% \__,_|_|  \__,_|_|  \__,_|
%
%Processing 'hello.tex' (size: 103 bytes, last modified:
%05/03/2018 15:40:16), please wait.
%
%I have spotted an error in rule 'pdflatex' located at
%'/opt/paulo/arara/rules'. I found these unknown keys
%in the directive: (foo). This should be an easy fix,
%just remove them from your map.
%
%Total: 0.21 seconds
%\end{codebox}
%
%As the message suggests, we need to remove the unknown argument 
%key from our directive or rewrite the rule in order to include 
%it. The first option is, of course, easier.
%
%\begin{messagebox}{Helpful messages}{araracolour}{\icinfo}{white}
%Make sure to read all messages \arara\ raises, they will help 
%you!
%\end{messagebox}
%
%Sometimes, directives can span several columns of a line, 
%particularly the ones with several arguments. From \arara\ 4.0 
%on, we can split a directive into multiple lines by adding
%\verb|% arara: -->| to each line which should compose the
%directive:
%
%\begin{codebox}{Multiline directive}{teal}{\icnote}{white}
%% arara: pdflatex: {
%% arara: --> shell: yes,
%% arara: --> synctex: yes
%% arara: --> }
%\end{codebox}
%
%It is important to observe that there is no need of them to be
%in contiguous lines, that is, provided that the syntax for
%parametrized directives hold for the line composition, lines can
%be distributed all over the code.
%
%\begin{messagebox}{New feature in version 4.0}{araracolour}{\icinfo}{white}
%\textbf{Conditionals} -- From version 4.0 on, \arara\ provides
%logical expressions processed at runtime to determine whether
%and  how a directive should be processed. This is a huge 
%improvement towards better user experience.
%\end{messagebox}
%
%One of the most awaited features that version 4.0 introduces is 
%the support of conditionals, that is, logical expressions 
%processed at runtime in order to determine whether and how the 
%directive should be processed. The following types are allowed:
%
%\begin{keywords}
%\item[if] evaluated beforehand, the directive is interpreted
%if and only if the result is true.
%
%\item[unless] evaluated beforehand, the directive
%is interpreted if and only if the result is false.
%
%\item[until] directive is interpreted the first time,
%then the evaluation is done; while the result
%is false, the directive is interpreted again and
%again.
%
%\item[while] evaluated beforehand, the directive is
%interpreted if and only if the result is true,
%and the process is repeated while the result
%still holds true.
%\end{keywords}
%
%We will discuss this special feature later on, including methods
%available in the directive scope in order to ease the writing
%of conditionals, as it would be highly advisable to have
%orb tags covered beforehand.
%
%\section{Orb tags}
%\label{sec:orbtags}

\end{document}
