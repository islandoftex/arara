% arara: manual

% Arara, the cool TeX automation tool
% Copyright (c) 2012 -- 2018, Paulo Roberto Massa Cereda
% All rights reserved.
%
% Redistribution and  use in source  and binary forms, with  or without
% modification, are  permitted provided  that the  following conditions
% are met:
%
% 1. Redistributions  of source  code must  retain the  above copyright
% notice, this list of conditions and the following disclaimer.
%
% 2. Redistributions in binary form  must reproduce the above copyright
% notice, this list  of conditions and the following  disclaimer in the
% documentation and/or other materials provided with the distribution.
%
% 3. Neither  the name  of the  project's author nor  the names  of its
% contributors may be used to  endorse or promote products derived from
% this software without specific prior written permission.
%
% THIS SOFTWARE IS  PROVIDED BY THE COPYRIGHT  HOLDERS AND CONTRIBUTORS
% "AS IS"  AND ANY  EXPRESS OR IMPLIED  WARRANTIES, INCLUDING,  BUT NOT
% LIMITED  TO, THE  IMPLIED WARRANTIES  OF MERCHANTABILITY  AND FITNESS
% FOR  A PARTICULAR  PURPOSE  ARE  DISCLAIMED. IN  NO  EVENT SHALL  THE
% COPYRIGHT HOLDER OR CONTRIBUTORS BE  LIABLE FOR ANY DIRECT, INDIRECT,
% INCIDENTAL, SPECIAL, EXEMPLARY,  OR CONSEQUENTIAL DAMAGES (INCLUDING,
% BUT  NOT LIMITED  TO, PROCUREMENT  OF SUBSTITUTE  GOODS OR  SERVICES;
% LOSS  OF USE,  DATA, OR  PROFITS; OR  BUSINESS INTERRUPTION)  HOWEVER
% CAUSED AND  ON ANY THEORY  OF LIABILITY, WHETHER IN  CONTRACT, STRICT
% LIABILITY, OR TORT (INCLUDING NEGLIGENCE OR OTHERWISE) ARISING IN ANY
% WAY  OUT  OF  THE USE  OF  THIS  SOFTWARE,  EVEN  IF ADVISED  OF  THE
% POSSIBILITY OF SUCH DAMAGE.
\documentclass[a4paper,oneside,12pt]{memoir}

\usepackage[T1]{fontenc}
\usepackage[utf8]{inputenc}
\usepackage[margin=2.5cm]{geometry}
\usepackage{arara}
\usepackage[record,postpunc=dot]{glossaries-extra}

\newcommand{\araraversion}{4.0}
\newcommand{\todo}[1]{\fbox{\em#1}}

\glssetcategoryattribute{abbreviation}{glossdesc}{firstuc}
\glssetcategoryattribute{general}{glossname}{firstuc}
\glssetcategoryattribute{general}{glossdesc}{firstuc}

\setabbreviationstyle{short-nolong-desc}
\renewcommand{\glsxtrshortdescname}{%
 \protect\glsabbrvfont{\the\glsshorttok} (\the\glslongtok)%
}

% No automated sorting. The list is just in order of definition.
% If the list gets too long, we can switch to using bib2gls.
 \newabbreviation
 [description={an interface that allows users to interact
through graphical components, such as buttons and menus}]
 {GUI}{GUI}{Graphical User Interface}

 \newabbreviation
 [description={an organisation that develops and promotes Internet standards}]
 {IETF}{IETF}{Internet Engineering Task Force}

 \newabbreviation
 [description={a virtual machine that enables Java programs to be run}]
 {JVM}{JVM}{Java Virtual Machine}

 \newabbreviation
 [description={a hybrid, dynamic, statically typed, embeddable
 expression language and runtime for the Java platform},
 location={(See Chapter~\ref{chap:mvel}.)}]
 {MVEL}{MVEL}{MVFLEX Expression Language}

\newglossaryentry{orb-tag}{name={orb tag},
 description={a dynamic element of an \gls{MVEL} template which is
 evaluated at runtime}}

\newabbreviation
 [description={a simple computer programming environment that takes
  a single expression (input), evaluates it and results the result}]
 {REPL}{REPL}{Read--Eval--Print Loop}

\newabbreviation
 [description={a database language}]
 {SQL}{SQL}{Structured Query Language}

\newabbreviation
 [description={a markup language that defines a set of rules for
encoding documents in a format that is both human-readable and
machine-readable}]
 {XML}{XML}{Extensible Markup Language}

\newabbreviation
 [description={human-friendly data, commonly used for configuration
files but also used for data storage or transmission},
 location={(See Chapter~\ref{chap:yaml}.)}]
 {YAML}{YAML}{YAML Ain't Markup Language}
 
\begin{document}

\begin{titlingpage}
\vspace*{2em}

\begin{center}
\includegraphics[scale=0.7]{../logos/logo2.pdf}

\vspace{4em}

\begin{tcolorbox}[
  boxrule=0pt,
  colback=araracolour,
  top=1em,
  bottom=1em
]
  \color{white}
  \centering
  \Huge
  \sffamily
  \bfseries User manual
\end{tcolorbox}

\vspace{6em}

{\large\em Paulo Cereda, Marco Daniel,\\
Brent Longborough, and Nicola Talbot\par}

\vspace{3em}

\href{mailto:cereda.paulo@gmail.com}{\fpemail{0.4}}%
\quad\href{https://github.com/cereda/arara}{\fpgithub{0.4}}%
\quad\href{http://twitter.com/paulocereda}{\fptwitter{0.4}}

\vfill

{\color{araracolour}
\LARGE
\sffamily
\bfseries
Version \araraversion}

\end{center}
\end{titlingpage}

\chapterstyle{araraheadings}
\pagestyle{headings}
\frontmatter
\nouppercaseheads

\cleardoublepage

\vspace*{25em}

\begin{flushright}
\em No birds were harmed in the making of this manual.
\end{flushright}

% !TeX root = ../arara-manual.tex
\chapter*{Foreword}
\label{chap:foreword}

\epigraph{That deserves no less than a ``Holy guacamole!''.}{\textsc{Gonzalo Medina}}

\emph{Foreword here.}

\vfill

\begin{flushright}
Nicola Louise Cecilia Talbot\\
\emph{on behalf of the \arara\ team}
\end{flushright}

% !TeX root = ../arara-manual.tex
\chapter*{Prologue}
\label{chap:prologue}

\epigraph{Moral of the story: never read the
documentation, bad things happen.}{\textsc{David Carlisle}}

\emph{Prologue here.}

\vfill

\begin{flushright}
Paulo Roberto Massa Cereda\\
\emph{on behalf of the \arara\ team}
\end{flushright}

% !TeX root = ../arara-manual.tex
\chapter*{License}
\label{chap:license}

\epigraph{Anything that prevents you from being friendly, a good neighbour, is a terror tactic.}{\textsc{Richard Stallman}}

\arara\ is licensed under the \href{http://www.opensource.org/licenses/bsd-license.php}{New BSD License}. It is important to observe that the New BSD License has been verified as a GPL-compatible free software license by the \href{http://www.fsf.org/}{Free Software Foundation}, and has been vetted as an open source license by the \href{http://www.opensource.org/}{Open Source Initiative}.

\vfill

\begin{messagebox}{New BSD License}{araracolour}{\icinfo}{white}
\footnotesize
\includegraphics[scale=0.25]{logos/logo1.pdf}

Copyright \textcopyright\ 2012--2018, Paulo Roberto Massa Cereda\\
All rights reserved.

\vspace{1em}

Redistribution and use in source and binary forms, with or without modification, are permitted provided that the following conditions are met:

\begin{itemize}
\item Redistributions of source code must retain the above copyright notice, this list of conditions and the following disclaimer.

\item Redistributions in binary form must reproduce the above copyright notice, this list of conditions and the following disclaimer in the documentation and/or other materials provided with the distribution.
\end{itemize}

This software is provided by the copyright holders and contributors ``as is'' and any express or implied warranties, including, but not limited to, the implied warranties of merchantability and fitness for a particular purpose are disclaimed. In no event shall the copyright holder or contributors be liable for any direct, indirect, incidental, special, exemplary, or consequential damages (including, but not limited to, procurement of substitute goods or services; loss of use, data, or profits; or business interruption) however caused and on any theory of liability, whether in contract, strict liability, or tort (including negligence or otherwise) arising in any way out of the use of this software, even if advised of the possibility of such damage.
\end{messagebox}

\printunsrtglossary

\cleardoublepage

\vspace*{25em}

\thispagestyle{empty}
\begin{flushright}
\em To Marco's son Niclas.
\end{flushright}

\cleardoublepage

\tableofcontents*

\cleardoublepage

\mainmatter

% !TeX root = ../arara-manual.tex
\chapter{Introduction}
\label{chap:introduction}

Hello there, welcome to \arara, the cool \TeX\ automation tool! This chapter is actually a quick introduction to what you can (and cannot) expect from \arara. For now, concepts will be informally presented and will be detailed later on, in the next chapters.

\section{What is this tool?}
\label{sec:whatisthistool}

Good question! \arara\ is a \TeX\ automation tool based on rules and directives. It is, in some aspects, similar to other well-known tools like \rbox{latexmk} and \rbox{rubber}. The key difference (and probably the selling point) might be the fact that \arara\ aims at explicit instructions in the source code (in the form of comments) in order to determine what to do instead of relying on other resources, such as log file analysis. It is a different approach for an automation tool, and we have both advantages and disadvantages of such design. Let us use the following file \rbox{hello.tex} as an example:

\begin{ncodebox}{Source file}{teal}{\icnote}{white}{hello.tex}
\documentclass{article}

\begin{document}
Hello world!
\end{document}
\end{ncodebox}

How would one successfully compile \rbox{hello.tex} with \rbox{latexmk} and \rbox{rubber}, for instance? It is quite straightforward: it is just a matter of providing the file to the tool and letting it do the hard work:

\begin{codebox}{Terminal}{teal}{\icnote}{white}
$ latexmk -pdf mydoc.tex
$ rubber --pdf mydoc.tex
\end{codebox}

The mentioned tools perform an analysis on the file and decide what has to be done. However, if one tries to invoke \rbox{arara} on \rbox{hello.tex}, I am afraid \emph{nothing} will be generated; the truth is, \arara\ does not know what to do with your file, and the tool will even raise an error message complaining about this issue:

\begin{codebox}{Terminal}{teal}{\icnote}{white}
$ arara hello.tex
  __ _ _ __ __ _ _ __ __ _ 
 / _` | '__/ _` | '__/ _` |
| (_| | | | (_| | | | (_| |
 \__,_|_|  \__,_|_|  \__,_|

Processing 'hello.tex' (size: 86 bytes, last modified: 05/03/2018
07:28:30), please wait.

It looks like no directives were found in the provided file. Make
sure to include at least one directive and try again.

Total: 0.00 seconds
\end{codebox}

Quite surprising. However, this behaviour is not wrong at all, it is completely by design: \arara\ needs to know what you want. And for that purpose, you need to tell the tool what to do.

\begin{messagebox}{A very important concept}{attentioncolour}{\icattention}{black}
That is the major difference of \arara\ when compared to other tools: \emph{it is not an automatic process and the tool does not employ any guesswork on its own}. You are in control of your documents; \arara\ will not do anything unless you \emph{teach it how to do a task and explicitly tell it to execute the task}.
\end{messagebox}

Now, how does one tell \arara\ to do a task? That is the actually the easy part, provided that you have everything up and running. We accomplish the task by adding a special comment line, hereafter known as \emph{directive}, somewhere in our \rbox{hello.tex} file (preferably in the first lines):

\begin{ncodebox}{Source file}{teal}{\icnote}{white}{hello.tex}
% arara: pdflatex
\documentclass{article}

\begin{document}
Hello world!
\end{document}
\end{ncodebox}

For now, do not worry too much about the terms, we will come back to them later on, in Chapter~\ref{chap:importantconcepts}, on page~\pageref{chap:importantconcepts}. It suffices to say that \arara\ expects \emph{you} to provide a list of tasks, and this is done by inserting special comments in the source file. Let us see how \arara\ behaves with this updated code:

\begin{codebox}{Terminal}{teal}{\icnote}{white}
$ arara hello.tex 
  __ _ _ __ __ _ _ __ __ _ 
 / _` | '__/ _` | '__/ _` |
| (_| | | | (_| | | | (_| |
 \__,_|_|  \__,_|_|  \__,_|

Processing 'hello.tex' (size: 86 bytes, last modified: 05/03/2018
07:28:30), please wait.

(PDFLaTeX) PDFLaTeX engine .............................. SUCCESS

Total: 0.73 seconds
\end{codebox}

Hurrah, we finally got our document properly compiled with a \TeX\ engine by the inner workings of our beloved tool, resulting in an expected \rbox{hello.pdf} file created using the very same system call that typical automation tools like \rbox{latexmk} and \rbox{rubber} use. Observe that \arara\ works practically on other side of the spectrum: you need to tell it how and when to do a task.

\section{Core concepts}
\label{sec:coreconcepts}

When adding a directive in our source code, we are explicitly telling the tool what we want it to do, but I am afraid that is not sufficient at all. So far, \arara\ knows \emph{what} to do, but now it needs to know \emph{how} the task should be done. If we want \arara\ to run \rbox{pdflatex} on \rbox{hello.tex}, we need to have instructions telling our tool how to run that specific application. This particular sequence of instructions is referred as a \emph{rule} in our context. 

\begin{messagebox}{Note on rules}{attentioncolour}{\icattention}{black}
Although the core team provides a lot of rules shipped with \arara\ out of the box, with the possibility of extending the set by adding more rules, some users might find this decision rather annoying, since other tools have most of their rules hard-coded, making the automation process even more transparent. However, since \arara\ does not rely on a specific automation or compilation scheme, it becomes more extensible. The use of directives in the source code make the automation steps more fluent, which allows the specification of complex workflows very easy.
% "very easy" -> "much easier" perhaps?
\end{messagebox}

Despite the inherited verbosity of automation steps not being suitable for small documents, \arara\ really shines when you have a document which needs full control of the automation process (for instance, a thesis or a manual). Complex workflows are easily tackled by our tool.

Rules and directives are the core concepts of \arara: the first dictates how a task is done, and the latter is the proper instance of the associated rule on the current document, i.e, when and where the commands must be executed.

\begin{messagebox}{The name}{araracolour}{\icok}{white}
\begin{minipage}{0.45\textwidth}
\vspace{.8em}
{\centering\includegraphics[width=0.9\textwidth]{figures/arara.png}\par}

\vspace{.7em}

\em Do you like araras? We do, specially our tool which shares the same name of this colorful bird.
\end{minipage}\hspace{1em}
\begin{minipage}{0.5\textwidth}
The tool name was chosen as an homage to a Brazilian bird of the same name, which is a macaw. The word \emph{arara} comes from the Tupian word \emph{a'rara}, which means \emph{big bird} (much to my chagrin, Sesame Street's iconic character Big Bird is not a macaw; according to some sources, he claims to be a golden condor). Araras are colorful, noisy, naughty and very funny. Everybody loves araras. The name seemed catchy for a tool and, in the blink of an eye, \arara\ was quickly spread to the whole \TeX\ world.
\end{minipage}
\end{messagebox}

Now that we informally introduced rules and directives, let us take a look on how \arara\ actually works given those two elements. The whole idea is pretty straightforward, and I promise to revisit these concepts later on in this manual for a comprehensive explanation (more precisely, in Chapter~\ref{chap:importantconcepts}).

First and foremost, we need to add at least one instruction in the source code to tell \arara\ what to do. This instruction is named a \emph{directive} and it will be parsed during the preparation phase. Observe that \arara\ will tell you if no directive was found in a file, as seen in our first interaction with the tool.

An \arara\ directive is usually defined in a line of its own, started with a comment (denoted by a percent sign in \TeX\ and friends), followed by the word \rbox{arara:} and task name:

\begin{codebox}{A typical directive}{teal}{\icnote}{white}
% arara: pdflatex
\documentclass{article}
...
\end{codebox}

Our example has one directive, referencing \rbox{pdflatex}. It is important to observe that the \rbox{pdflatex} identifier \emph{does not represent the command to be executed}, but \emph{the name of the rule associated with that directive}.

\begin{messagebox}{New feature in version 4.0}{araracolour}{\icinfo}{white}
\textbf{Multiline directives} -- Later on, in Section~\ref{sec:directives}, on page~\pageref{sec:directives}, we will discover that a directive can also span several lines in order to provide a better code organization. For now, let us assume a typical directive occupies only one line.
\end{messagebox}

Once \arara\ finds a directive, it will look for the associated \emph{rule}. In our example, it will look for a rule named \rbox{pdflatex} which will evidently run the \rbox{pdflatex} command line application. Rules are \gls{YAML} files named according to their identifiers followed by the \rbox{yaml} extension and follow a strict structure. This concept is covered in Section~\ref{sec:rule}, on page~\pageref{sec:rule}.

\begin{messagebox}{New feature in version 4.0}{araracolour}{\icattention}{white}
\textbf{\gls{REPL} workflow} -- \arara\ now employs a \gls{REPL} workflow for rules and directives. In previous versions, directives were extracted, their corresponding rules were analyzed, commands were built and added to a queue before any proper execution or evaluation. I decided to change this workflow, so now \arara\ evaluates each rule on demand, i.e, there is no \emph{a priori} checking. A rule will \emph{always} reflect the current state, including potential side effects from previous executed rules.
\end{messagebox}

Now, we have a queue of pairs $(\textit{directive}, \textit{rule})$ to process. For each pair, \arara\ will map the directive to its corresponding rule, evaluate it and run the proper command. The execution chain requires that command $i$ was successfully executed to then proceed to command $i+1$, and so forth. This is also by design: \arara\ will halt the execution if any of the commands in the queue had raised an error. How does one know if a command was successfully executed? \arara\ checks the corresponding \emph{exit status} available after a command execution. In general, a successful execution yields 0 as its exit status.

\begin{messagebox}{New feature in version 4.0}{araracolour}{\icinfo}{white}
\textbf{Custom exit status checking} -- In previous versions, there was no way of customizing the exit status checking of a command. A command was successful if, and only if, its resulting exit status was 0 and no other value. From now on, we can define any value, or even forget about it and make it always return a valid status regardless of execution (for instance, in a rule that always is successful -- see, for instance, the \rbox{clean}  rule).
\end{messagebox}

That is pretty much how \arara\ works: directives in the source code are mapped to rules. These pairs are added to a queue. The queue is then executed and the status is reported. More details about the expansion process are presented in Chapter~\ref{chap:importantconcepts}, on page~\pageref{chap:importantconcepts}. In short, we teach \arara\ to do a task by providing a rule, and tell it to execute it through directives in the source code.

\section{Operating system remarks}
\label{sec:operatingsystemremarks}

The application is written using the Java language, so \arara\ runs on top of a Java virtual machine, available on all the major operating systems~--~in some cases, you might need to install the proper virtual machine. We tried very hard to keep both code and libraries compatible with older virtual machines or from other vendors. Currently, \arara\ is known to run on Oracle's Java 5 to 10, and OpenJDK 5 to 10. We also have reports of users successfully using the tool with virtual machines provided by Azul Systems, so your mileage might vary greatly.

\begin{messagebox}{Outdated Java virtual machines}{attentioncolour}{\icerror}{black}
Dear reader, beware of outdated software, mainly Java virtual machines! Although \arara\ offers support for older virtual machines, try your best to keep your software updated as frequently as possible. The legacy support exists only for historical reasons, and also due to the sheer fact that we know some people that still runs \arara\ on very old hardware. If you are not in this particular scenario, get the latest virtual machine.
\end{messagebox}

In Chapter~\ref{chap:buildingfromsource}, on page~\pageref{chap:buildingfromsource}, we provide instructions on how to build \arara\ from sources using Apache Maven. Even if you use multiple operating systems, \arara\ should behave the same, including the rules. There are helper functions available in order to provide support for system-specific rules based on the underlying operating system.

\section{Support}
\label{sec:support}

If you run into any issue with \arara, please let us know. We all have very active profiles in the \href{https://tex.stackexchange.com/}{\TeX\ community at StackExchange}, so just use the \rbox[araracolour]{arara} tag in your question and we will help you the best we can (also, take a look at their \href{https://tex.meta.stackexchange.com/q/1436}{starter guide}).  We also have a \href{https://gitter.im/cereda/arara}{Gitter chat room}, in which we occasionally hang out. Also, if you think the report is worthy of an issue, open one in our \href{https://github.com/cereda/arara/issues}{GitHub repository}. And last, but not least, feel free to poke us by good old electronic mail (please try the other approaches first).

We really hope you like our humble contribution to the \TeX\ community. Let \arara\ enhance your \TeX\ experience, it will help you when you will need it the most. Enjoy the manual.


\part{The application}
\label{part:application}

% !TeX root = ../arara-manual.tex
\chapter{Important concepts}
\label{chap:importantconcepts}

Time for our first contact with \arara! I must strees that is very important to understand a few concepts in which \arara\ relies before we proceed to the usage itself. Do not worry, these concepts are easy to follow, yet they are vital to the comprehension of the application and the logic behind it.

\section{Rules}
\label{sec:rule}

A \emph{rule} is a formal description of how \arara\ handles a certain task. For instance, if we want to use \abox{pdflatex} with our tool, we should have a rule for that. Directives are mapped to rules, so a call to a nonexistent rule \abox{foo}, for instance, will indeed raise an error:

\begin{codebox}{Terminal}{teal}{\icnote}{white}
  __ _ _ __ __ _ _ __ __ _ 
 / _` | '__/ _` | '__/ _` |
| (_| | | | (_| | | | (_| |
 \__,_|_|  \__,_|_|  \__,_|

Processing 'doc1.tex' (size: 83 bytes, last modified: 05/03/2018
12:10:33), please wait.

I could not find a rule named 'foo' in the provided rule paths.
Perhaps a misspelled word? I was looking for a file named
'foo.yaml' in the following paths in order of priority:
(/opt/paulo/arara/rules)

Total: 0.09 seconds
\end{codebox}

% TODO fix reference
Once a rule is defined, \arara\ automatically provides an access layer to that rule through directives in the source code, a concept to be formally introduced later on, in Section~\ref{foo}. Observe that a directive reflects a particular instance of a rule of the same name (i.e, a \abox{foo} directive in a certain source code is an instance of the \abox{foo} rule).

In short, a rule is a plain text file written in the YAML format, introduced in Chapter~\ref{foo} (page~\pageref{foo}). I opted for this format because back then it was cleaner and more intuitive to use than other markup languages such as XML, besides of course being a data serialization standard for programming languages.

\begin{messagebox}{Animal jokes}{araracolour}{\icok}{white}
As a bonus, the acronym \emph{YAML} rhymes with the word \emph{camel}, so \arara\ is heavily environmentally friendly. Speaking of camels, there is the programming reference as well, since this amusing animal is usually associated with Perl and friends.
\end{messagebox}

% TODO fix reference
The default rules, that is, the rules shipped with \arara, are placed inside a special subdirectory named \abox[araracolour]{rules/} inside another special directory named \abox[araracolour]{ARARA\_HOME} (the place where our tool is installed). We will learn later on, in Section~\ref{foo} (page~\pageref{foo}), that we can add an arbitrary number of paths for storing our own rules, in order of priority, so do not worry too much with the location of the default rules, although it is important to understand and acknowledge their existance.

The following list describes the basic structure of an \arara\ rule by presenting the proper elements (or keys, if we consider the proper YAML nomenclature). Observe that elements marked as \rbox[araracolour]{M} are mandatory (i.e, the rule \emph{has} to have them in order to work). Similarly, elements marked as \rbox[araracolour]{O} are optional, so you can safely ignore them when writing a rule for our tool. A key preceded by \rbox{context$\rightarrow$} indicates a context and should be properly defined inside it.

\begin{description}
\item[\describe{M}{!config}] This keyword is mandatory and must be the first line of any \arara\ rule. It denotes the object mapping metadata to be internally used by the tool. Actually, the tool is not too demanding on using it (in fact, you could suppress it entirely and \arara\ will not complain), but it is considered good practice to start all rules with a \abox{!config} keyword regardless.

\item[\describe{M}{identifier}] This key acts as a unique identifier for the rule (as expected). It is highly recommended to use lowercase letters without spaces, accents or punctuation symbols, as good practice (again). As a convention, if you have an identifier named \abox{pdflatex}, the rule filename must be \abox{pdflatex.yaml} (like our own instance). Please note that, although \abox{.yml} is known to be a valid YAML extension as well, \arara\ only considers files ending with the \abox{.yaml} extension. This is a deliberate decision.

\begin{codebox}{Example}{teal}{\icnote}{white}
identifier: pdflatex
\end{codebox}

\item[\describe{M}{name}] This key holds the name of the task as a plain string. When running \arara, this value will be displayed in the output. We like to call \emph{task} an instantiated rule (through a directive). Task names are displayed enclosed in parenthesis.

\begin{codebox}{Example}{teal}{\icnote}{white}
name: PDFLaTeX
\end{codebox}

\item[\describe{O}{authors}] We do love blaming people, so \arara\ features a special key to name the rule authors (if any) so you can write stern electronic communications to them! This key holds a list of strings. If the rule has just one author, add it as the first (and only) element of the list.

\begin{codebox}{Example}{teal}{\icnote}{white}
authors:
- Marco Daniel
- Paulo Cereda
\end{codebox}

\item[\describe{M}{commands}] This key is introduced in version 4.0 of \arara\ and denotes a potential list of commands. From the user perspective, each command is called \emph{subtask} within a task (rule and directive) context. A task may represent only a single command (a single subtask), as well as a sequence of commands (subtasks). For instance, the \abox{frontespizio} rule requires at least two commands. So, as a means of normalizing the representation, a task composed of a single command (single subtask) is defined as the only element of such list, as opposed to previous versions of \arara, which had an specific key to hold just one command.

\begin{messagebox}{Incompatibility with older versions}{attentioncolour}{\icerror}{black}
Dear reader, note that rules from version 4.0 are incompatible with older versions of \arara. If you are migrating from old versions to version 4.0, we need to replace \abox{command} by \abox{commands} and specifying a contextual element, as seen in the following descriptions. Please refer to Section~\ref{foo} (page~\pageref{foo}) for a comprehensible migration guide.
\end{messagebox}

In order to properly set a subtask, the keys used in this specification are defined inside the \rbox{commands$\rightarrow$} context and presented as follows.

\begin{description}
\item[\describecontext{O}{commands}{name}] This key holds the name of the subtask as a plain string. When running \arara, this value will be displayed in the output. Subtask names are displayed after the main task name. By the way, did you notice that this key is entirely optional? That means that a subtask can simply be unnamed, if you decide so. However, such practice is not recommended, as is always good to have a visual description of what \arara\ is running at the moment, so name your subtasks properly.

\item[\describecontext{M}{commands}{command}] This key holds the action to be performed, typically a system command. In previous versions, \arara\ would rely solely on a string. For this version on, as a means to enhance the user experience (and also fix serious blockers when handling spaces in file names, as seen in \href{https://github.com/cereda/arara/issues}{previous issues} reported in the repository), the tool offers four types of returned values:

\begin{itemize}[label={--}]
\item A plain string: this is the default (and only) behaviour in older versions of \arara. The plain string is processed as it is by the underlying execution engine. However, automatic argument parsing poses as a complex issue, so this approach, although supported, is not recommended anymore.

\begin{codebox}{Example}{teal}{\icnote}{white}
command: 'ls'
\end{codebox}

% TODO fix reference
It is important to observe that you can use either a plain string directly or using an orb tag with an explicit \abox{return} command (as seen in Section~\ref{foo}, page~\pageref{foo}). Personally, I favour the explict indication for a quick understanding.

% TODO fix reference
\item A \abox{Command} object: \arara\ 4.0 features a new approach for handling system commands based on a high level structure with explict argument parsing named \abox{Command} (for our curious users, it is a plain Java object). In order to use this approach, we need to rely on orb tags and use a helper method named \mtbox{getCommand} to obtain the desired result. We will detail this method later on, in Section~\ref{foo} (page~\pageref{foo}). We highly recommend the adoption of this approach for rule writing instead of using plain strings.

\begin{codebox}{Example}{teal}{\icnote}{white}
command: "@{ return getCommand('ls') }"
\end{codebox}

% TODO fix reference
\item A boolean value: it is also possible to exploit the expressive power of the underlying scripting language available in the rule context (see Chapter~\ref{foo}, in page~\pageref{foo}, for more details) for writing complex code. In this particular case, since the computation is being done by \arara\ itself and not the underlying operating system, there will not be a command to be executed, so simply return a boolean value -- either an explicit \abox{true} or \abox{false} value or a logical expression -- to indicate whether the computation was successfull.

\begin{codebox}{Example}{teal}{\icnote}{white}
command: "@{ return 1 == 1 }"
\end{codebox}

\item A \abox{Trigger} object: this is surely the least common type of returned value and it is mentioned here just for documentation purposes. In simple terms, a \abox{Trigger} object constitutes a special command that changes the internal workings of \arara\ at runtime. We have not worked much on this concept, so there is only one trigger available, seen in action in the official \abox{halt} rule. In order to use this approach, we need to rely on orb tags and use a helper method named \mtbox{getTrigger} to obtain the desired result.
\end{itemize}

It is also worth mentioning that \arara\ also supports lists of commands represented as plain strings, \abox{Command} or \abox{Trigger} objects, boolean values or a mix of them. This is useful if your rule has to decide whether more actions are required in order to accomplish a task. In this case, our tool will take care of the list and execute each element in the specified order.

\begin{codebox}{Example}{teal}{\icnote}{white}
command: "@{ return [ 'ls', 'ls', 'ls' ] }"
\end{codebox}

As an example, please refer to the official \abox{clean} rule for a real scenario where a list of commands is successfully employed: for each provided extension, the rule creates a new cleaning command and adds it to a list of removals to be processed later.

\begin{messagebox}{Plain string is deprecated}{attentioncolour}{\icattention}{black}
It took me a lot of effort to find out that handling plain strings and employing guesswork to parse arguments are the root of several issues reported by users. Therefore, this approach is being marked as \emph{deprecated} and will be removed in future versions.
\end{messagebox}

% TODO fix reference
There are at least two variables available in the \abox{command} context and are described as follows (note that MVEL variables and orb tags are discussed in Chapter~\ref{foo}). A variable will be denoted by \varbox{variable} in this list. For each rule argument (defined later on), there will be a corresponding variable in the \abox{command} context, directly accessed through its unique identifier.

\begin{description}
\item[\varbox{file}] This variable holds the file name, without any path reference, as a plain string. It is usually composed by the base name and the extension. This variable is available since the first release of \arara.

\item[\varbox{reference}] This variable is introduced in version 4.0 of \arara\ and holds the canonical, absolute path representation of the \varbox{file} variable as a \abox{File} object. This is useful if there is a need of understanding the hierarchical structure of a project. Since the reference is a Java object, we can use all methods available in the \abox{File} class.
\end{description}

\begin{messagebox}{Quote handling}{araracolour}{\icinfo}{white}
\setlength{\parskip}{1em}
The YAML format disallows key values starting with \abox{@} without proper quoting. This is the reason we had to use double quotes for the value and internally using single quotes for the command string. Also, we could use the other way around, or even using only one type and then escaping them when needed. This is excessively verbose but needed due to the format requirement. Thankfully, \arara\ offers two solutions for removing the quoting verbosity when writing commands.

The first solution is used in previous versions and it still works like a charm in modern days. We need to precede our command with a special keyword \abox{<arara>} which will be removed afterwards. This solution works on virtually every key in the rule context, so it is a bonus. The new code will look like this:

\begin{codebox}{Example}{teal}{\icnote}{white}
command: <arara> @{ return getCommand('ls') }
\end{codebox}

% TODO fix reference
The second approach is more of a YAML feature rather than a tool exclusive, although we have to do a couple of checkings under the hood in order to ensure the correct execution. The idea here is to use the scalar content in folded style, as seen in Section~\ref{foo} (page~\pageref{foo}). The new code will look like this:

\begin{codebox}{Example}{teal}{\icnote}{white}
command: >
  @{
    return getCommand('ls')
  }
\end{codebox}

Mind the indentation, as YAML requires it to properly identify blocks. I personally recommend this approach for longer code, as it provides a better visual representation. You will see the second solution all around the default rules, but feel free to use the one you feel more comfortable.
\end{messagebox}

\item[\describecontext{O}{commands}{exit}] This key holds a special purpose in \arara\ 4.0, as it represents a custom exit status evaluation for the corresponding command. In general, a successful execution has zero as an exit status, but sometimes we end up with tools or situations that we need to override this checking for whatever reason. For this purpose, simply write a MVEL expression \emph{without orb tags} as plain string and use the special variable \varbox{value} if you need the actual exit status returned by the command, available at runtime. For example, if the command returns a non-zero value indicating a successful execution, we can write this key as:

\begin{codebox}{Example}{teal}{\icnote}{white}
exit: value > 0
\end{codebox}

If the execution should be marked as successful by \arara\ regardless of the actual exit status, you can simply write \abox{true} as the key value and this rule will never fail, for obvious reasons.
\end{description}

For instance, consider a full example of the \abox{commands} key, defined with only one command, presented as follows. The hyphen denotes a list element, so mind the indentation for correctly specifying the component keys. Also, note that, in this case, the \abox{exit} key was completely optional, as it does the default checking, and it was included for didactic purposes.

\begin{codebox}{Example}{teal}{\icnote}{white}
commands:
- name: The PDFLaTeX engine
  command: >
    @{
      return getCommand('pdflatex', file)
    }
  exit: value == 0
\end{codebox}

\item[\describe{M}{arguments}] This key holds a list of arguments for the current rule, if any. The arguments specified in this list will be available to the user later on for potential completion through directives. Once instantiated, they will become proper variables in the \abox{command} contexts. This key is mandatory, so even if your rule does not have arguments, you need to specify a list regardless. In this case, use the empty list notation:

\begin{codebox}{Example}{teal}{\icnote}{white}
arguments: []
\end{codebox}

In order to properly set an argument, the keys used in this specification are defined inside the \rbox{arguments$\rightarrow$} context and presented as follows.

\begin{description}
\item[\describecontext{M}{arguments}{identifier}] This key acts as a unique identifier for the argument. It is highly recommended to use lowercase letters without spaces, accents or punctuation symbols, as a good practice. This key will be used later on to set the corresponding value in the directive context.

\begin{codebox}{Example}{teal}{\icnote}{white}
identifier: shell
\end{codebox}

\item[\describecontext{O}{arguments}{flag}] This key holds a plain string and is evaluated when the corresponding argument is defined in the directive context.  After being evaluated, the result will be stored in a variable of the same name to be later accessed in the \abox{command} context. In the scenario where the argument is not defined in the directive, the variable will hold an empty string.

\begin{codebox}{Example}{teal}{\icnote}{white}
flag: >
  @{
      isTrue(parameters.shell, '--shell-escape',
             '--no-shell-escape')
  }
\end{codebox}

% TODO fix reference
There is one variable available in the \abox{flag} context and is described as follows. Note that are also several helper methods available in the rule context (for instance, \mtbox{isTrue} presented in the previous example) which provide interesting features for rule writing. They are detailed later on, in Section~\ref{foo} (page~\pageref{foo}).

\begin{description}
\item[\varbox{parameters}] This variable holds a map of directive parameters available at runtime. For each argument identifier listed in the \abox{arguments} list in the rule context, there will be an entry in this variable. This is useful to get the actual values provided during execution and take proper actions. If a parameter is not set in the directive context, the reference will still exist in the map, but it will be mapped to an empty string.
\end{description}

In the previous example, observe that the MVEL expression defined in the \abox{flag} key checks if the user provided an affirmative value regarding shell escape, through comparing \varbox{parameters.shell} with a set of predefined affirmative values. In any case, the corresponding command flag is defined as result of such evaluation.

\item[\describecontext{O}{arguments}{default}] As default behaviour, if a parameter is not set in the directive context, the reference will be mapped to an empty string. This key exists for the exact purpose of overriding such behaviour and expects a plain string as value.

\begin{codebox}{Example}{teal}{\icnote}{white}
default: ''
\end{codebox}

\item[\describecontext{O}{arguments}{required}] There might be certain scenarios in which a rule could make use of required arguments (for instance, a copy operation in which source and target must be provided). The \abox{required} key acts as a boolean switch to indicate whether the corresponding argument should be mandatory. In this case, set the key value to \abox{true} and the argument turns required. Later on at runtime, \arara\ will throw an error if a required parameter is missing in the directive.

\begin{codebox}{Example}{teal}{\icnote}{white}
required: false
\end{codebox}

Note that setting the \abox{required} key value to \abox{false} corresponds to omitting the key completely in the rule context, which resorts to the default behaviour (i.e, all arguments are optional).
\end{description}

\begin{messagebox}{Note on argument keys}{attentioncolour}{\icattention}{black}
As seen previously, both \abox{flag} and \abox{default} are marked as optional, but at least one of them must occur in the argument specification, otherwise \arara\ will throw an error, as it makes no sense to have no argument handling at all. Please make sure to specify at least one of them!
\end{messagebox}

For instance, consider a full example of the \abox{arguments} key, defined with only one argument, presented as follows. The hyphen denotes a list element, so mind the indentation for correctly specifying the component keys. Also, note that, in this case, keys \abox{required} and \abox{default} were completely optional, and they were included for didactic purposes.

\begin{codebox}{Example}{teal}{\icnote}{white}
arguments:
- identifier: shell
  flag: >
    @{
        isTrue(parameters.shell, '--shell-escape',
               '--no-shell-escape')
    }
  required: false
  default: ''
\end{codebox}
\end{description}

% TODO fix reference
This is the rule structure in the YAML format used by \arara. Keep in mind that all subtasks in a rule are checked against their corresponding exit status. If an abnormal execution is detected, the tool will instantly halt and the rule will fail. Even \arara\ itself will return an exit code different than zero when this situation happens (detailed in Section~\ref{foo}, in page~\pageref{foo}).

\section{Directives}
\label{sec:directives}

%A \emph{directive} is a special comment inserted in the
%source file in which you indicate how \arara\ should
%behave. You can insert as many directives as you
%want and in any position of the file. The tool will
%read the whole file and extract the directives.
%
%\begin{messagebox}{New features in version 4.0}{araracolour}{\icinfo}{white}
%\textbf{Partial directive extraction} -- From version 4.0 on,
%it is now possible to extract directives only available in the
%file preamble, that is, all lines from the beginning that are
%comments until reaching the first line that is not a comment.
%To this end, a new command line flag is introduced. We will
%discuss this feature later on.
%
%\vspace{1em}
%
%\textbf{Predefined preambles} -- It is now possible to set up
%a common preamble to be used with files that require the same
%automation steps, then \arara\ can be invoked based on such
%specifications. We will discuss this feature later on.
%\end{messagebox}
%
%There are two types of directives in \arara. The first one has
%already been mentioned, it has only the rule name (which refers 
%to the \verb|identifier| key from the rule of the same name). It 
%is called \emph{empty directive}:
%
%\begin{codebox}{Empty directive}{teal}{\icnote}{white}
%% arara: pdflatex
%\end{codebox}
%
%Sometimes, however, we need to provide additional information to 
%the rule. That is reason for the second type, the 
%\emph{parametrized directive}, to exist. As the name indicates, 
%we have directive arguments! They are mapped by their identifiers
%and not by their positions. The syntax for a parametrized 
%directive is:
%
%\begin{codebox}{Parametrized directive}{teal}{\icnote}{white}
%% arara: pdflatex: { shell: yes }
%\end{codebox}
%
%Each argument is defined according to the rule mapped by the 
%directive. This means you cannot use an argument \verb|foo| in a 
%directive \verb|bar| if the rule \verb|bar| does not offer 
%support for it (that is, \verb|bar| has to have \verb|foo| 
%defined as argument in its list of arguments inside the rule 
%scope, as seen in the previous section). The syntax for
%an argument is:
%
%\begin{codebox}{Argument syntax}{teal}{\icnote}{white}
%key : value
%\end{codebox}
%
%Suppose we would like to enable shell escape for \verb|pdflatex| 
%when compiling a \verb|hello.tex| file. We can achieve that by 
%providing a parametrized directive, like this one:
%
%\begin{codebox}{\texttt{hello.tex} with a parametrized directive}{teal}{\icnote}{white}
%% arara: pdflatex: { shell: yes }
%
%\documentclass{article}
%\begin{document}
%Hello world!
%\end{document}
%\end{codebox}
%
%Of course, the \verb|shell| argument is defined in the
%\verb|pdflatex| rule scope, otherwise \arara\ would raise
%an error about an invalid argument key. If we try to
%inject a nonexistent \verb|foo| argument in the previous 
%parametrized directive, we will get this message:
%
%\begin{codebox}{Terminal}{teal}{\icnote}{white}
%  __ _ _ __ __ _ _ __ __ _ 
% / _` | '__/ _` | '__/ _` |
%| (_| | | | (_| | | | (_| |
% \__,_|_|  \__,_|_|  \__,_|
%
%Processing 'hello.tex' (size: 103 bytes, last modified:
%05/03/2018 15:40:16), please wait.
%
%I have spotted an error in rule 'pdflatex' located at
%'/opt/paulo/arara/rules'. I found these unknown keys
%in the directive: (foo). This should be an easy fix,
%just remove them from your map.
%
%Total: 0.21 seconds
%\end{codebox}
%
%As the message suggests, we need to remove the unknown argument 
%key from our directive or rewrite the rule in order to include 
%it. The first option is, of course, easier.
%
%\begin{messagebox}{Helpful messages}{araracolour}{\icinfo}{white}
%Make sure to read all messages \arara\ raises, they will help 
%you!
%\end{messagebox}
%
%Sometimes, directives can span several columns of a line, 
%particularly the ones with several arguments. From \arara\ 4.0 
%on, we can split a directive into multiple lines by adding
%\verb|% arara: -->| to each line which should compose the
%directive:
%
%\begin{codebox}{Multiline directive}{teal}{\icnote}{white}
%% arara: pdflatex: {
%% arara: --> shell: yes,
%% arara: --> synctex: yes
%% arara: --> }
%\end{codebox}
%
%It is important to observe that there is no need of them to be
%in contiguous lines, that is, provided that the syntax for
%parametrized directives hold for the line composition, lines can
%be distributed all over the code.
%
%\begin{messagebox}{New feature in version 4.0}{araracolour}{\icinfo}{white}
%\textbf{Conditionals} -- From version 4.0 on, \arara\ provides
%logical expressions processed at runtime to determine whether
%and  how a directive should be processed. This is a huge 
%improvement towards better user experience.
%\end{messagebox}
%
%One of the most awaited features that version 4.0 introduces is 
%the support of conditionals, that is, logical expressions 
%processed at runtime in order to determine whether and how the 
%directive should be processed. The following types are allowed:
%
%\begin{keywords}
%\item[if] evaluated beforehand, the directive is interpreted
%if and only if the result is true.
%
%\item[unless] evaluated beforehand, the directive
%is interpreted if and only if the result is false.
%
%\item[until] directive is interpreted the first time,
%then the evaluation is done; while the result
%is false, the directive is interpreted again and
%again.
%
%\item[while] evaluated beforehand, the directive is
%interpreted if and only if the result is true,
%and the process is repeated while the result
%still holds true.
%\end{keywords}
%
%We will discuss this special feature later on, including methods
%available in the directive scope in order to ease the writing
%of conditionals, as it would be highly advisable to have
%orb tags covered beforehand.
%
%\section{Orb tags}
%\label{sec:orbtags}

% !TeX root = ../arara-manual.tex
\chapter{Command line}
\label{chap:commandline}


% !TeX root = ../arara-manual.tex
\chapter{Configuration file}
\label{chap:configurationfile}

\arara\ provides a persistent model of modifying the underlying execution behaviour or enhance the execution workflow through the concept of a configuration file. This chapter provides the basic structure of such file, as well as details on the file lookup in the operating system.

\section{File lookup}
\label{sec:filelookup}

Our tool looks for the presence of at least one of four very specific files before execution. These files are presented as follows. Observe that the directories must have the correct permissions for proper lookup and access. The lookup order is also presented.

\vspace{1em}

{\centering
\begin{tabular}{cccc}
{\footnotesize\textit{attempt 1}} &
{\footnotesize\textit{attempt 2}} &
{\footnotesize\textit{attempt 3}} &
{\footnotesize\textit{attempt 4}} \\
\rbox{.araraconfig.yaml} &
\rbox{araraconfig.yaml} &
\rbox{.arararc.yaml} &
\rbox{arararc.yaml}
\end{tabular}
\par}

\vspace{1.4em}

From version 4.0 on, \arara\ provides two approaches regarding the location of a configuration file. They dictate how the execution should behave and happen from a user perspective, and are described as follows.

\begin{description}
\item[global configuration file] For this approach, the configuration file should be located at \abox[araracolour]{USER\_HOME} which is the home directory of the current user. All subsequent executions of \arara\ will read this configuration file and apply the specified settings accordingly. However, it is important to note that this approach has the lowest lookup priority, which means that a local configuration, presented as follows, will always supersede a global counterpart.

\item[local configuration file] For this approach, the configuration file should be located at \abox[araracolour]{USER\_DIR} which is the working directory associated with the current execution. Such directory can also be interpreted as the one relative to the processed file. This approach offers a project-based solution for complex workflows, e.g, a thesis or a book. However, \arara\ must be executed within the working directory, or the local configuration file lookup will fail. Observe that this approach has the highest lookup priority, which means that it will always supersede a global configuration.
\end{description}

\begin{messagebox}{Beware of empty configuration files}{attentioncolour}{\icattention}{black}
A configuration file should never be empty, otherwise \arara\ will complain about it. Make sure to populate it with at least one key, or do not write a configuration file at all. The available options are described in Section~\ref{foo}, page~\pageref{foo}.
\end{messagebox}

% see which CF is used: log file

%\begin{codebox}{Terminal}{teal}{\icnote}{white}
%\end{codebox}

%\begin{ncodebox}{Source file}{teal}{\icnote}{white}{}
%\end{ncodebox}

%\begin{messagebox}{}{araracolour}{\icok}{white}
%\end{messagebox}

% !TeX root = ../arara-manual.tex
\chapter{Logging}
\label{chap:logging}

A log file is a special type of file that records events that occur in a software run. To this end, \arara\ provides such feature through the \opbox{{-}log} command line option (Section~\ref{foo}, page~\pageref{foo}) or the equivalent key in the configuration file (Section~\ref{foo}, page~\pageref{foo}). This chapter covers the basic structure of a typical log file provided by our tool, including the important sections that can be used to identify potential issues. The following example is used to illustrate the logging feature:

\begin{ncodebox}{Source file}{teal}{\icnote}{white}{doc12.tex}
% arara: pdftex
% arara: clean: { extensions: [ log ] }
Hello world.
\bye
\end{ncodebox}

When running the tool on the previous example with the \opbox{{-}log} command line option (otherwise, the logging framework will not provide a file at all), we will obtain the expected \rbox{arara.log} log file containing the most significant events that happened during this particular execution. Note that the timestamps were deliberated removed from the log entries in order to declutter the output, and line breaks were included as to easily spot each entry,

\section{System information}
\label{sec:systeminformation}

The very first entry to appear in the log file is the current version of \arara\ followed by a revision number. The revision number acts as a counter for the last review on the major version. The counter starts at 1 to denote the first release in the version 4.0 series. The revision number is also important to indicate possible new features introduced later on in the application.

\begin{codebox}{Log file}{teal}{\icnote}{white}
Welcome to arara 4.0 (revision 1)!
\end{codebox}

The following entries in the log file are the absolute path of the current deployment of \arara\ (line 1), details about the current Java virtual machine (namely, vendor and absolute path, in lines 2 and 3, respectively), the underlying operating system information (namely, system name, architecture and eventually the kernel version, in line 4), home and working directories (lines 5 and 6, respectively), and the absolute path of the applied configuration file, if any (line 7). This section is very important to help tracking possible issues related to the underlying operating system and the tool configuration itself.

\begin{codebox}{Log file}{teal}{\icnote}{white}
::: arara @ /opt/paulo/arara
::: Java 1.8.0_171, Oracle Corporation
::: /usr/lib/jvm/java-1.8.0-openjdk-1.8.0.171-4.b10.fc28.x86_64/jre
::: Linux, amd64, 4.16.12-300.fc28.x86_64
::: user.home @ /home/paulo
::: user.dir @ /home/paulo/Testes
::: CF @ [none]
\end{codebox}

\begin{messagebox}{A privacy note}{araracolour}{\icok}{white}
\setlength{\parskip}{1em}
I understand that the previous entries containing information about the underlying operating system might pose as a privacy threat to some users. However, it is worth noting that \arara\ does not share any sensitive information about your system, the entries are listed in the log file for debugging purposes only, locally in your computer.

From experience, these entries greatly help our users to track down errors in the execution, as well as learning more about the underlying operating system. However, be mindful of sharing your log file! Since the log file contains structured sections, it is highly advisable to selectively choose the ones relevant to the current discussion.
\end{messagebox}

It is important to observe that localized messages are also applied to the log file. If a language other than English is selected, either through the \opbox{{-}language} command line option or the equivalent key in the configuration file, the logging framework will honour the current setting and entries will be available in the specified language. Having a log file on your own language might mitigate the traumatic experience of error tracking for \TeX\ newbies.

\section{Directive extraction}
\label{sec:directiveextraction}

The following section in the log file refers to file information and directive extration. First, as the terminal output counterpart, the tool will display details about the file being processed, including size and modification status:

\begin{codebox}{Log file}{teal}{\icnote}{white}
Processing 'doc12.tex' (size: 74 bytes, last modified:
06/02/2018 05:36:40), please wait.
\end{codebox}

%\begin{codebox}{Terminal}{teal}{\icnote}{white}
%\end{codebox}

%\begin{ncodebox}{Source file}{teal}{\icnote}{white}{}
%\end{ncodebox}

%\begin{messagebox}{}{araracolour}{\icok}{white}
%\end{messagebox}

% !TeX root = ../arara-manual.tex
\chapter{Methods}
\label{chap:methods}

\arara\ features several helper methods available in directive conditional and rule contexts which provide interesting features for enhancing the user experience, as well as improving the automation itself. This chapter provides a list of such methods. It is important to observe that virtually all classes from the Java runtime environment can be used within MVEL expressions, so your milleage might vary.

\begin{messagebox}{A note on writing code}{araracolour}{\icok}{white}
% TODO fix reference
As seen in Section~\ref{foo}, on page~\pageref{foo}, Java and MVEL code be used interchangeably within expressions and orb tags, including instantiation of classes into objects and invocation of methods. However, be mindful of explicitly importing Java packages and classes through the classic \rbox{import} statement, as MVEL does not automatically handle imports, or an exception will surely be raised. Alternatively, you can provide the full qualified name to classes as well.
\end{messagebox}

Methods are listed with their complete signatures, including potential  parameters and corresponding types. Also, the return type of a method is denoted by \rrbox{type} and refers to a typical Java data type (either class or primitive). Do not worry too much, as there are illustrative examples. A method available in the directive conditional context will be marked by \ctbox{C} next to the corresponding signature. Similarly, an entry marked by \ctbox{R} denotes that the corresponding method is available in the rule context.

\section{Files}
\label{sec:files}

This section introduces methods related to file handling, searching and hashing. It is important to observe that no exception is thrown in case of an anomalous method call. In this particular scenario, the methods return empty references, when applied.

\begin{description}
% TODO fix reference
\item[\mdbox{R}{getOriginalFile()}{String}] This method returns the original file name, as plain string, regardless of a potential override through the special \abox{files} parameter in the directive mapping, as seen in Section~\ref{foo}, on page~\pageref{foo}.

\begin{codebox}{Example}{teal}{\icnote}{white}
if (file == getOriginalFile()) {
    System.out.println("The 'file' variable
       was not overriden.");
}
\end{codebox}

% TODO fix reference
\item[\mdbox{R}{getOriginalReference()}{File}] This method returns the original file reference, as a \rbox{File} object, regardless of a potential reference override indirectly through the special \abox{files} parameter in the directive mapping, as seen in Section~\ref{foo}, on page~\pageref{foo}.

\begin{codebox}{Example}{teal}{\icnote}{white}
if (reference.equals(getOriginalFile())) {
    System.out.println("The 'reference' variable
       was not overriden.");
}
\end{codebox}

\item[\mddbox{C}{R}{currentFile()}{File}] This method returns the file reference, as a \rbox{File} object, for the current directive. It is important to observe that, from version 4.0 on, \arara\ replicates the directive when the special \abox{files} parameter is detected amongst the parameters, so each instance will have a different reference.

\begin{codebox}{Example}{teal}{\icnote}{white}
% arara: pdflatex if currentFile().getName() == 'thesis.tex'
\end{codebox}

\item[\mddbox{C}{R}{toFile(String reference)}{File}] This method returns a file (or directory) reference, as a \rbox{File} object, based on the provided string. Note that such string can refer to either a relative entry or a complete, absolute path. It is worth mentioning that, in Java, despite the curious name, a \rbox{File} object can be assigned to either a file or a directory.

\begin{codebox}{Example}{teal}{\icnote}{white}
f = toFile('thesis.tex');
\end{codebox}

\item[\mdbox{R}{getBasename(File file)}{String}] This method returns the base name (i.e, the name without the associated extension) of the provided \rbox{File} reference, as a string. Observe that this method ignores a potential path reference when extracting the base name. For a complete base name extraction with full path support, please refer to the \mtbox{getFullBasename} methods. Also, this method will throw an exception if the provided reference is not a proper file.

\begin{codebox}{Example}{teal}{\icnote}{white}
basename = getBasename(toFile('thesis.tex'));
\end{codebox}

\item[\mdbox{R}{getBasename(String reference)}{String}] This method returns the base name (i.e, the name without the associated extension) of the provided \rbox{String} reference, as a string. Observe that this method ignores a potential path reference when extracting the base name. For a complete base name extraction with full path support, please refer to the \mtbox{getFullBasename} methods.

\begin{codebox}{Example}{teal}{\icnote}{white}
basename = getBasename('thesis.tex');
\end{codebox}

\item[\mdbox{R}{getFullBasename(File file)}{String}] This method returns the full base name (i.e, the name without the associated extension, as well as the potential path reference) of the provided \rbox{File} reference, as a string. This method will throw an exception if the provided reference is not a proper file.

\begin{codebox}{Example}{teal}{\icnote}{white}
basename = getFullBasename(toFile('/home/paulo/thesis.tex'));
\end{codebox}

\item[\mdbox{R}{getFullBasename(String reference)}{String}] This method returns the full base name (i.e, the name without the associated extension, as well as the potential path reference) of the provided \rbox{String} reference, as a string. As the path discovery requires an underlying file conversion, this method will throw an exception if the provided reference is not a proper file.

\begin{codebox}{Example}{teal}{\icnote}{white}
basename = getFullBasename('/home/paulo/thesis.tex');
\end{codebox}

\item[\mdbox{R}{getFiletype(File file)}{String}] This method returns the file type (i.e, the associated extension specified as a suffix to the name, typically delimited with a full stop) of the provided \rbox{File} reference, as a string. This method will throw an exception if the provided reference is not a proper file. An empty string is returned if, and only if, the provided file name has no associated extension.

\begin{codebox}{Example}{teal}{\icnote}{white}
extension = getFiletype(toFile('thesis.pdf'));
\end{codebox}

\item[\mdbox{R}{getFiletype(String reference)}{String}] This method returns the file type (i.e, the associated extension specified as a suffix to the name, typically delimited with a full stop) of the provided \rbox{String} reference, as a string. An empty string is returned if, and only if, the provided file name has no associated extension.

\begin{codebox}{Example}{teal}{\icnote}{white}
extension = getFiletype('thesis.pdf');
\end{codebox}

\item[\mddbox{C}{R}{exists(File file)}{boolean}] This method, as the name implies, returns a boolean value according to whether the provided \rbox{File} reference exists. Observe that the provided reference can be either a file or a directory.

\begin{codebox}{Example}{teal}{\icnote}{white}
% arara: bibtex if exists(toFile('references.bib'))
\end{codebox}

\item[\mddbox{C}{R}{exists(String extension)}{boolean}] This method returns a boolean value according to whether the base name of the \mtbox{currentFile} reference (i.e, the name without the associated extension) as a string concatenated with the provided \rbox{String} extension exists. This method eases the checking of files which share the current file name modulo extension (e.g, log and auxiliary files). Note that the provided string refers to the extension, not the file name.

\begin{codebox}{Example}{teal}{\icnote}{white}
% arara: pdftex if exists('tex')
\end{codebox}

\item[\mddbox{C}{R}{missing(File file)}{boolean}] This method, as the name implies, returns a boolean value according to whether the provided \rbox{File} reference does not exist. It is important to observe that the provided reference can be either a file or a directory.

\begin{codebox}{Example}{teal}{\icnote}{white}
% arara: pdftex if missing(toFile('thesis.pdf'))
\end{codebox}

\item[\mddbox{C}{R}{missing(String extension)}{boolean}] This method returns a boolean value according to whether the base name of the \mtbox{currentFile} reference (i.e, the name without the associated extension) as a string concatenated with the provided \rbox{String} extension does not exist. This method eases the checking of files which share the current file name modulo extension (e.g, log and auxiliary files). Note that the provided string refers to the extension, not the file name.

\begin{codebox}{Example}{teal}{\icnote}{white}
% arara: pdftex if missing('pdf')
\end{codebox}

% TODO fix reference
\item[\mddbox{C}{R}{changed(File file)}{boolean}] This method returns a boolean value according to whether the provided \rbox{File} reference has changed since last verification, based on a traditional cyclic redundancy check. The file reference, as well as the associated hash, is stored in a XML database file named \rbox{arara.xml} located at the same directory of the current file (the database name can be overriden in the configuration file, as discussed in Section~\ref{foo}, on page~\pageref{foo}). The method semantics (including the return values) is presented as follows.

\vspace{1em}

{\centering\small
\setlength\tabcolsep{0.8em}
\begin{tabular}{@{}ccccc@{}}
\toprule
\emph{file exists?} & \emph{entry exists?} &
\emph{has changed?} & \emph{DB action} &
\emph{result} \\
\midrule
\cbyes{-2} & \cbyes{-2} & \cbyes{-2} & update & \cbyes{-2} \\
\cbyes{-2} & \cbyes{-2} & \cbno{-2} & --- & \cbno{-2} \\
\cbyes{-2} & \cbno{-2} & --- & insert & \cbyes{-2} \\
\cbno{-2} & \cbno{-2} & --- & --- & \cbno{-2} \\
\cbno{-2} & \cbyes{-2} & --- & remove & \cbyes{-2} \\
\bottomrule
\end{tabular}\par}

\vspace{1.4em}

It is important to observe that this method \emph{always} performs a database operation, either an insertion, removal or update on the corresponding entry. When using \mtbox{changed} within a logical expression, make sure the evaluation order is correct, specially regarding the use of short-circuiting operations. In some scenarios, order does matter.

\begin{codebox}{Example}{teal}{\icnote}{white}
% arara: pdflatex if changed(toFile('thesis.tex'))
\end{codebox}

\begin{messagebox}{Short-circuit evaluation}{araracolour}{\icok}{white}
According to the \href{https://en.wikipedia.org/wiki/Short-circuit_evaluation}{Wikipedia entry}, a \emph{short-circuit evaluation} is the semantics of some boolean operators in some programming languages in which the second argument is executed or evaluated only if the first argument does not suffice to determine the value of the expression. In Java (and consequently MVEL), both short-circuit and standard boolean operators are available.
\end{messagebox}

\begin{messagebox}{CRC as a hashing algorithm}{attentioncolour}{\icattention}{black}
\arara\ internally relies on a CRC32 implementation for file hashing. This particular choice, although not designed for hashing, offers an interesting tradeoff between speed and quality. Besides, since it is not computationally expensive as strong algorithms such as MD5 and SHA1, CRC32 can be used for hashing typical \TeX\ documents and plain text files with little to no collisions.
\end{messagebox}

% TODO fix reference
\item[\mddbox{C}{R}{changed(String extension)}{boolean}] This method returns a boolean value according to whether the base name of the \mtbox{currentFile} reference (i.e, the name without the associated extension) as a string concatenated with the provided \rbox{String} extension has changed since last verification, based on a traditional cyclic redundancy check. The file reference, as well as the associated hash, is stored in a XML database file named \rbox{arara.xml} located at the same directory of the current file (the database name can be overriden in the configuration file, as discussed in Section~\ref{foo}, on page~\pageref{foo}). The method semantics (including the return values) is presented as follows.

\vspace{1em}

{\centering\small
\setlength\tabcolsep{0.8em}
\begin{tabular}{@{}ccccc@{}}
\toprule
\emph{file exists?} & \emph{entry exists?} &
\emph{has changed?} & \emph{DB action} &
\emph{result} \\
\midrule
\cbyes{-2} & \cbyes{-2} & \cbyes{-2} & update & \cbyes{-2} \\
\cbyes{-2} & \cbyes{-2} & \cbno{-2} & --- & \cbno{-2} \\
\cbyes{-2} & \cbno{-2} & --- & insert & \cbyes{-2} \\
\cbno{-2} & \cbno{-2} & --- & --- & \cbno{-2} \\
\cbno{-2} & \cbyes{-2} & --- & remove & \cbyes{-2} \\
\bottomrule
\end{tabular}\par}

\vspace{1.4em}

It is important to observe that this method \emph{always} performs a database operation, either an insertion, removal or update on the corresponding entry. When using \mtbox{changed} within a logical expression, make sure the evaluation order is correct, specially regarding the use of short-circuiting operations. In some scenarios, order does matter.

\begin{codebox}{Example}{teal}{\icnote}{white}
% arara: pdflatex if changed('tex')
\end{codebox}

% TODO fix reference
\item[\mddbox{C}{R}{unchanged(File file)}{boolean}] This method returns a boolean value according to whether the provided \rbox{File} reference has not changed since last verification, based on a traditional cyclic redundancy check. The file reference, as well as the associated hash, is stored in a XML database file named \rbox{arara.xml} located at the same directory of the current file (the database name can be overriden in the configuration file, as discussed in Section~\ref{foo}, on page~\pageref{foo}). The method semantics (including the return values) is presented as follows.

\vspace{1em}

{\centering\small
\setlength\tabcolsep{0.8em}
\begin{tabular}{@{}ccccc@{}}
\toprule
\emph{file exists?} & \emph{entry exists?} &
\emph{has changed?} & \emph{DB action} &
\emph{result} \\
\midrule
\cbyes{-2} & \cbyes{-2} & \cbyes{-2} & update & \cbno{-2} \\
\cbyes{-2} & \cbyes{-2} & \cbno{-2} & --- & \cbyes{-2} \\
\cbyes{-2} & \cbno{-2} & --- & insert & \cbno{-2} \\
\cbno{-2} & \cbno{-2} & --- & --- & \cbyes{-2} \\
\cbno{-2} & \cbyes{-2} & --- & remove & \cbno{-2} \\
\bottomrule
\end{tabular}\par}

\vspace{1.4em}

It is important to observe that this method \emph{always} performs a database operation, either an insertion, removal or update on the corresponding entry. When using \mtbox{unchanged} within a logical expression, make sure the evaluation order is correct, specially regarding the use of short-circuiting operations. In some scenarios, order does matter.

\begin{codebox}{Example}{teal}{\icnote}{white}
% arara: pdflatex if !unchanged(toFile('thesis.tex'))
\end{codebox}

% TODO fix reference
\item[\mddbox{C}{R}{unchanged(String extension)}{boolean}] This method returns a boolean value according to whether the base name of the \mtbox{currentFile} reference (i.e, the name without the associated extension) as a string concatenated with the provided \rbox{String} extension has not changed since last verification, based on a traditional cyclic redundancy check. The file reference, as well as the associated hash, is stored in a XML database file named \rbox{arara.xml} located at the same directory of the current file (the database name can be overriden in the configuration file, as discussed in Section~\ref{foo}, on page~\pageref{foo}). The method semantics (including the return values) is presented as follows.

\vspace{1em}

{\centering\small
\setlength\tabcolsep{0.8em}
\begin{tabular}{@{}ccccc@{}}
\toprule
\emph{file exists?} & \emph{entry exists?} &
\emph{has changed?} & \emph{DB action} &
\emph{result} \\
\midrule
\cbyes{-2} & \cbyes{-2} & \cbyes{-2} & update & \cbno{-2} \\
\cbyes{-2} & \cbyes{-2} & \cbno{-2} & --- & \cbyes{-2} \\
\cbyes{-2} & \cbno{-2} & --- & insert & \cbno{-2} \\
\cbno{-2} & \cbno{-2} & --- & --- & \cbyes{-2} \\
\cbno{-2} & \cbyes{-2} & --- & remove & \cbno{-2} \\
\bottomrule
\end{tabular}\par}

\vspace{1.4em}

It is important to observe that this method \emph{always} performs a database operation, either an insertion, removal or update on the corresponding entry. When using \mtbox{unchanged} within a logical expression, make sure the evaluation order is correct, specially regarding the use of short-circuiting operations. In some scenarios, order does matter.

\begin{codebox}{Example}{teal}{\icnote}{white}
% arara: pdflatex if !unchanged('tex')
\end{codebox}

\item[\mdbox{R}{writeToFile(File file, String text, boolean append)}{boolean}] This method performs a write operation based on the provided parameters. In this case, the method writes the \rbox{String} text to the \rbox{File} reference and returns a boolean value according to whether such operation was successful. The third parameter holds a \rbox{boolean} value and acts as a switch indicating whether the text should be appended to the existing content of the provided file. Keep in mind that the existing content of a file is always overwritten if such switch is disabled. Also, note that the switch has no effect if the file is being created at that moment. It is important to observe that this method does not raise any exception.

\begin{codebox}{Example}{teal}{\icnote}{white}
result = writeToFile(toFile('foo.txt'), 'hello world', false);
\end{codebox}

\begin{messagebox}{Read and write operations in Unicode}{attentioncolour}{\icattention}{black}
\arara\ \emph{always} uses Unicode as encoding format for read and write operations. This decision is deliberate as a means to offer a consistent representation and handling of text. Unicode can be implemented by different character encodings. In our case, the tool relies on UTF-8, which uses one byte for the first 128 code points, and up to 4 bytes for other characters. The first 128 Unicode code points are the ASCII characters, which means that any ASCII text is also a UTF-8 text.
\end{messagebox}

\begin{messagebox}{File system permissions}{attentioncolour}{\icattention}{black}
Most file systems have methods to assign permissions or access rights to specific users and groups of users. These permissions control the ability of the users to view, change, navigate, and execute the contents of the file system. Keep in mind that read and write operations depend on such permissions.
\end{messagebox}

\item[\mdbox{R}{writeToFile(String reference, String text, boolean append)}{boolean}] This method performs a write operation based on the provided parameters. In this case, the method writes the \rbox{String} text to the \rbox{String} reference and returns a boolean value according to whether such operation was successful. The third parameter holds a \rbox{boolean} value and acts as a switch indicating whether the text should be appended to the existing content of the provided file. Keep in mind that the existing content of a file is always overwritten if such switch is disabled. Also, note that the switch has no effect if the file is being created at that moment. It is important to observe that this method does not raise any exception.

\begin{codebox}{Example}{teal}{\icnote}{white}
result = writeToFile('foo.txt', 'hello world', false);
\end{codebox}

\item[\mdbox{R}{writeToFile(File file, List<String> lines, boolean append)}{boolean}] This method performs a write operation based on the provided parameters. In this case, the method writes the \rbox{List<String>} lines to the \rbox{File} reference and returns a boolean value according to whether such operation was successful. The third parameter holds a \rbox{boolean} value and acts as a switch indicating whether the text should be appended to the existing content of the provided file. Keep in mind that the existing content of a file is always overwritten if such switch is disabled. Also, note that the switch has no effect if the file is being created at that moment. It is important to observe that this method does not raise any exception.

\begin{codebox}{Example}{teal}{\icnote}{white}
result = writeToFile(toFile('foo.txt'),
         [ 'hello world', 'how are you?' ], false);
\end{codebox}

\item[\mdbox{R}{\parbox{0.51\textwidth}{writeToFile(String reference,\\\hspace*{1em} List<String> lines, boolean append)}}{boolean}] This method performs a write operation based on the provided parameters. In this case, the method writes the \rbox{List<String>} lines to the \rbox{String} reference and returns a boolean value according to whether such operation was successful. The third parameter holds a \rbox{boolean} value and acts as a switch indicating whether the text should be appended to the existing content of the provided file. Keep in mind that the existing content of a file is always overwritten if such switch is disabled. Also, note that the switch has no effect if the file is being created at that moment. It is important to observe that this method does not raise any exception.

\begin{codebox}{Example}{teal}{\icnote}{white}
result = writeToFile('foo.txt', [ 'hello world',
         'how are you?' ], false);
\end{codebox}

\item[\mdbox{R}{readFromFile(File file)}{List<String>}] This method performs a read operation based on the provided parameter. In this case, the method reads the content from the \rbox{File} reference and returns a \rbox{List<String>} object representing the lines as a list of strings. If the reference does not exist or an exception is raised due to access permission constraints, the \mtbox{readFromFile} method returns an empty list. Keep in mind that, as a design decision, UTF-8 is \emph{always} used as character encoding for read operations.

\begin{codebox}{Example}{teal}{\icnote}{white}
lines = readFromFile(toFile('foo.txt'));
\end{codebox}

\item[\mdbox{R}{readFromFile(String reference)}{List<String>}] This method performs a read operation based on the provided parameter. In this case, the method reads the content from the \rbox{String} reference and returns a \rbox{List<String>} object representing the lines as a list of strings. If the reference does not exist or an exception is raised due to access permission constraints, the \mtbox{readFromFile} method returns an empty list. Keep in mind that, as a design decision, UTF-8 is \emph{always} used as character encoding for read operations.

\begin{codebox}{Example}{teal}{\icnote}{white}
lines = readFromFile('foo.txt');
\end{codebox}

\item[\mdbox{R}{\parbox{0.61\textwidth}{listFilesByExtensions(File file,\\\hspace*{1em} List<String> extensions, boolean recursive)}}{List<File>}] This methods performs a file search operation based on the provided parameters. In this case, the method list all files from the provided \rbox{File} reference according to the \rbox{List<String>} extensions as a list of strings, and returns a \rbox{List<File>} object representing all matching files. The leading full stop in each extension must be omitted, unless it is part of the search pattern. The third parameter holds a \rbox{boolean} value and acts as a switch indicating whether the search must be recursive, i.e, whether all subdirectories must be searched as well. If the reference is not a proper directory or an exception is raised due to access permission constraints, the \mtbox{listFilesByExtensions} method returns an empty list.

\begin{codebox}{Example}{teal}{\icnote}{white}
files = listFilesByExtensions(toFile('/home/paulo/Documents'),
        [ 'aux', 'log' ], false);
\end{codebox}

\item[\mdbox{R}{\parbox{0.61\textwidth}{listFilesByExtensions(String reference,\\\hspace*{1em} List<String> extensions, boolean recursive)}}{List<File>}] This methods performs a file search operation based on the provided parameters. In this case, the method list all files from the provided \rbox{String} reference according to the \rbox{List<String>} extensions as a list of strings, and returns a \rbox{List<File>} object representing all matching files. The leading full stop in each extension must be omitted, unless it is part of the search pattern. The third parameter holds a \rbox{boolean} value and acts as a switch indicating whether the search must be recursive, i.e, whether all subdirectories must be searched as well. If the reference is not a proper directory or an exception is raised due to access permission constraints, the \mtbox{listFilesByExtensions} method returns an empty list.

\begin{codebox}{Example}{teal}{\icnote}{white}
files = listFilesByExtensions('/home/paulo/Documents',
        [ 'aux', 'log' ], false);
\end{codebox}

\item[\mdbox{R}{\parbox{0.59\textwidth}{listFilesByPatterns(File file,\\\hspace*{1em} List<String> patterns, boolean recursive)}}{List<File>}] This methods performs a file search operation based on the provided parameters. In this case, the method list all files from the provided \rbox{File} reference according to the \rbox{List<String>} patterns as a list of strings, and returns a \rbox{List<File>} object representing all matching files. The pattern specification is presented as follows. The third parameter holds a \rbox{boolean} value and acts as a switch indicating whether the search must be recursive, i.e, whether all subdirectories must be searched as well. If the reference is not a proper directory or an exception is raised due to access permission constraints, the \mtbox{listFilesByPatterns} method returns an empty list. It is very important to observe that this file search operation might be slow depending on the provided directory. It is highly advisable to not rely on recursive searches whenever possible.

\begin{messagebox}{Patterns for file search operations}{araracolour}{\icattention}{white}
\arara\ employs wildcard filters as patterns for file search operations. Testing is case-sensitive by default. The wildcard matcher uses the characters \rbox[araracolour]{?} and \rbox[araracolour]{*} to represent a single or multiple wildcard characters. This is the same as often found on typical terminals.
\end{messagebox}

\begin{codebox}{Example}{teal}{\icnote}{white}
files = listFilesByPatterns(toFile('/home/paulo/Documents'),
        [ '*.tex', 'foo?.txt' ], false);
\end{codebox}


\item[\mdbox{R}{\parbox{0.59\textwidth}{listFilesByPatterns(String reference,\\\hspace*{1em} List<String> patterns, boolean recursive)}}{List<File>}] This methods performs a file search operation based on the provided parameters. In this case, the method list all files from the provided \rbox{String} reference according to the \rbox{List<String>} patterns as a list of strings, and returns a \rbox{List<File>} object representing all matching files. The pattern specification follows a wildcard filter. The third parameter holds a \rbox{boolean} value and acts as a switch indicating whether the search must be recursive, i.e, whether all subdirectories must be searched as well. If the reference is not a proper directory or an exception is raised due to access permission constraints, the \mtbox{listFilesByPatterns} method returns an empty list. It is very important to observe that this file search operation might be slow depending on the provided directory. It is highly advisable to not rely on recursive searches whenever possible.

\begin{codebox}{Example}{teal}{\icnote}{white}
files = listFilesByPatterns('/home/paulo/Documents',
        [ '*.tex', 'foo?.txt' ], false);
\end{codebox}
\end{description}

As the methods presented in this section have transparent error handling, the writing of rules and conditionals becomes more fluent and not too complex for the typical user.

\section{Conditional flow}
\label{sec:conditionalflow}

This section introduces methods related to conditional flow based on \emph{natural boolean values}, i.e, words that semantically represent truth and falsehood signs. Such concept provides a friendly representation of boolean values and eases the use of switches in directive parameters. The tool relies on the following set of natural boolean values:

\vspace{1em}

{\centering
\setlength\tabcolsep{0.2em}
\begin{tabular}{ccccccccccc}
\cbyes{-2} &
\rbox[araracolour]{\hphantom{w}yes\hphantom{w}} &
\rbox[araracolour]{\hphantom{w}true\hphantom{w}} &
\rbox[araracolour]{\hphantom{w}1\hphantom{w}} &
\rbox[araracolour]{\hphantom{w}on\hphantom{w}} &
\hspace{2em} &
\cbno{-2} &
\rbox[warningcolour]{\hphantom{w}no\hphantom{w}} &
\rbox[warningcolour]{\hphantom{w}false\hphantom{w}} &
\rbox[warningcolour]{\hphantom{w}0\hphantom{w}} &
\rbox[warningcolour]{\hphantom{w}off\hphantom{w}}
\end{tabular}\par}

\vspace{1.4em}

All elements from the provided set of natural boolean values can be used interchangeably in directive parameters. It is important to observe that, from version 4.0 on, \arara\ throws an exception if a value absent from the set is provided to the methods described in this section.

\begin{description}
\item[\mdbox{R}{isTrue(String string)}{boolean}] This method returns a boolean value according to whether the provided \rbox{String} value is contained in the set of natural boolean values. It is worth mentioning that the verification is case insensitive, i.e, upper case and lower case symbols are treated as equivalent. If the provided value is an empty string, the method returns false.

\begin{codebox}{Example}{teal}{\icnote}{white}
result = isTrue('yes');
\end{codebox}

\item[\mdbox{R}{isFalse(String string)}{boolean}] This method returns a boolean value according to whether the provided \rbox{String} value is not contained in the set of natural boolean values. It is worth mentioning that the verification is case insensitive, i.e, upper case and lower case symbols are treated as equivalent. If the provided value is an empty string, the method returns false.

\begin{codebox}{Example}{teal}{\icnote}{white}
result = isFalse('off');
\end{codebox}

\item[\mdbox{R}{isTrue(String string, Object yes)}{Object}] This method checks if the first parameter is contained in the set of natural boolean values. If the result holds true, the second parameter is returned. Otherwise, an empty string is returned. It is worth mentioning that the verification is case insensitive, i.e, upper case and lower case symbols are treated as equivalent. If the first parameter is an empty string, the method returns false.

\begin{codebox}{Example}{teal}{\icnote}{white}
result = isTrue('on', [ 'ls', '-la' ]);
\end{codebox}

\item[\mdbox{R}{isFalse(String string, Object yes)}{Object}] This method checks if the first parameter is not contained in the set of natural boolean values. If the result holds true, the second parameter is returned. Otherwise, an empty string is returned. It is worth mentioning that the verification is case insensitive, i.e, upper case and lower case symbols are treated as equivalent. If the first parameter is an empty string, the method returns false.

\begin{codebox}{Example}{teal}{\icnote}{white}
result = isFalse('0', 'pwd');
\end{codebox}

\item[\mdbox{R}{isTrue(String string, Object yes, Object no)}{Object}] This method checks if the first parameter is contained in the set of natural boolean values. If the result holds true, the second parameter is returned. Otherwise, the third parameter is returned. It is worth mentioning that the verification is case insensitive, i.e, upper case and lower case symbols are treated as equivalent. If the first parameter is an empty string, the method returns false.

\begin{codebox}{Example}{teal}{\icnote}{white}
result = isTrue('on', [ 'ls', '-la' ], 'pwd');
\end{codebox}

\item[\mdbox{R}{isFalse(String string, Object yes, Object no)}{Object}] This method checks if the first parameter is not contained in the set of natural boolean values. If the result holds true, the second parameter is returned. Otherwise, the third parameter is returned. It is worth mentioning that the verification is case insensitive, i.e, upper case and lower case symbols are treated as equivalent. If the first parameter is an empty string, the method returns false.

\begin{codebox}{Example}{teal}{\icnote}{white}
result = isFalse('0', 'pwd', 'ps');
\end{codebox}

\item[\mdbox{R}{\parbox{0.45\textwidth}{isTrue(String string, Object yes,\\\hspace*{1em} Object no, Object fallback)}}{Object}] This method checks if the first parameter is contained in the set of natural boolean values. If the result holds true, the second parameter is returned. Otherwise, the third parameter is returned. It is worth mentioning that the verification is case insensitive, i.e, upper case and lower case symbols are treated as equivalent. If the first parameter is an empty string, the method returns the fourth parameter as default value.

\begin{codebox}{Example}{teal}{\icnote}{white}
result = isTrue('on', 'ls', 'pwd', 'who');
\end{codebox}

\item[\mdbox{R}{\parbox{0.46\textwidth}{isFalse(String string, Object yes,\\\hspace*{1em} Object no, Object fallback)}}{Object}] This method checks if the first parameter is not contained in the set of natural boolean values. If the result holds true, the second parameter is returned. Otherwise, the third parameter is returned. It is worth mentioning that the verification is case insensitive, i.e, upper case and lower case symbols are treated as equivalent. If the first parameter is an empty string, the method returns the fourth parameter as default value.

\begin{codebox}{Example}{teal}{\icnote}{white}
result = isFalse('0', 'pwd', 'ps', 'ls');
\end{codebox}

\item[\mdbox{R}{isTrue(boolean value, Object yes)}{Object}] This method evaluates the first parameter as a boolean expression. If the result holds true, the second parameter is returned. Otherwise, an empty string is returned.

\begin{codebox}{Example}{teal}{\icnote}{white}
result = isTrue(1 == 1, 'yes');
\end{codebox}

\item[\mdbox{R}{isFalse(boolean value, Object yes)}{Object}] This method evaluates the first parameter as a boolean expression. If the result holds false, the second parameter is returned. Otherwise, an empty string is returned.

\begin{codebox}{Example}{teal}{\icnote}{white}
result = isFalse(1 != 1, 'yes');
\end{codebox}

\item[\mdbox{R}{isTrue(boolean value, Object yes, Object no)}{Object}] This method evaluates the first parameter as a boolean expression. If the result holds true, the second parameter is returned. Otherwise, the third parameter is returned.

\begin{codebox}{Example}{teal}{\icnote}{white}
result = isTrue(1 == 1, 'yes', 'no');
\end{codebox}

\item[\mdbox{R}{isFalse(boolean value, Object yes, Object no)}{Object}] This method evaluates the first parameter as a boolean expression. If the result holds false, the second parameter is returned. Otherwise, the third parameter is returned.

\begin{codebox}{Example}{teal}{\icnote}{white}
result = isFalse(1 != 1, 'yes', 'no');
\end{codebox}
\end{description}

Supported by the concept of natural boolean values, the methods presented in this section ease the use of switches in directive parameters and can be adopted as valid alternatives for traditional conditional flows, when applied.

\section{Strings}
\label{sec:strings}

String manipulation constitutes one of the foundations of rule interpretation in our tool. This section introduces methods for handling such types, as a means to offer high level constructs for users.

\begin{description}
\item[\mdbox{R}{isEmpty(String string)}{boolean}] This method returns a boolean value according to whether the provided \rbox{String} value is empty, i.e, the string length is equal to zero.

\begin{codebox}{Example}{teal}{\icnote}{white}
result = isEmpty('not empty');
\end{codebox}

\item[\mdbox{R}{isNotEmpty(String string)}{boolean}] This method returns a boolean value according to whether the provided \rbox{String} value is not empty, i.e, the string length is greater than zero.

\begin{codebox}{Example}{teal}{\icnote}{white}
result = isNotEmpty('not empty');
\end{codebox}

\item[\mdbox{R}{isEmpty(String string, Object yes)}{boolean}] This method checks if the first parameter is empty, i.e, if the string length is equal to zero. If the result holds true, the second parameter is returned. Otherwise, an empty string is returned.

\begin{codebox}{Example}{teal}{\icnote}{white}
result = isEmpty('not empty', 'ps');
\end{codebox}

\item[\mdbox{R}{isNotEmpty(String string, Object yes)}{boolean}] This method checks if the first parameter is not empty, i.e, if the string length is greater than zero. If the result holds true, the second parameter is returned. Otherwise, an empty string is returned.

\begin{codebox}{Example}{teal}{\icnote}{white}
result = isNotEmpty('not empty', 'ls');
\end{codebox}

\item[\mdbox{R}{isEmpty(String string, Object yes, Object no)}{boolean}] This method checks if the first parameter is empty, i.e, if the string length is equal to zero. If the result holds true, the second parameter is returned. Otherwise, the third parameter is returned.

\begin{codebox}{Example}{teal}{\icnote}{white}
result = isEmpty('not empty', 'ps', 'ls');
\end{codebox}

\item[\mdbox{R}{isNotEmpty(String string, Object yes, Object no)}{boolean}] This method checks if the first parameter is not empty, i.e, if the string length is greater than zero. If the result holds true, the second parameter is returned. Otherwise, the third parameter is returned.

\begin{codebox}{Example}{teal}{\icnote}{white}
result = isNotEmpty('not empty', 'ls', 'ps');
\end{codebox}

\item[\mdbox{R}{buildString(Object... objects)}{String}] This method returns a string based on the provided array of objects, separating each element by one blank space. It is important to observe that empty values are not considered. Also, note that the object array is denoted by a  comma-separated sequence of elements in the actual method call, resulting in a variable number of parameters.

\begin{codebox}{Example}{teal}{\icnote}{white}
result = buildString('a', 'b', 'c', 'd');
\end{codebox}

\item[\mdbox{R}{trimSpaces(String string)}{String}] This method trims spaces from the provided parameter, i.e, leading and trailing spaces in the \rbox{String} reference are removed, and returns the resulting string. It is important to observe that superfluous spaces along the string are not removed at all.

\begin{codebox}{Example}{teal}{\icnote}{white}
result = trimSpaces('   hello world   ');
\end{codebox}

\item[\mdbox{R}{addQuotes(String string)}{String}] This method returns the provided parameter enclosed in double quotes, as a plain string. It is important to observe that there is no automatic quote handling.

\begin{codebox}{Example}{teal}{\icnote}{white}
result = addQuotes('to be or not to be');
\end{codebox}

\item[\mdbox{R}{replicatePattern(String pattern, List<Object> values)}{List<Object>}] This method replicates the provided pattern to each element of the second parameter and returns the resulting list. The pattern must contain exactly one placeholder. For instance, \rbox{\%s} denotes a string representation of the provided argument. Please refer to the \rbox{Formatter} class reference in the \href{https://docs.oracle.com/javase/7/docs/api/java/util/Formatter.html}{Java documentation} for more information on placeholders. This method raises an exception if an invalid pattern is applied.

\begin{codebox}{Example}{teal}{\icnote}{white}
names = replicatePattern('My name is %s', [ 'Brent', 'Nicola' ]);
\end{codebox}

\item[\mddbox{C}{R}{found(File file, String regex)}{boolean}] This method returns a boolean value according to whether the content of the provided \rbox{File} reference contains at least one match of the provided \rbox{String} regular expression. It is important to observe that this method raises an exception if an invalid regular expression is provided as parameter or if the provided file reference does not exist.

\begin{codebox}{Example}{teal}{\icnote}{white}
% arara: pdflatex while found(toFile('article.tex'),
% arara: --> 'undefined references')
\end{codebox}

\item[\mddbox{C}{R}{found(String extension, String regex)}{boolean}] This method returns a boolean value according to whether the content of the base name of the \mtbox{currentFile} reference (i.e, the name without the associated extension) as a string concatenated with the provided \rbox{String} extension contains at least one match of the provided \rbox{String} regular expression. It is important to observe that this method raises an exception if an invalid regular expression is provided as parameter or if the provided file reference does not exist.

\begin{codebox}{Example}{teal}{\icnote}{white}
% arara: pdflatex while found('tex', 'undefined references')
\end{codebox}
\end{description}

The string manipulation methods presented in this section constitute an interesting and straightforward approach to handling directive parameters without the usual verbosity in writing typical Java constructs.

\section{Operating systems}
\label{sec:operatingsystems}

This section introduces methods related to the underlying operating system detection, as a means to provide a straightforward approach to writing cross-platform rules.

\begin{description}
\item[\mdbox{R}{isWindows()}{boolean}] This method returns a boolean value according to whether the underlying operating system vendor is Microsoft Windows.

\begin{codebox}{Example}{teal}{\icnote}{white}
if (isWindows()) { System.out.println('Running Windows.'); }
\end{codebox}

\item[\mdbox{R}{isLinux()}{boolean}] This method returns a boolean value according to whether the underlying operating system vendor is a Linux instance.

\begin{codebox}{Example}{teal}{\icnote}{white}
if (isLinux()) { System.out.println('Running Linux.'); }
\end{codebox}

\item[\mdbox{R}{isMac()}{boolean}] This method returns a boolean value according to whether the underlying operating system vendor is Apple Mac OS.

\begin{codebox}{Example}{teal}{\icnote}{white}
if (isMac()) { System.out.println('Running Mac OS.'); }
\end{codebox}

\item[\mdbox{R}{isUnix()}{boolean}] This method returns a boolean value according to whether the underlying operating system vendor is any Unix variation.

\begin{codebox}{Example}{teal}{\icnote}{white}
if (isUnix()) { System.out.println('Running Unix.'); }
\end{codebox}

\item[\mdbox{R}{isAIX()}{boolean}] This method returns a boolean value according to whether the underlying operating system vendor is IBM AIX.

\begin{codebox}{Example}{teal}{\icnote}{white}
if (isAIX()) { System.out.println('Running AIX.'); }
\end{codebox}

\item[\mdbox{R}{isIrix()}{boolean}] This method returns a boolean value according to whether the underlying operating system vendor is Silicon Graphics Irix.

\begin{codebox}{Example}{teal}{\icnote}{white}
if (isIrix()) { System.out.println('Running Irix.'); }
\end{codebox}

\item[\mdbox{R}{isOS2()}{boolean}] This method returns a boolean value according to whether the underlying operating system vendor is IBM OS/2 Warp.

\begin{codebox}{Example}{teal}{\icnote}{white}
if (isOS2()) { System.out.println('Running OS/2 Warp.'); }
\end{codebox}

\item[\mdbox{R}{isSolaris()}{boolean}] This method returns a boolean value according to whether the underlying operating system vendor is Oracle Solaris.

\begin{codebox}{Example}{teal}{\icnote}{white}
if (isSolaris()) { System.out.println('Running Solaris.'); }
\end{codebox}

\item[\mdbox{R}{isCygwin()}{boolean}] This method returns a boolean value according to whether the underlying operating system vendor is Microsoft Windows and \arara\ is being executed inside a Cygwin environment.

\begin{codebox}{Example}{teal}{\icnote}{white}
if (isCygwin()) { System.out.println('Running Cygwin.'); }
\end{codebox}

\item[\mdbox{R}{isWindows(Object yes, Object no)}{boolean}] This method checks if the underlying operating system vendor is Microsoft Windows. If the result holds true, the first parameter is returned. Otherwise, the second parameter is returned.

\begin{codebox}{Example}{teal}{\icnote}{white}
command = isWindows('del', 'rm');
\end{codebox}

\item[\mdbox{R}{isLinux(Object yes, Object no)}{boolean}] This method checks if the underlying operating system vendor is a Linux instance. If the result holds true, the first parameter is returned. Otherwise, the second parameter is returned.

\begin{codebox}{Example}{teal}{\icnote}{white}
command = isLinux('rm', 'del');
\end{codebox}

\item[\mdbox{R}{isMac(Object yes, Object no)}{boolean}] This method checks if the underlying operating system vendor is Apple Mac OS. If the result holds true, the first parameter is returned. Otherwise, the second parameter is returned.

\begin{codebox}{Example}{teal}{\icnote}{white}
command = isMac('ls', 'dir');
\end{codebox}

\item[\mdbox{R}{isUnix(Object yes, Object no)}{boolean}] This method checks if the underlying operating system vendor is any Unix variation. If the result holds true, the first parameter is returned. Otherwise, the second parameter is returned.

\begin{codebox}{Example}{teal}{\icnote}{white}
command = isUnix('tree', 'dir');
\end{codebox}

\item[\mdbox{R}{isAIX(Object yes, Object no)}{boolean}] This method checks if the underlying operating system vendor is IBM AIX. If the result holds true, the first parameter is returned. Otherwise, the second parameter is returned.

\begin{codebox}{Example}{teal}{\icnote}{white}
command = isAIX('pwd', 'ls');
\end{codebox}

\item[\mdbox{R}{isIrix(Object yes, Object no)}{boolean}] This method checks if the underlying operating system vendor is Silicon Graphics Irix. If the result holds true, the first parameter is returned. Otherwise, the second parameter is returned.

\begin{codebox}{Example}{teal}{\icnote}{white}
command = isIrix('ls', 'pwd');
\end{codebox}

\item[\mdbox{R}{isOS2(Object yes, Object no)}{boolean}] This method checks if the underlying operating system vendor is IBM OS/2 Warp. If the result holds true, the first parameter is returned. Otherwise, the second parameter is returned.

\begin{codebox}{Example}{teal}{\icnote}{white}
command = isOS2('ls', 'cd');
\end{codebox}

\item[\mdbox{R}{isSolaris(Object yes, Object no)}{boolean}] This method checks if the underlying operating system vendor is Oracle Solaris. If the result holds true, the first parameter is returned. Otherwise, the second parameter is returned.

\begin{codebox}{Example}{teal}{\icnote}{white}
command = isSolaris('ls', 'cat');
\end{codebox}

\item[\mdbox{R}{isCygwin(Object yes, Object no)}{boolean}] This method checks if the underlying operating system vendor is Microsoft Windows and if \arara\ is being executed inside a Cygwin environment. If the result holds true, the first parameter is returned. Otherwise, the second parameter is returned.

\begin{codebox}{Example}{teal}{\icnote}{white}
command = isCygwin('ls', 'dir');
\end{codebox}
\end{description}

The methods presented in the section provide useful information to help users write cross-platform rules and thus enhance the automation experience based on specific features of the underlying operating system.

\section{Type checking}
\label{sec:typechecking}

In certain scenarios, a plain string representation of directive parameters might be inadequate or insufficient given the rule requirements. To this end, this section introduces methods related to type checking as a means to provide support and verification for common data types.

\begin{description}
\item[\mdbox{R}{isString(Object object)}{boolean}] This method returns a boolean value according to whether the provided \rbox{Object} object is a string or any extended type.

\begin{codebox}{Example}{teal}{\icnote}{white}
result = isString('foo');
\end{codebox}

\item[\mdbox{R}{isList(Object object)}{boolean}] This method returns a boolean value according to whether the provided \rbox{Object} object is a list or any extended type.

\begin{codebox}{Example}{teal}{\icnote}{white}
result = isList([ 1, 2, 3 ]);
\end{codebox}

\item[\mdbox{R}{isMap(Object object)}{boolean}] This method returns a boolean value according to whether the provided \rbox{Object} object is a map or any extended type.

\begin{codebox}{Example}{teal}{\icnote}{white}
result = isMap([ 'Paulo' : 'Palmeiras', 'Carla' : 'Inter' ]);
\end{codebox}

\item[\mdbox{R}{isBoolean(Object object)}{boolean}] This method returns a boolean value according to whether the provided \rbox{Object} object is a boolean or any extended type.

\begin{codebox}{Example}{teal}{\icnote}{white}
result = isBoolean(false);
\end{codebox}

\item[\mdbox{R}{checkClass(Class clazz, Object object)}{boolean}] This method returns a boolean value according to whether the provided \rbox{Object} object is an instance or a subtype of the provided \rbox{Class} class. It is interesting to note that all methods presented in this section internally rely on \mtbox{checkClass} for type checking.

\begin{codebox}{Example}{teal}{\icnote}{white}
result = checkClass(List.class, [ 'a', 'b' ]);
\end{codebox}
\end{description}

The methods presented in this section cover the most common types used in directive parameters and should suffice for expressing the rule requirements. If a general approach is needed, please refer to the \mtbox{checkClass} method for checking virtually any type available in the Java environment.

\section{Classes and objects}
\label{sec:classesandobjects}

From version 4.0 on, \arara\ can be extended at runtime with code from JVM languages, such as Groovy, Scala, Clojure and Kotlin. The tool can load classes from \rbox{class} and \rbox{jar} files and even instantiate them. This section introduces methods related to classloading and object instantiation.

\begin{messagebox}{Ordered pairs}{araracolour}{\icok}{white}
According to the \href{https://en.wikipedia.org/wiki/Ordered_pair}{Wikipedia entry}, in mathematics, an \emph{ordered pair} $(a, b)$ is a pair of objects. The order in which the objects appear in the pair is significant: the ordered pair $(a, b)$ is different from the ordered pair $(b, a)$ unless $a = b$. In the ordered pair $(a, b)$, the object $a$ is called the \emph{first} entry, and the object $b$ the \emph{second} entry of the pair. \arara\ relies on such concept with the helper \rbox{Pair<A, B>} class, in which \rbox{A} and \rbox{B} denote the component classes, i.e, the types associated to the pair elements. In order to access the pair entries, the class provides two methods:

\begin{description}
\item[\mtbox{first()}\hfill\rrbox{A}] This method, as the name implies, returns the first entry of the ordered pair, as a \rbox{A} object. It is important to observe that, from the MVEL context, as the method constitutes a property accessor (namely, a getter), the parentheses can be safely omitted.

\item[\mtbox{second()}\hfill\rrbox{B}] This method, as the name implies, returns the second entry of the ordered pair, as a \rbox{B} object. It is important to observe that, from the MVEL context, as the method constitutes a property accessor (namely, a getter), the parentheses can be safely omitted.
\end{description}

Keep in mind that the entries in the \rbox{Pair} class, once defined, cannot be modified to other values. The initial values are set during instantiation and, therefore, only entry getters are available to the user during the object life cycle. 
\end{messagebox}

\begin{messagebox}{Status for classloading and instantiation}{araracolour}{\icok}{white}
The classloading and instantiation methods provided by \arara\ typically return a pair composed of an integer value and a class or object reference. This integer value acts as a status of the underlying operation itself and might indicate potential issues. The possible values are:

\vspace{1em}

{\centering
\def\arraystretch{1.5}
\begin{tabular}{lllll}
\rbox[araracolour]{\hphantom{x}0\hphantom{x}} & Successful execution & \hspace{1.5em} &
\rbox[araracolour]{\hphantom{x}3\hphantom{x}} & Class was not found \\
\rbox[araracolour]{\hphantom{x}1\hphantom{x}} & File does not exist & &
\rbox[araracolour]{\hphantom{x}4\hphantom{x}} & Access policy violation \\
\rbox[araracolour]{\hphantom{x}2\hphantom{x}} & File URL is incorrect & &
\rbox[araracolour]{\hphantom{x}5\hphantom{x}} & Instantiation exception
\end{tabular}\par}

\vspace{1.4em}

Please make sure to \emph{always} check the returned integer status when using classloading and instantiation methods in directive and rule contexts. This feature is quite powerful yet tricky and subtle! 
\end{messagebox}

\begin{description}
\item[\mddbox{C}{R}{loadClass(File file, String name)}{Pair<Integer, Class>}] This method loads a class based on the canonical name from the provided \rbox{File} reference and returns an ordered pair containing the status and the class reference itself. The file must contain the Java bytecode, either directly accessible from a \rbox{class} file or packaged inside a \rbox{jar} file. If an exception is raised, this method returns the \rbox{Object} class reference as second entry of the pair.

\begin{codebox}{Example}{teal}{\icnote}{white}
result = loadClass(toFile('mymath.jar'),
         'com.github.cereda.mymath.Arithmetic');
\end{codebox}

\item[\mddbox{C}{R}{loadClass(String reference, String name)}{Pair<Integer, Class>}] This method loads a class based on the canonical name from the provided \rbox{String} reference and returns an ordered pair containing the status and the class reference itself. The file must contain the Java bytecode, either directly accessible from a \rbox{class} file or packaged inside a \rbox{jar} file.  If an exception is raised, this method returns the \rbox{Object} class reference as second entry of the pair.

\begin{codebox}{Example}{teal}{\icnote}{white}
result = loadClass('mymath.jar',
         'com.github.cereda.mymath.Arithmetic');
\end{codebox}

\item[\mddbox{C}{R}{loadObject(File file, String name)}{Pair<Integer, Object>}] This method loads a class based on the canonical name from the provided \rbox{File} reference and returns an ordered pair containing the status and a proper corresponding object instantation. The file must contain the Java bytecode, either directly accessible from a \rbox{class} file or packaged inside a \rbox{jar} file. If an exception is raised, this method returns a \rbox{Object} object as second entry of the pair.

\begin{codebox}{Example}{teal}{\icnote}{white}
result = loadObject(toFile('mymath.jar'),
         'com.github.cereda.mymath.Trigonometric');
\end{codebox}

\item[\mddbox{C}{R}{loadObject(String reference, String name)}{Pair<Integer, Object>}] This method loads a class based on the canonical name from the provided \rbox{String} reference and returns an ordered pair containing the status and a proper corresponding object instantation. The file must contain the Java bytecode, either directly accessible from a \rbox{class} file or packaged inside a \rbox{jar} file. If an exception is raised, this method returns a \rbox{Object} object as second entry of the pair.

\begin{codebox}{Example}{teal}{\icnote}{white}
result = loadObject('mymath.jar',
         'com.github.cereda.mymath.Trigonometric');
\end{codebox}
\end{description}

This section presented classloading and instantion methods which may significantly enhance the expressiveness of rules and directives. However, make sure to use such feature with great care and attention.

\section{Dialog boxes}
\label{sec:dialogboxes}

% TODO fix reference
A \emph{dialog box} is a graphical control element, typically a small window, that communicates information to the user and prompts them for a response. This section introduces UI methods related to such interactions.

\begin{messagebox}{UI elements}{araracolour}{\icok}{white}
The graphical elements are provided by the Swing toolkit from the Java runtime environment. Note that the default look and feel class reference can be modified through a key in the configuration file, as seen in Section~\ref{foo}, on page~\pageref{foo}. It is important to observe that the methods presented in this section require a graphical interface. If \arara\ is being executed in a headless environment (i.e, an environment with no graphical display available), an exception will be thrown when trying to use such UI methods in either directive or rule contexts.
\end{messagebox}

Each dialog box provided by the UI methods of \arara\ requires the specification of an associated icon. An \emph{icon} is a pictogram displayed on a computer screen in order to help the user quickly identify the message by conveying its meaning through a visual resemblance to a physical object. Our tool features five icons, illustrated as follows, to be used with dialog boxes. Observe that each icon is associated with a unique integer value which is provided later on to the actual method call. Also, it is worth mentioning that the visual appearance of such icons is based on the underlying Java virtual machine and the current look and feel, so your milleage might vary.

\vspace{1em}

{\centering
\begin{tabularx}{0.8\textwidth}{YYYYY}
\uierror{cyan} &
\uiinfo{cyan} &
\uiwarning{cyan} &
\uiquestion{cyan} &
\uiplain{cyan} \\
{\footnotesize\emph{error}} &
{\footnotesize\emph{information}} &
{\footnotesize\emph{attention}} &
{\footnotesize\emph{question}} &
{\footnotesize\emph{plain}} \\
\rbox[cyan]{\hphantom{ww}1\hphantom{ww}} &
\rbox[cyan]{\hphantom{ww}2\hphantom{ww}} &
\rbox[cyan]{\hphantom{ww}3\hphantom{ww}} &
\rbox[cyan]{\hphantom{ww}4\hphantom{ww}} &
\rbox[cyan]{\hphantom{ww}5\hphantom{ww}}
\end{tabularx}\par}

\vspace{1.4em}

As good practice, make sure to provide descriptive messages to be placed in dialog boxes in order to ease and enhance the user experience. It is also highly advisable to always provide an associated icon, so avoid the plain option whenever possible.

\begin{messagebox}{Message text width}{araracolour}{\icok}{white}
\arara\ sets the default message text width to 250 pixels. Feel free to override this value according to your needs. Please refer to the appropriate method signatures for specifying a new width.
\end{messagebox}

The UI method signatures are followed by a visual representation of the provided dialog box. For the sake of simplicity, each parameter index refers to the associated number in the figure.

\begin{description}
\item[\mdbox{R}{showMessage(int width, int icon, String title, String text)}{void}]\uimethod{messagebox1}

This method shows a message box according to the provided parameters. The dialog box is disposed when the user either presses the confirmation button or closes the window. It is important to observe that \arara\ temporarily interrupts the execution and waits for the dialog box disposal. Also note that the total time consider the idle period as well.

\begin{codebox}{Example}{teal}{\icnote}{white}
showMessage(250, 2, 'My title', 'My message');
\end{codebox}

\item[\mdbox{R}{showMessage(int icon, String title, String text)}{void}]\uimethod{messagebox2}

This method shows a message box according to the provided parameters. The dialog box is disposed when the user either presses the confirmation button or closes the window. It is important to observe that \arara\ temporarily interrupts the execution and waits for the dialog box disposal. Also note that the total time consider the idle period as well.

\begin{codebox}{Example}{teal}{\icnote}{white}
showMessage(2, 'My title', 'My message');
\end{codebox}

\item[\mddbox{C}{R}{\parbox{0.62\textwidth}{showOptions(int width, int icon, String title,\\\hspace*{1em} String text, Object... options)}}{int}]\uimethod{optionbox1}

This method shows a message box according to the provided parameters, including options represented as an array of \rbox{Object} objects. This array is portrayed in the dialog box as a list of buttons. The dialog box is disposed when the user either presses one of the buttons or closes the window. The method returns the natural index of the selected button, starting from \rbox{1}. If no button is pressed (e.g, the window is closed), \rbox{0} is returned. Note that the object array is denoted by a  comma-separated sequence of elements in the actual method call, resulting in a variable number of parameters. It is important to observe that \arara\ temporarily interrupts the execution and waits for the dialog box disposal. Also note that the total time consider the idle period as well.

\begin{codebox}{Example}{teal}{\icnote}{white}
% arara: pdflatex if showOptions(250, 4, 'Important!',
% arara: --> 'Do you like ice cream?', 'Yes!', 'No!') == 1
\end{codebox}

\begin{messagebox}{Button orientation}{attentioncolour}{\icattention}{black}
Keep in mind that your window manager might render the button orientation differently than the original arrangement specified in your array of objects. For instance, I had a window manager that rendered the buttons in the reverse order. However, note that the visual appearance should not interfere with the programming logic! The indices shall remain the same, pristine as ever, regardless of the actual rendering. Trust your code, not your eyes.
\end{messagebox}

\item[\mddbox{C}{R}{\parbox{0.49\textwidth}{showOptions(int icon, String title,\\\hspace*{1em} String text, Object... options)}}{int}]\uimethod{optionbox2}

This method shows a message box according to the provided parameters, including options represented as an array of \rbox{Object} objects. This array is portrayed in the dialog box as a list of buttons. The dialog box is disposed when the user either presses one of the buttons or closes the window. The method returns the natural index of the selected button, starting from \rbox{1}. If no button is pressed (e.g, the window is closed), \rbox{0} is returned. Note that the object array is denoted by a  comma-separated sequence of elements in the actual method call, resulting in a variable number of parameters. It is important to observe that \arara\ temporarily interrupts the execution and waits for the dialog box disposal. Also note that the total time consider the idle period as well.

\begin{codebox}{Example}{teal}{\icnote}{white}
% arara: pdflatex if showOptions(4, 'Important!',
% arara: --> 'Do you like ice cream?', 'Yes!', 'No!') == 1
\end{codebox}

\item[\mddbox{C}{R}{\parbox{0.62\textwidth}{showDropdown(int width, int icon, String title,\\\hspace*{1em} String text, Object... options)}}{int}]\uimethod{dropdown1}

This method shows a dialog box according to the provided parameters, including options represented as an array of \rbox{Object} objects. This array is portrayed in the dialog box as a dropdown list. The first element from the array is automatically selected. The dialog box is disposed when the user either presses one of the buttons or closes the window. The method returns the natural index of the selected item, starting from \rbox{1}. If the user cancels the dialog or closes the window, \rbox{0} is returned.  Note that the object array is denoted by a comma-separated sequence of elements in the actual method call, resulting in a variable number of parameters. It is important to observe that \arara\ temporarily interrupts the execution and waits for the dialog box disposal. Also note that the total time consider the idle period as well.

\begin{codebox}{Example}{teal}{\icnote}{white}
% arara: pdflatex if showDropdown(250, 4, 'Important!',
% arara: --> 'Who deserves the tick?', 'David Carlisle',
% arara: --> 'Enrico Gregorio', 'Joseph Wright',
% arara: --> 'Heiko Oberdiek') == 2
\end{codebox}

\begin{messagebox}{Combo boxes and dropdown lists}{attentioncolour}{\icattention}{black}
According to the \href{https://en.wikipedia.org/wiki/Combo_box}{Wikipedia entry}, a \emph{combo box} is a combination of a dropdown list or list box and a single line editable textbox, allowing the user to either type a value directly or select a value from the list. The term is sometimes used to mean a dropdown list, but in Java, the term is definitely not a synonym! A dropdown list is sometimes clarified with terms such as non-editable combo box to distinguish it from the original definition of a combo box.
\end{messagebox}

\item[\mddbox{C}{R}{\parbox{0.49\textwidth}{showDropdown(int icon, String title,\\\hspace*{1em} String text, Object... options)}}{int}]\uimethod{dropdown2}

This method shows a dialog box according to the provided parameters, including options represented as an array of \rbox{Object} objects. This array is portrayed in the dialog box as a dropdown list. The first element from the array is automatically selected. The dialog box is disposed when the user either presses one of the buttons or closes the window. The method returns the natural index of the selected item, starting from \rbox{1}. If the user cancels the dialog or closes the window, \rbox{0} is returned.  Note that the object array is denoted by a comma-separated sequence of elements in the actual method call, resulting in a variable number of parameters. It is important to observe that \arara\ temporarily interrupts the execution and waits for the dialog box disposal. Also note that the total time consider the idle period as well.

\begin{codebox}{Example}{teal}{\icnote}{white}
% arara: pdflatex if showDropdown(4, 'Important!',
% arara: --> 'Who deserves the tick?', 'David Carlisle',
% arara: --> 'Enrico Gregorio', 'Joseph Wright',
% arara: --> 'Heiko Oberdiek') == 2
\end{codebox}

\begin{messagebox}{Swing toolkit}{araracolour}{\icok}{white}
According to the \href{https://en.wikipedia.org/wiki/Swing_(Java)}{Wikipedia entry}, the Swing toolkit was developed to provide a more sophisticated set of GUI components than the earlier  AWT widget system. Swing provides a look and feel that emulates the look and feel of several platforms, and also supports a pluggable look and feel that allows applications to have a look and feel unrelated to the underlying platform. It has more powerful and flexible components than AWT. In addition to familiar components such as buttons, check boxes and labels, Swing provides several advanced components. such as scroll panes, trees, tables, and lists.
\end{messagebox}

\item[\mdbox{R}{showInput(int width, int icon, String title, String text)}{String}]\uimethod{inputbox1}

This method shows an input dialog box according to the provided parameters. The dialog box is disposed when the user either presses one of the buttons or closes the window. The method returns the content of the input text field, as a trimmed \rbox{String} object. If the user cancels the dialog or closes the window, an empty string is returned. It is important to observe that \arara\ temporarily interrupts the execution and waits for the dialog box disposal. Also note that the total time consider the idle period as well.

\begin{codebox}{Example}{teal}{\icnote}{white}
% arara: pdflatex if showInput(250, 4, 'Important!',
% arara: --> 'Who wrote arara?') == 'Paulo'
\end{codebox}

\item[\mdbox{R}{showInput(int icon, String title, String text)}{String}]\uimethod{inputbox2}

This method shows an input dialog box according to the provided parameters. The dialog box is disposed when the user either presses one of the buttons or closes the window. The method returns the content of the input text field, as a trimmed \rbox{String} object. If the user cancels the dialog or closes the window, an empty string is returned. It is important to observe that \arara\ temporarily interrupts the execution and waits for the dialog box disposal. Also note that the total time consider the idle period as well.

\begin{codebox}{Example}{teal}{\icnote}{white}
% arara: pdflatex if showInput(4, 'Important!',
% arara: --> 'Who wrote arara?') == 'Paulo'
\end{codebox}
\end{description}

The UI methods presented in this section can be used for writing \TeX\ tutorials and assisted compilation workflows based on user interactions, including visual input and feeback through dialog boxes.

\section{Commands and triggers}
\label{sec:commandsandtriggers}

From version 4.0 on, \arara\ features the \rbox{Command} object, a new approach for handling system commands based on a high level structure with explicit argument parsing. Similarly, there is also the mystical \rbox{Trigger} object that constitutes a very special command that changes the inner workings of our tool at runtime. This section introduces methods for generating such objects.

\begin{messagebox}{The anatomy of a command}{araracolour}{\icok}{white}
\setlength{\parskip}{1em}
From the user perspective, a \rbox{Command} object is simply a good old list of \rbox{Object} objects, in which the list head (i.e, the first element) is the underlying system command, and the list tail (i.e, the remaining elements), if any, contains the associated command line arguments. For instance:

{\centering
\setlength\tabcolsep{0.2em}
\begin{tabular}{cccc}
{\footnotesize\em head} &
\multicolumn{3}{c}{\footnotesize\em tail (associated command line arguments)} \\
\rbox[cyan]{pdflatex} &
\rbox[araracolour]{{-}-shell-escape} &
\rbox[araracolour]{{-}-synctex=1} &
\rbox[araracolour]{thesis.tex}
\end{tabular}\par}

\vspace{0.4em}

From the previous example, it is important to observe that a potential file name quoting is not necessary. The underlying system command execution library handles the provided arguments accordingly. 

Behind the scenes, however, \arara\ employs a different workflow when constructing a \rbox{Command} object. The tool sets the working directory path for the current command to \abox[araracolour]{USER\_DIR} which is based on the current execution. The working directory path can be explicitly set through specific method calls, described later on in this section.

The list of objects is then completely flattened and all elements are mapped to their string representations through corresponding \mtbox{toString} calls. Finally, the proper \rbox{Command} object is constructed. Keep in mind that, although a command takes a list (or even an array) of objects, which can be of any type, the internal representation is \emph{always} a list of strings.
\end{messagebox}

A list of objects might contain nested lists, i.e, a list within another list. As previously mentioned, \arara\ employs \emph{list flattening} when handling a list of objects during a \rbox{Command} object instantation. As a means to illustrate this handy feature, consider the following list of integers:

\begin{codebox}{A list with nested lists}{teal}{\icnote}{white}
[ 1, 2, [ 3, 4 ], 5, [ [ 6, 7 ], 8 ], 9, [ [ 10 ] ]
\end{codebox}

Note that the previous list of integers contains nested lists. When applying list flattening, \arara\ recursively adds the elements of nested lists to the original list and then removes the nested occurrences. Please refer to the source code for implementation details. The new flattened list is presented as follows.

\begin{codebox}{A flattened list}{teal}{\icnote}{white}
[ 1, 2, 3, 4, 5, 6, 7, 8, 9, 10 ]
\end{codebox}

List flattening and string mapping confer expressiveness and flexibility to the \rbox{Command} object construction, as users can virtually use any data type to describe the underlying rule logic and yet obtain a consistent representation.

\begin{messagebox}{The anatomy of a trigger}{araracolour}{\icok}{white}
\setlength{\parskip}{1em}
A \rbox{Trigger} object constitutes a special command that changes the internal workings of \arara\ at runtime. It is a highly experimental feature. A trigger is basically a function call defined in terms of a \rbox{String} reference representing the actual function name followed by an optional list of \rbox{Object} arguments.

This description is, in some aspects, very much alike a typical \rbox{Command} construction. However, a trigger is less forgiving on data types and does not apply transformations on the provided list of arguments. Therefore, the argument types \emph{must match} the trigger signature.

So far, the tool provides only one trigger, seen in action in one of the official rules, under the name \rbox{halt} and with no parameters. This particular trigger halts the current interpretation workflow, such that subsequent directives are ignored. We have not worked much on triggers yet, and the concept is mentioned here for documentation purposes only.


\end{messagebox}

%\begin{}{Source file}{teal}{\icnote}{white}{}
%\end{ncodebox}

%\begin{codebox}{}{teal}{\icnote}{white}
%\end{codebox}

%\begin{messagebox}{}{araracolour}{\icok}{white}
%\end{messagebox}

% !TeX root = ../arara-manual.tex
\chapter{The official rule pack}
\label{chap:theofficialrulepack}

\arara\ ships with a pack of default rules, placed inside a special subdirectory named \abox[araracolour]{rules/} inside another special directory named \abox[araracolour]{ARARA\_HOME} (the place where our tool is installed). This chapter introduces the official rules, including proper listings and descriptions of associated parameters whenever applied. Note that such rules work out of the shelf, without any special installation, configuration or modification. An option marked by \rbox[araracolour]{S} after the corresponding identifier indicates a natural boolean switch.

\begin{messagebox}{Can my rule be distributed within the official pack?}{araracolour}{\icok}{white}
% TODO fix reference
As seen in Section~\ref{foo}, on page~\pageref{foo}, the default rule path can be extended to include a list of directories in which our tool should search for rules. However, if you believe your rule is comprehensive enough and deserves to be in the official pack, please contact us! We will be more  than happy to discuss the inclusion of your rule in forthcoming updates.
\end{messagebox}

\begin{description}
\item[\rulebox{animate}{Chris Hughes, Paulo Cereda}]
This rule creates an animated \rbox{gif} file from the corresponding base name of the \mtbox{currentFile} reference (i.e, the name without the associated extension) as a string concatenated with the \rbox{pdf} suffix, using the \rbox{convert} command line utility from the ImageMagick suite.

\begin{description}
\item[\rpbox{delay}{10}] This option regulates the number of ticks before the display of the next image sequence, acting as a pause between still frames.

\item[\rpbox{loop}{0}] This option regulates the number of repetions for the animation. When set to zero, the animation repeats itself an infinite number of times.

\item[\rpbox{density}{300}] This option specifies the horizontal and vertical canvas resolution while rendering vector formats into a proper raster image.

\item[\rpbox{program}{convert}] This option specifies the command utility path as a means to avoid potential clashes with underlying operating system commands.

\begin{messagebox}{Microsoft Windows woes}{attentioncolour}{\icattention}{black}
\setlength{\parskip}{1em}
According to the \href{http://www.imagemagick.org/Usage/windows/}{ImageMagick website}, the Windows installation routine adds the program directory to the system path, such that one can call command line tools directly from the command prompt, without providing a path name. However, \rbox{convert} is also the name of Windows system tool, located in the system directory, which converts file systems from one format to another.

The best solution to avoid possible future name conflicts, according to the ImageMagick team, is to call such command line tools by their full path in any script. Therefore, the \rbox{convert} rule provides the \abox{program} option for this specific scenario.
\end{messagebox}

\item[\abox{options}] This option, as the name indicates, takes a list of raw command line options and appends it to the actual system call. An error is thrown if any data structure other than a proper list is provided as value.
\end{description}

\begin{codebox}{Example}{teal}{\icnote}{white}
% TODO
\end{codebox}

\item[\rulebox{bib2gls}{Nicola Talbot, Paulo Cereda}] This rule executes the \rbox{bib2gls} command line application which extract glossary information stored in a \rbox{bib} file and converts it into glossary entry definitions in resource files.

\begin{description}
\item[\abox{dir}] This option sets the directory used for writing auxiliary files. Note that this option does not change the current working directory.

\item[\abox{trans}] This option sets the name of the transcript file. By default, the name is the same as the \rbox{aux} file but with a \rbox{glg} extension.

\item[\abox{locale}] This option specifies the preferred language resource file. Please keep in mind that the provided value must be a valid IETF language tag.

\item[\rpsbox{group}] This option sets whether \rbox{bib2gls} will try to determine the letter group for each entry and add it to a new field called \rbox{group} when sorting. Be mindful that some \rbox{sort} options ignore this setting.

\item[\rpsbox{interpret}] This option sets whether the interpreter mode of \rbox{bib2gls} is enabled. If the interpreter is off, the transcript file will not be parsed.

\item[\rpsbox{breakspace}] This option sets whether the interpreter will treat a tilde character as a normal space. The default behaviour treats it as nonbreakable.

\item[\rpsbox{trimfields}] This option sets whether \rbox{bib2gls} will trim leading and trailing spaces from field values. The default behaviour does not trim spaces.

\item[\rpsbox{recordcount}] This option sets whether the record counting will be enabled. If activated, \rbox{bib2gls} will add record count fields to entries.

\item[\rpsbox{recordcountunit}] This option sets whether \rbox{bib2gls} will add unit record count fields to entries. These fields can then be used with special commands.

\item[\rpsbox{cite}] This option sets whether \rbox{bib2gls} will treat citation instances found in the \rbox{aux} file as though it was actually an ignored record.

\item[\rpsbox{verbose}] This option sets whether \rbox{bib2gls} will be executed in verbose mode. When enabled, the application will write extra information to the terminal and transcript file. The default behaviour runs in silent mode.

\item[\rpsbox{merge}] This option sets whether the program will merge \rbox{wrglossary} counter records. If disabled, one may end up with duplicate page numbers in the list of entry locations, but linking to different parts of the page.

\item[\rpsbox{uniscript}] This option sets whether text superscript and subscript will use the corresponding Unicode characters if available.

\item[\abox{packages}] This option instructs the interpreter to assume the packages from the provided list have been used by the document.

\item[\abox{ignore}] This option instructs \rbox{bib2gls} to skip the check for any package from the provided list when parsing the corresponding log file.

\item[\abox{custom}] This option instructs the interpreter to parse the package files from the provided list. The package files need to be quite simple.

\item[\abox{mapformats}] This option takes a list and sets up the rule of precedence for partial location matches. Each element from the provided list must be another list of exactly two entries representing a conflict resolution.

\item[\abox{options}] This option, as the name indicates, takes a list of raw command line options and appends it to the actual system call. An error is thrown if any data structure other than a proper list is provided as value.
\end{description}

\begin{codebox}{Example}{teal}{\icnote}{white}
% TODO
\end{codebox}

\item[\rulebox{biber}{Marco Daniel, Paulo Cereda}] This rule runs \rbox{biber}, the backend bibliography processor for \rbox{biblatex}, on the corresponding base name of the \mtbox{currentFile()} reference (i.e, the name without the associated extension) as a string.

\begin{description}
\item[\abox{options}] This option, as the name indicates, takes a list of raw command line options and appends it to the actual system call. An error is thrown if any data structure other than a proper list is provided as value.
\end{description}

\begin{codebox}{Example}{teal}{\icnote}{white}
% TODO
\end{codebox}

\item[\rulebox{bibtex8}{Marco Daniel, Paulo Cereda}] This rule runs \rbox{bibtex8}, an enhanced, portable C version of \rbox{bibtex}, on the corresponding base name of the \mtbox{currentFile()} reference (i.e, the name without the associated extension) as a string. It is important to note that this tool can read a character set file containing encoding details.

\begin{description}
\item[\abox{options}] This option, as the name indicates, takes a list of raw command line options and appends it to the actual system call. An error is thrown if any data structure other than a proper list is provided as value.
\end{description}

\begin{codebox}{Example}{teal}{\icnote}{white}
% TODO
\end{codebox}

\item[\rulebox{bibtexu}{Marco Daniel, Paulo Cereda}] This rule runs the \rbox{bibtexu} program, an enhanced version of \rbox{bibtex} with Unicode support and language features, on the corresponding base name of the \mtbox{currentFile()} reference (i.e, the name without the associated extension) as a string.

\begin{description}
\item[\abox{options}] This option, as the name indicates, takes a list of raw command line options and appends it to the actual system call. An error is thrown if any data structure other than a proper list is provided as value.
\end{description}

\begin{codebox}{Example}{teal}{\icnote}{white}
% TODO
\end{codebox}

\item[\rulebox{bibtex}{Marco Daniel, Paulo Cereda}] This rule runs the \rbox{bibtex} program, a reference management software, on the corresponding base name of the \mtbox{currentFile()} reference (i.e, the name without the associated extension) as a string.

\begin{description}
\item[\abox{options}] This option, as the name indicates, takes a list of raw command line options and appends it to the actual system call. An error is thrown if any data structure other than a proper list is provided as value.
\end{description}

\begin{codebox}{Example}{teal}{\icnote}{white}
% TODO
\end{codebox}
\end{description}



%\begin{}{Source file}{teal}{\icnote}{white}{}
%\end{ncodebox}

%\begin{codebox}{}{teal}{\icnote}{white}
%\end{codebox}

%\begin{messagebox}{}{araracolour}{\icok}{white}
%\end{messagebox}

\part{Development and deployment}
\label{part:developmentanddeployment}

% !TeX root = ../arara-manual.tex
\chapter{Building from source}
\label{chap:buildingfromsource}

\arara\ is a Java application licensed under the \href{http://www.opensource.org/licenses/bsd-license.php}{New BSD License}, a verified GPL-compatible free software license, and the source code is available in the project repository at \href{https://github.com/cereda/arara}{GitHub}. This chapter provides detailed instructions on how to build our tool from source.

\section{Requirements}
\label{sec:requirements}

In order to build our tool from source, we need to ensure that our development environment has the minimum requirements for a proper compilation. Make sure the following items are contemplated:

\begin{itemize}[label={\cbyes{-2}}]
\item On account of our project being hosted at \href{https://github.com}{GitHub}, an online source code repository, we highly recommend the installation of \rbox{git}, a version control system for tracking changes in computer files and coordinating work on those files among multiple people. Alternatively, you can directly obtain the source code by requesting a \href{https://github.com/cereda/arara/archive/master.zip}{source code download} in the repository. In order to check if \rbox{git} is available in your operating system, run the following command in the terminal (version numbers might vary):

\begin{codebox}{Terminal}{teal}{\icnote}{white}
$ git --version
git version 2.17.1
\end{codebox}

Please refer to \href{https://git-scm.com/}{project website} in order to obtain specific installation instructions for your operating system. In general, most recent Unix systems have \rbox{git} installed out of the shelf.

\item Our tool is written in the Java programming language, so we need a proper Java Development Kit,  a collection of programming tools for the Java platform. Our source code is known to be compliant with several vendors, including Oracle, OpenJDK, and Azul Systems. In order to check if your operating system has the proper tools, run the following command in the terminal (version numbers might vary):

\begin{codebox}{Terminal}{teal}{\icnote}{white}
$ javac -version
javac 1.8.0_171
\end{codebox}

The previous command, as the name suggests, refers to the \rbox{javac} tool, which is the Java compiler itself. The most common Java Development Kit out there is from \href{http://www.oracle.com/technetwork/java/javase/downloads/index.html}{Oracle}. However, several Linux distributions (as well as some developers, yours truly included) favour the OpenJDK vendor, so your milleage may vary. Please refer to the corresponding website of the vendor of your choice in order to obtain specific installation instructions for your operating system.

\item As a means to provide a straightforward and simplified compilation workflow, \arara\ relies on Apache Maven, a software project management and comprehension tool. Based on the concept of a project object model, Maven can manage builds, reporting and documentation from a central piece of information. In order to check if \rbox{mvn}, the Maven binary, is available in your operating system, run the following command in the terminal (version numbers might vary):

\begin{codebox}{Terminal}{teal}{\icnote}{white}
$ mvn --version
Apache Maven 3.5.2 (Red Hat 3.5.2-5)
Maven home: /usr/share/maven
Java version: 1.8.0_171, vendor: Oracle Corporation
Java home: /usr/lib/jvm/java-1.8.0-openjdk-
    1.8.0.171-4.b10.fc28.x86_64/jre
Default locale: pt_BR, platform encoding: UTF-8
OS name: "linux", version: "4.16.16-300.fc28.x86_64",
    arch: "amd64", family: "unix"
\end{codebox}

Please refer to \href{https://maven.apache.org/}{project website} in order to obtain specific installation instructions for your operating system. In general, most recent Linux distributions have the Maven binary, as well the proper associated dependencies, available in their corresponding repositories.

\item For a proper repository cloning, as well as the first Maven build, an active Internet connection is required. In particular, Maven dynamically downloads Java libraries and plug-ins from one or more online repositories and stores them in a local cache. Be mindful that subsequent builds can occur offline, provided that the local Maven cache exists.
\end{itemize}

\arara\ can be easily built from source, provided that the aforementioned requirements are contemplated. The next section presents the compilation details, from repository cloning to a proper Java archive generation.

\begin{messagebox}{One tool to rule them all}{araracolour}{\icok}{white}
\setlength{\parskip}{1em}
For the brave, there is the \href{https://sdkman.io/}{Software Development Kit Manager}, an interesting tool for managing parallel versions of multiple software development kits on most Unix based systems. In particular, this tool provides out of the shelf support for several Java Development Kit vendors and versions, as well as most recent versions Apache Maven.

Personally, I prefer the packaged versions provided by my favourite Linux distribution (Fedora), but this tool is a very interesting alternative to set up a development environment with little to no effort.
\end{messagebox}

\section{Compiling the tool}
\label{sec:compilingthetool}

First and foremost, we need to clone the project repository into our development environment, so we can build our tool from source. Run the following command in the terminal:

\begin{codebox}{Terminal}{teal}{\icnote}{white}
$ git clone https://github.com/cereda/arara
\end{codebox}

Wait a couple of seconds (or minutes, depending on your Internet connection) while the previous command clones the project repository hosted at GitHub into a directory named \abox[araracolour]{arara/} within the working directory. Be mindful that this operation pulls down every version of every file for the history of the project. Fortunately, the version control system has the notion of a \emph{shallow clone}, which is a more succinctly meaningful way of describing a local repository with history truncated to a particular depth during the clone operation. If you want to get only the latest revision of everything in our repository, run the following command in the terminal:

\begin{codebox}{Terminal}{teal}{\icnote}{white}
$ git clone https://github.com/cereda/arara --depth 1
\end{codebox}

This operation is way faster than the previous one, for obvious reasons. Unix terminals typically start at \abox[araracolour]{USER\_HOME} as working directory, so the newly cloned \abox[araracolour]{arara/} directory is almost certain to be accessible from that level. Now, we need to navigate to a directory named \abox[araracolour]{application/} inside our project structure, where the source code and the corresponding build file are located. Run the following command in the terminal:

\begin{codebox}{Terminal}{teal}{\icnote}{white}
$ cd arara/application
\end{codebox}

%\begin{}{Source file}{teal}{\icnote}{white}{}
%\end{ncodebox}

%\begin{codebox}{}{teal}{\icnote}{white}
%\end{codebox}

%\begin{messagebox}{}{araracolour}{\icok}{white}
%\end{messagebox}

% !TeX root = ../arara-manual.tex
\chapter{Deploying the tool}
\label{chap:deployingthetool}

%\begin{}{Source file}{teal}{\icnote}{white}{}
%\end{ncodebox}

%\begin{codebox}{}{teal}{\icnote}{white}
%\end{codebox}

%\begin{messagebox}{}{araracolour}{\icok}{white}
%\end{messagebox}


\part{A primer on formats and scripting}
\label{part:primer}

% !TeX root = ../arara-manual.tex
\chapter{YAML}
\label{chap:yaml}

According to the \href{http://yaml.org/spec/1.2/spec.html}{specification}, YAML (a recursive acronym for \emph{YAML Ain't Markup Language}) is a human-friendly, cross language, Unicode-based data serialization language designed around the common native data type of programming languages. \arara\ uses this format in three circumstances:

\begin{enumerate}
\item\emph{Parametrized directives}, as the set of attribute/value pairs (namely, argument name and corresponding value) is represented by a map. This particular type of directive is formally introduced in Section~\ref{sec:directives}, on page~\pageref{sec:directives}.

\item\emph{Rules}, as their entire structure is represented by a set of specific keys and their corresponding values (a proper YAML document). A rule follows a very strict model, detailed in Section~\ref{sec:rule}, on page~\pageref{sec:rule}.

\item\emph{Configuration files}, as the general settings are represented by a set of specific keys and their corresponding values (a proper YAML document). Configuration files are covered in Chapter~\ref{chap:configurationfile}, on page~\pageref{chap:configurationfile}.
\end{enumerate}

This chapter only covers the relevant parts of the YAML format for a consistent use with \arara. For advanced topics, I highly recommend the complete format specification, available online.

\section{Collections}
\label{sec:yamlcollections}

According to the specification, YAML's block collections use indentation for scope and begin each entry on its own line. Block sequences indicate each entry with a dash and space. Mappings use a colon and space to mark each \emph{key: value} pair. Comments begin with an octothorpe \rbox{\#}. \arara\ relies solely on mappings and a few scalars to sequences at some point. Let us see an example of a sequence:

\begin{codebox}{A sequence of scalars in YAML}{teal}{\icnote}{white}
team:
- Paulo Cereda
- Marco Daniel
- Brent Longborough
- Nicola Talbot
- Ben Frank
\end{codebox}

It is quite straightforward: \abox{team} holds a sequence of four scalars. YAML also has flow styles, using explicit indicators rather than indentation to denote scope. The flow sequence is written as a comma-separated list within square brackets:

\begin{codebox}{A sequence of scalars in YAML}{teal}{\icnote}{white}
primes: [ 2, 3, 5, 7, 11 ]
\end{codebox}

Attribute maps are easily represented by nesting entries, respecting indentation. For instance, consider a map \abox{developer} containing two keys, \abox{name} and \abox{country}. The YAML representation is presented as follows:

\begin{codebox}{An attribute map in YAML}{teal}{\icnote}{white}
developer:
 name: Paulo
 country: Brazil
\end{codebox}

Similarly, the flow mapping uses curly braces. Observe that this is the form adopted by a parametrized directive (see syntax in Section~\ref{sec:directives}, on page~\pageref{sec:directives}):

\begin{codebox}{An attribute map in YAML (flow mapping)}{teal}{\icnote}{white}
developer: { name: Paulo, country: Brazil }
\end{codebox}

An attribute map can contain sequences as well. Consider the following code where \abox{developers} holds a list of two developers containing their names and countries:

\begin{codebox}{An attribute map with sequences in YAML}{teal}{\icnote}{white}
developers:
- name: Paulo
  country: Brazil
- name: Marco
  country: Germany
\end{codebox}

The previous code can be easily represented in flow style by using square and curly brackets to represent sequences and attribute maps.

\section{Scalars}
\label{sec:yamlscalars}

Scalar content can be written in block notation, using a literal style, indicated by a vertical bar, where \emph{all line breaks are significant}. Alternatively, they can be written with the folded style, denoted by a greater-than sign, where \emph{each line break is folded to a space} unless it ends an empty or a more-indented line. It is mportant to note that \arara\ intensively uses both styles (as seen in Section~\ref{sec:rule}, on page~\pageref{sec:rule}). Let us see an example:

\begin{codebox}{Scalar content in literal and folded styles}{teal}{\icnote}{white}
logo: |
  This is the arara logo
  in its ASCII glory! 
    __ _ _ __ __ _ _ __ __ _ 
   / _` | '__/ _` | '__/ _` |
  | (_| | | | (_| | | | (_| |
   \__,_|_|  \__,_|_|  \__,_|
slogan: >
  The cool TeX
  automation tool
\end{codebox}

As seen in the previous code, \abox{logo} holds the ASCII logo of our tool, respecting line breaks. Similarly, observe that the \abox{slogan} key holds the text with line breaks replaced by spaces (in the same fashion \TeX\ does with consecutive, non-empty lines).

\begin{messagebox}{Block indentation indicator}{attentioncolour}{\icattention}{black}
\setlength{\parskip}{1em}
According to the YAML specification, the indentation level of a block scalar is typically detected from its first non-empty line. It is an error for any of the leading empty lines to contain more spaces than the first non-empty line, hence the ASCII logo could not be represented, as it starts with a space.

When detection would fail, YAML requires that the indentation level for the content be given using an explicit indentation indicator. This level is specified as the integer number of the additional indentation spaces used for the content, relative to its parent node. It would be the case if we want to represent our logo without the preceding text.
\end{messagebox}

YAML's flow scalars include the plain style and two quoted styles. The double-quoted style provides escape sequences. The single-quoted style is useful when escaping is not needed. All flow scalars can span multiple lines. Note that line breaks are always folded. Since \arara\ uses MVEL as its underlying scripting language (Chapter~\ref{chap:mvel}, on page~\pageref{chap:mvel}), it might be advisable to quote scalars when starting with forbidden symbols in YAML.

\section{Tags}
\label{sec:yamltags}

According to the specification, in YAML, untagged nodes are given a type depending on the application. The examples covered in this primer use the \rbox{seq}, \rbox{map} and \rbox{str} types from the fail safe schema. Explicit typing is denoted with a tag using the exclamation point symbol. Global tags are usually uniform resource identifiers and may be specified in a tag shorthand notation using a handle. Application-specific local tags may also be used. For \arara, there is a special schema used for both rules and configuration files, so in those cases, make sure to add \abox{!config} as global tag:

\begin{codebox}{Global tag for rules and configuration files}{teal}{\icnote}{white}
!config
\end{codebox}

In particular, rules and configuration files of \arara\ are properly covered in Section~\ref{sec:rule} and Chapter~\ref{chap:configurationfile}, on pages~\pageref{sec:rule} and~\pageref{chap:configurationfile}, respectively. For now, it suffices to say that the \abox{!config} global tag is necessary to provide the correct mapping of values inside our tool.

\section{Further reading}
\label{sec:yamlfurtherreading}

This chapter does not cover all features of the YAML format, so further reading is advisable. I highly recommend the \href{http://yaml.org/spec/1.2/spec.html}{official YAML specification}, currently covering the third version of the format.

% !TeX root = ../arara-manual.tex
\chapter{MVEL}
\label{chap:mvel}

According to the \href{https://en.wikipedia.org/wiki/MVEL}{Wikipedia entry}, the MVFLEX Expression Language (hereafter referred as MVEL) is a hybrid, dynamic, statically typed, embeddable expression language and runtime for the Java platform. Originally started as a utility language for an application framework, the project is now developed completely independently. \arara\ relies on such scripting language in two circumstances:

\begin{enumerate}
\item\emph{Rules}, as nominal attributes gathered from directives are used to build complex command invocations and additional computations. A rule follows a very strict model, detailed in Section~\ref{sec:rule}, on page~\pageref{sec:rule}.

\item\emph{Conditionals}, as logical expressions must be evaluated in order to decide whether and how a directive should be interpreted. Conditionals are detailed in Section~\ref{sec:directives}, on page~\pageref{sec:directives}.
\end{enumerate}

This chapter only covers the relevant parts of the MVEL language for a consistent use with \arara. For advanced topics, I highly recommend the official language guide for MVEL 2.0, available online.

\section{Basic usage}
\label{sec:mvelbasicusage}

The following primer is provided by the \href{https://mvel.documentnode.com/}{official language guide}, almost verbatim, with a few modifications to make it more adherent to our needs with \arara. Consider the following expression:

\begin{codebox}{Simple property expression}{teal}{\icnote}{white}
user.name
\end{codebox}

In this expression, we have a single identifier \rbox{user.name}, which by itself is a property expression, in that the only purpose of such expression is to extract a property out of a variable or context object, namely \rbox{user}. Property expressions are widely used by \arara, as directive parameters are converted to a map inside the corresponding rule scope. For instance, a parameter \rbox{foo} in a directive will be mapped as \rbox{parameters.foo} inside a rule during interpretation. This topic is detailed in Section~\ref{sec:directives}, on page~\pageref{sec:directives}. The scripting language can also be used for evaluating a boolean expression:

\begin{codebox}{Boolean expression evaluation}{teal}{\icnote}{white}
user.name == 'John Doe'
\end{codebox}

This expression yields a boolean result, either \rbox{true} or \rbox{false} based on a comparation operation. Like a typical programming language, MVEL supports the full gambit of operator precedence rules, including the ability to use bracketing to control execution order:

\begin{codebox}{Execution order control through bracketing}{teal}{\icnote}{white}
(user.name == 'John Doe') && ((x * 2) - 1) > 20
\end{codebox}

You may write scripts with an arbitrary number of statements using semicolon to denote the termination of a statement. This is required in all cases except in cases where there is only one statement, or for the last statement in a script:

\begin{codebox}{Multiple statements}{teal}{\icnote}{white}
statement1; statement2; statement3
\end{codebox}

It is important to observe that MVEL expressions use a \emph{last value out} principle. This means, that although MVEL supports the \rbox{return} keyword, it can be safely omitted. For example:

\begin{codebox}{Automatic return}{teal}{\icnote}{white}
foo = 10;
bar = (foo = foo * 2) + 10;
foo;
\end{codebox}

In this particular example, the expression automatically returns the value of \rbox{foo} as it is the last value of the expression. It is functionally identical to:

\begin{codebox}{Explicit return}{teal}{\icnote}{white}
foo = 10;
bar = (foo = foo * 2) + 10;
return foo;
\end{codebox}

Personally, I like to explicitly add a \rbox{return} statement, as it provides a visual indication of the expression exit point. All rules released with \arara\ favour this writing style. However, feel free to choose any writing style you want, as long as the resulting code is consistent.

The type coercion system of MVEL is applied in cases where two incomparable types are presented by attempting to coerce the right value to that of the type of the left value, and then vice-versa. For example:

\begin{codebox}{Type coercion}{teal}{\icnote}{white}
"123" == 123;
\end{codebox}

Surprisingly, the evaluation of such expression holds \rbox{true} in MVEL because the underlying type coercion system will coerce the untyped number \rbox{123} to a string \rbox{123} in order to perform the comparison.

\section{Inline lists, maps and arrays}
\label{sec:mvelinlinelistsmapsandarrays}

According to the documentation, MVEL allows you to express lists, maps and arrays using simple elegant syntax. Lists are expressed in the following format:

\begin{codebox}{Creating a list}{teal}{\icnote}{white}
[ "Jim", "Bob", "Smith" ]
\end{codebox}

Note that lists are denoted by comma-separated values delimited by square brackets. Similarly, maps (sets of key/value attributes) are expressed in the following format: 

\begin{codebox}{Creating a map}{teal}{\icnote}{white}
[ "Foo" : "Bar", "Bar" : "Foo" ]
\end{codebox}

Note that attributes are composed by a key, a colon and the corresponding value. A map is denoted by comma-separated attributes delimited  by square brackets. Finally, arrays are expressed in the following format:

\begin{codebox}{Creating an array}{teal}{\icnote}{white}
{ "Jim", "Bob", "Smith" }
\end{codebox}

One important aspect about inline arrays is their special ability to be coerced to other array types. When you declare an inline array, it is untyped at first and later coerced to the type needed in context. For instance, consider the following code, in which \rbox{sum} takes an array of integers:

\begin{codebox}{Array coercion}{teal}{\icnote}{white}
math.sum({ 1, 2, 3, 4 });
\end{codebox}

In this case, the scripting language will see that the target method accepts an integer array and automatically type the provided untyped array as such. This is an important feature exploited by \arara\ when calling methods within the rule or conditional scope.

\section{Property navigation}
\label{sec:propertynavigation}

MVEL provides a single, unified syntax for accessing properties, static fields, maps and other structures. Lists are accessed the same as arrays. For example, these two constructs are equivalent (MVEL and Java access styles for lists and arrays, respectively):

\begin{codebox}{MVEL access style for lists and arrays}{teal}{\icnote}{white}
user[5]
\end{codebox}

\begin{codebox}{Java access style for lists and arrays}{teal}{\icnote}{white}
user.get(5)
\end{codebox}

Observe that MVEL accepts plain Java methods as well. Maps are accessed in the same way as arrays except any object can be passed as the index value. For example, these two constructs are equivalent (MVEL and Java access styles for maps, respectively):

\begin{codebox}{MVEL access style for maps}{teal}{\icnote}{white}
user["foobar"]
user.foobar
\end{codebox}

\begin{codebox}{Java access style for maps}{teal}{\icnote}{white}
user.get("foobar")
\end{codebox}

It is advisable to favour such access styles over their Java counterparts when writing rules and conditionals for \arara. The clean syntax offer subsidies for a more readable code.

\section{Flow control}
\label{sec:mvelflowcontrol}

The expression language goes beyond simple evaluations. In fact, MVEL supports an assortment of control flow operators (namely, conditionals and repetitions) which allows advanced scripting operations. Consider this conditional statement:

\begin{codebox}{Conditional statement}{teal}{\icnote}{white}
if (var > 0) {
   r = "greater than zero";
}
else if (var == 0) { 
   r = "exactly zero";
}
else { 
   r = "less than zero";
}
\end{codebox}

As seen in the previous code, the syntax is very similar to the ones found in typical programming languages. MVEL also provides a shorter version, know as ternary statement:

\begin{codebox}{Ternary statement}{teal}{\icnote}{white}
answer == true ? "yes" : "no";
\end{codebox}

The \rbox{foreach} statement accepts two parameters separated by a colon, being the first the local variable holding the current element, and the second the collection or array to be iterated. For example:

\begin{codebox}{Iteration statement}{teal}{\icnote}{white}
foreach (name : people) {
    System.out.println(name);
}
\end{codebox}

As expected, MVEL also implements the standard C \rbox{for} loop. Observe that newer versions of MVEL allow an abreviation of \rbox{foreach} to the usual \rbox{for} statement, as syntactic sugar. In order to explicitly indicate a collection iteration, we usually use \rbox{foreach} in the default rules for \arara, but both statements behave exactly the same from a semantic point of view. 

\begin{codebox}{Iteration statement}{teal}{\icnote}{white}
for (int i = 0; i < 100; i++) { 
   System.out.println(i);
}
\end{codebox}

The scripting language also provides two versions of the \rbox{do} statement: one with \rbox{while} and one with \rbox{until} (being the latter the exact inverse of the former):

\begin{codebox}{Iteration statement}{teal}{\icnote}{white}
do {
    x = something();
} while (x != null);
\end{codebox}

\begin{codebox}{Iteration statement}{teal}{\icnote}{white}
do {
   x = something();
} until (x == null);
\end{codebox}

At last, MVEL also implements the standard \rbox{while}, with the significant addition of a \rbox{until} counterpart (for inverted logic):

\begin{codebox}{Iteration statement}{teal}{\icnote}{white}
while (isTrue()) {
   doSomething();
}
\end{codebox}

\begin{codebox}{Iteration statement}{teal}{\icnote}{white}
until (isFalse()) {
   doSomething();
}
\end{codebox}

Since \rbox{while} and \rbox{until} are unbounded (i.e, the number of iterations required to solve a problem may be unpredictable), we usually tend to avoid using such statements when writing rules for \arara.

\section{Projections and folds}
\label{sec:mvelprojectionsandfolds}

Projections are a way of representing collections. According to the official documentation, using a very simple syntax, one can inspect very complex object models inside collections in MVEL using the \rbox{in} operator. For example:

\begin{codebox}{Projection and fold}{teal}{\icnote}{white}
names = (user.name in users);
\end{codebox}

As seen in the previous code, \rbox{names} holds all values from the \rbox{name} property of each element, represented locally by a placeholder \rbox{user}, from the collection \rbox{users} being inspected. This feature can even perform nested operations.

\section{Assignments}
\label{sec:mvelassignments}

According to the official documentation, the scripting language allows variable assignment in expressions, either for extraction from the runtime, or for use inside the expression. As MVEL is a dynamically typed language, there is no need to specify a type in order to declare a new variable. However, feel free to explicitly declare the type when desired.

\begin{codebox}{Assignment}{teal}{\icnote}{white}
str = "My string";
String str = "My string";
\end{codebox}

Unlike Java, however, the scripting language provides automatic type conversion (when possible) when assigning a value to a typed variable. In the following example, an integer value is assigned to a string:

\begin{codebox}{Assignment}{teal}{\icnote}{white}
String num = 1;
\end{codebox}

For dynamically typed variables, in order to perform a type conversion, it is just a matter of explicitly casting the value to the desired type. In the following example, an explicit string cast is assigned to the \rbox{num} variable:

\begin{codebox}{Assignment}{teal}{\icnote}{white}
num = (String) 1;
\end{codebox}

When writing rules for \arara, is advisable to keep variables to a minimum in order to avoid unnecessary assignments and a potential performance drop. However, make sure to favour readability over unmaintained code.

\section{Basic templating}
\label{sec:mvelbasictemplating}

MVEL templates are comprised of \emph{orb} tags inside a plaintext document. Orb tags denote dynamic elements of the template which the engine will evaluate at runtime. \arara\ heavily relies on this concept for runtime evaluation of conditionals and rules. For rules, we use orb tags to return either a string from a textual template or a proper command object. The former constituted the basis of command generation in previous versions of our tool; from version 4.0 on, we highly recommend the latter, detailed in Section~\ref{sec:rule}, on page~\ref{sec:rule}. Conditionals are in fact orb tags in desguise, such that the expression (or a sequence of expression) is properly evaluated at runtime. Consider the following example:

\begin{codebox}{Template}{teal}{\icnote}{white}
My favourite team is @{ person.name == 'Enrico'
? 'Juventus' : 'Palmeiras' }!
\end{codebox}

The previous code features a basic form of orb tag named \emph{expression orb}. It contains a expression (or a sequence of expressions) which will be evaluated to a certain value, as seen earlier on, when discussing the \emph{last value out} principle. In the example, the value to be returned will be a string containing a football team name (the result is of course based on the comparison outcome).

\section{Further reading}
\label{sec:mvelfurtherreading}

This chapter does not cover all features of the MVEL expression language, so further reading is advisable. I highly recommend the \href{http://mvel.documentnode.com/}{MVEL language guide} currently covering version 2.0 of the language.

\end{document}
