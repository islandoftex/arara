% !TeX root = ../arara-manual.tex
\chapter*{Foreword}
\label{chap:foreword}

\epigraph{That deserves no less than a ``Holy guacamole!''.}{\textsc{Gonzalo Medina}}

Creating a PDF from \LaTeX\ code can be quite tiresome. Suppose I'm
using TeXworks and I have a document that has a bibliography,
glossary and index, then I need to select the PDFLaTeX tool and click
on the typeset button, then select the BibTeX (or Biber) tool and click
on the typeset button, then select the MakeIndex tool and click on
the typeset button, then select the MakeGlossaries tool (which I may
need to add first) and click on the typeset button, then select the
PDFLaTeX tool and click on the typeset button, and once more to
ensure all the cross-references are up to date.

Then I edit the document and have to go through that whole process
all over again!

Automation makes life so much simpler. Instead of all those tools
that I need to keep selecting, I just need one tool, in this case
\arara, which will do all the necessary work for me behind the
scenes.

Some automation tools try to be clever, but there are invariably
exceptions that trip them up. \arara\ doesn't try to be clever; it 
just does what it's told to do. The instructions are provided as
special comments in the source code that \TeX\ ignores, but they are
human-readable and can also provide a hint to non-\arara\ co-authors
as to what tools are required in order to complete the document
build.

The new improved \arara\ version 4.0 now comes with some exciting
features, such as the ability to use conditionals, and it definitely
ranks as my favourite automation tool for document creation. Paulo
has done a great job, and I'd like to take this opportunity to thank
him for his patience in dealing with my many feature requests!

\vfill

\begin{flushright}
Nicola Louise Cecilia Talbot\\
\emph{on behalf of the \arara\ team}
\end{flushright}
