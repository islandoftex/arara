% !TeX root = ../arara-manual.tex
\chapter{The official rule pack}
\label{chap:theofficialrulepack}

\arara\ ships with a pack of default rules, placed inside a special subdirectory named \abox[araracolour]{rules/} inside another special directory named \abox[araracolour]{ARARA\_HOME} (the place where our tool is installed). This chapter introduces the official rules, including proper listings and descriptions of associated parameters whenever applied. Note that such rules work off the shelf, without any special installation, configuration or modification. An option marked by \rbox[araracolour]{S} after the corresponding identifier indicates a natural boolean switch. Similarly, the occurrence of an \rqbox\ mark indicates that the corresponding option is required.

\begin{messagebox}{Can my rule be distributed within the official pack?}{araracolour}{\icok}{white}
As seen in Section~\ref{sec:basicstructure}, on page~\pageref{sec:basicstructure}, the default rule path can be extended to include a list of directories in which our tool should search for rules. However, if you believe your rule is comprehensive enough and deserves to be in the official pack, please contact us! We will be more  than happy to discuss the inclusion of your rule in forthcoming updates.
\end{messagebox}

\begin{description}
\item[\rulebox{animate}]
This rule creates an animated \rbox{gif} file from the corresponding base name of the \mtbox{currentFile} reference (i.e, the name without the associated extension) as a string concatenated with the \rbox{pdf} suffix, using the \rbox{convert} command line utility from the ImageMagick suite.

\begin{description}
\item[\rpbox{delay}{10}] This option regulates the number of ticks before the display of the next image sequence, acting as a pause between still frames.

\item[\rpbox{loop}{0}] This option regulates the number of repetitions for the animation. When set to zero, the animation repeats itself an infinite number of times.

\item[\rpbox{density}{300}] This option specifies the horizontal and vertical canvas resolution while rendering vector formats into a proper raster image.

\item[\rpbox{program}{convert}] This option specifies the command utility path as a means to avoid potential clashes with underlying operating system commands.

\begin{messagebox}{Microsoft Windows woes}{attentioncolour}{\icattention}{black}
\setlength{\parskip}{1em}
According to the \href{http://www.imagemagick.org/Usage/windows/}{ImageMagick website}, the Windows installation routine adds the program directory to the system path, such that one can call command line tools directly from the command prompt, without providing a path name. However, \rbox{convert} is also the name of Windows system tool, located in the system directory, which converts file systems from one format to another.

The best solution to avoid possible future name conflicts, according to the ImageMagick team, is to call such command line tools by their full path in any script. Therefore, the \rbox{convert} rule provides the \abox{program} option for this specific scenario.
\end{messagebox}

\item[\abox{options}] This option, as the name indicates, takes a list of raw command line options and appends it to the actual system call. An error is thrown if any data structure other than a proper list is provided as the value.
\end{description}

\begin{codebox}{Example}{teal}{\icnote}{white}
% arara: animate: { delay: 15, density: 150 }
\end{codebox}

\item[\rulebox{asymptote}]
This rule executes the \rbox{asy} command line program, referring to Asymptote, a powerful descriptive vector graphics language for technical drawings, inspired by Metapost but with an improved syntax. Please note that you will have to make the \rbox{.asy} extension known to \arara\ in order to compile Asymptote files. Furthermore, it is advised to use this in your regular \TeX\ document specifying the \abox{files} parameter to include all graphics you want to compile for inclusion in your document.

\begin{description}
\item[\abox{color}] This option, as the name suggests, provides the underlying color model to be used in the current execution. Possible values are:

\begin{description}
\item[\povalue{\hphantom{x}bw\hphantom{x}}] This option value, as the name suggests, converts all colors to a black and white model.

\item[\povalue{cmyk}] This option value converts the RGB (red, green an blue) color model to the CMYK (cyan, magenta, yellow and black) counterpart.

\item[\povalue{rgb}] This option value converts the CMYK (cyan, magenta, yellow and black)  color model to the RGB (red, green an blue) counterpart.

\item[\povalue{gray}] This option value, as the name suggests, converts all colors to a grayscale model.
\end{description}

\item[\rpbox{engine}{latex}] This option, as the name indicates, sets the underlying \TeX\ engine to be used for the current compilation. Make sure to take a look at the Asymptote manual for further details on this option. Possible values are:

\begin{description}
\item[\povalue{latex}] This value, as the name suggests, sets the underlying \TeX\ engine to \rbox{latex} for the current compilation. Note that the engine might play a major role in the generated code.

\item[\povalue{pdflatex}] This value, as the name indicates, sets the underlying \TeX\ engine to \rbox{pdflatex} for the current compilation. Note that the engine might play a major role in the generated code.

\item[\povalue{xelatex}] This value, as the name suggests, sets the underlying \TeX\ engine to \rbox{xelatex} for the current compilation. Note that the engine might play a major role in the generated code.

\item[\povalue{lualatex}] This value, as the name indicates, sets the underlying \TeX\ engine to \rbox{lualatex} for the current compilation. Note that the engine might play a major role in the generated code.

\item[\povalue{tex}] This value, as the name suggests, sets the underlying \TeX\ engine to \rbox{tex} for the current compilation. Note that the engine might play a major role in the generated code.

\item[\povalue{pdftex}] This value, as the name indicates, sets the underlying \TeX\ engine to \rbox{pdftex} for the current compilation. Note that the engine might play a major role in the generated code.

\item[\povalue{luatex}] This value, as the name suggests, sets the underlying \TeX\ engine to \rbox{luatex} for the current compilation. Note that the engine might play a major role in the generated code.

\item[\povalue{context}] This value, as the name indicates, sets the underlying \TeX\ engine to \rbox{context} for the current compilation. Note that the engine might play a major role in the generated code.

\item[\povalue{none}] This value, as the name suggests, sets the underlying \TeX\ engine to \rbox{none} for the current compilation. In this case, there will be no associated engine.
\end{description}

\item[\abox{format}] This option, as the name suggests, converts each output file to a specified format. Make sure to take a look at the Asymptote manual for further details.

\item[\abox{output}] This option, as the name suggests, sets an alternative output directory or file name. Make sure to take a look at the Asymptote manual for further details.

\item[\abox{options}] This option, as the name indicates, takes a list of raw command line options and appends it to the actual system call. An error is thrown if any data structure other than a proper list is provided as the value.
\end{description}

\begin{codebox}{Example}{teal}{\icnote}{white}
% arara: asymptote: { files: [ mydrawing.asy ] }
\end{codebox}

\item[\rulebox{authorindex}]
This rule calls the \rbox{authorindex} wrapper, a Perl script that processes auxiliary files generated by the package of the same name, producing author index files, with the \rbox{ain} extension.

\begin{description}
\item[\rpsbox{draft}] This option sets whether the script should write additional information to the produced file. For each author, the
labels of all references and the page numbers where they are cited are included as comments. This detail may help if you manually edit the generated author index.

\item[\rpsbox{index}] This option sets whether the script should create a file suitable for further processing with \rbox{makeindex} or
the like. For example, you could use that to make a common author and subject index. Note the extension of the generated file still will be the default one.

\item[\rpsbox{keep}] This option sets whether the script should retain the temporarily generated \rbox{bst} file after the run finishes. This information will give you a good starting point for advanced customization of the author index.

\item[\rpsbox{print}] This option, as the name indicates, sets whether the script should print the result to standard output instead of writing it to the output file.

\item[\rpsbox{recurse}] This option sets whether the script should automatically process auxiliary files produced by included files. This behaviour is enabled by default.

\item[\abox{options}] This option, as the name indicates, takes a list of raw command line options and appends it to the actual system call. An error is thrown if any data structure other than a proper list is provided as the value.
\end{description}

\begin{codebox}{Example}{teal}{\icnote}{white}
% arara: authorindex: { draft: yes }
\end{codebox}

\item[\rulebox{bib2gls}]
This rule executes the \rbox{bib2gls} command line application which extracts glossary information stored in a \rbox{bib} file and converts it into glossary entry definitions in resource files. This rule passes the base name of the \mtbox{currentFile} reference (i.e, the name without the associated extension) as the mandatory argument.

\begin{description}
\item[\abox{dir}] This option sets the directory used for writing auxiliary files. Note that this option does not change the current working directory.

\item[\abox{trans}] This option sets the extension of the transcript file created by \rbox{bib2gls}. The value should be just the file extension without the leading dot. The default is \rbox{glg}.

\item[\abox{locale}] This option specifies the preferred language resource file. Please keep in mind that the provided value must be a valid IETF language tag. If omitted, the default is obtained by \rbox{bib2gls} from the JVM.

\item[\rpsbox{group}] This option sets whether \rbox{bib2gls} will try to determine the letter group for each entry and add it to a new field called \rbox{group} when sorting. Be mindful that some \rbox{sort} options ignore this setting. The default value is off.

\item[\rpsbox{interpret}] This option sets whether the interpreter mode of \rbox{bib2gls} is enabled. If the interpreter is on, \rbox{bib2gls} will attempt to convert any \LaTeX\ markup in the sort value to the closest matching Unicode characters. If the interpreter is off, the \rbox{log} file will not be parsed for recognised packages. The default value is on.

\item[\rpsbox{breakspace}] This option sets whether the interpreter will treat a tilde character as a non-breaking space (as with \TeX) or a normal space. The default behaviour treats it as non-breakable.

\item[\rpsbox{trimfields}] This option sets whether \rbox{bib2gls} will trim leading and trailing spaces from field values. The default behaviour does not trim spaces.

\item[\rpsbox{recordcount}] This option sets whether the record counting will be enabled. If activated, \rbox{bib2gls} will add record count fields to entries. The default behaviour is off.

\item[\rpsbox{recordcountunit}] This option sets whether \rbox{bib2gls} will add unit record count fields to entries. These fields can then be used with special commands. The default behaviour is off.

\item[\rpsbox{cite}] This option sets whether \rbox{bib2gls} will treat citation instances found in the \rbox{aux} file as though it was actually an ignored record. The default behaviour is off.

\item[\rpsbox{verbose}] This option sets whether \rbox{bib2gls} will be executed in verbose mode. When enabled, the application will write extra information to the terminal and transcript file. This option is unrelated to \arara's verbose mode. The default behaviour is off.

\item[\rpsbox{merge}] This option sets whether the program will merge \rbox{wrglossary} counter records. If disabled, one may end up with duplicate page numbers in the list of entry locations, but linking to different parts of the page. The default is on.

\item[\rpsbox{uniscript}] This option sets whether text superscript and subscript will use the corresponding Unicode characters if available. The default is on.

\item[\abox{packages}] This option instructs the interpreter to assume the packages from the provided list have been used by the document.

\item[\abox{ignore}] This option instructs \rbox{bib2gls} to skip the check for any package from the provided list when parsing the corresponding log file.

\item[\abox{custom}] This option instructs the interpreter to parse the package files from the provided list. The package files need to be quite simple.

\item[\abox{mapformats}] This option takes a list and sets up the rule of precedence for partial location matches. Each element from the provided list must be another list of exactly two entries representing a conflict resolution.

\item[\abox{options}] This option, as the name indicates, takes a list of raw command line options and appends it to the actual system call. An error is thrown if any data structure other than a proper list is provided as the value.
\end{description}

\begin{codebox}{Example}{teal}{\icnote}{white}
% arara: bib2gls: { group: true }
% arara: --> if found('aux', 'glsxtr@resource')
\end{codebox}

\item[\rulebox{biber}]
This rule runs \rbox{biber}, the backend bibliography processor for \rbox{biblatex}, on the corresponding base name of the \mtbox{currentFile} reference (i.e, the name without the associated extension) as a string.

\begin{description}
\item[\rpsbox{tool}]  This option sets whether the bibliography processor should be executed in \emph{tool mode}, intended for transformations and modifications of datasources. Since this mode is oriented towards a datasource rather than a document, make sure to use it alongside the \abox{options} option.

\item[\abox{options}] This option, as the name indicates, takes a list of raw command line options and appends it to the actual system call. An error is thrown if any data structure other than a proper list is provided as the value.
\end{description}

\begin{codebox}{Example}{teal}{\icnote}{white}
% arara: biber: { options: [ '--wraplines' ] }
\end{codebox}

\item[\rulebox{bibtex}]
This rule runs the \rbox{bibtex} program, a reference management software, on the corresponding base name of the \mtbox{currentFile} reference (i.e, the name without the associated extension) as a string.

\begin{description}
\item[\abox{options}] This option, as the name indicates, takes a list of raw command line options and appends it to the actual system call. An error is thrown if any data structure other than a proper list is provided as the value.
\end{description}

\begin{codebox}{Example}{teal}{\icnote}{white}
% arara: bibtex: { options: [ '-terse' ] }
% arara: --> if exists(toFile('references.bib'))
\end{codebox}

\item[\rulebox{bibtex8}]
This rule runs \rbox{bibtex8}, an enhanced, portable C version of \rbox{bibtex}, on the corresponding base name of the \mtbox{currentFile} reference (i.e, the name without the associated extension) as a string. It is important to note that this tool can read a character set file containing encoding details.

\begin{description}
\item[\abox{options}] This option, as the name indicates, takes a list of raw command line options and appends it to the actual system call. An error is thrown if any data structure other than a proper list is provided as the value.
\end{description}

\begin{codebox}{Example}{teal}{\icnote}{white}
% arara: bibtex8: { options: [ '--trace', '--huge' ] }
\end{codebox}

\item[\rulebox{bibtexu}]
This rule runs the \rbox{bibtexu} program, an enhanced version of \rbox{bibtex} with Unicode support and language features, on the corresponding base name of the \mtbox{currentFile} reference (i.e, the name without the associated extension) as a string.

\begin{description}
\item[\abox{options}] This option, as the name indicates, takes a list of raw command line options and appends it to the actual system call. An error is thrown if any data structure other than a proper list is provided as the value.
\end{description}

\begin{codebox}{Example}{teal}{\icnote}{white}
% arara: bibtexu: { options: [ '--language', 'fr' ] }
\end{codebox}

\item[\rulebox{clean}]
This rule removes the provided file reference through the underlying system command, which can be \rbox{rm} in a Unix environment or \rbox{del} in Microsoft Windows. As a security lock, this rule will always throw an error if \mtbox{currentFile} is equal to \mtbox{getOriginalFile}, so the main file reference cannot be removed. It is highly recommended to use the special \abox{files} parameter to indicate removal candidates. Alternatively, a list of file extensions can be provided as well. Be mindful that the security lock also applies to file removals based on extensions.

\begin{description}
\item[\abox{extensions}] This option, as the name indicates, takes a list of extensions and constructs a new list of removals commands according to the base name of the \mtbox{currentFile} reference (i.e, the name without the associated extension) as a string concatenated with each extension from the original list as suffixes. Keep in mind that, if the special \abox{files} parameter is used with this option, the resulting list will contain the cartesian product of file base names and extensions. An error is thrown if any data structure other than a proper list is provided as the value.

\begin{messagebox}{Better safe than sorry!}{attentioncolour}{\icattention}{black}
When in doubt, always remember that the \opbox{{-}dry-run} command line option is your friend! As seen in Section~\ref{sec:options}, on page~\pageref{sec:options}, this option makes \arara\ go through all the motions of running tasks and subtasks, but with no actual calls. It is a very useful feature for testing the sequence of removal commands without actually losing your files! Also, as good practice, always write plain, simple, understandable \rbox{clean} directives and use as many as needed in your \TeX\ documents.
\end{messagebox}
\end{description}

\begin{codebox}{Example}{teal}{\icnote}{white}
% arara: clean: { extensions: [ aux, log ] }
\end{codebox}

\item[\rulebox{context}]
This rule runs the \rbox{context} \TeX\ engine on the provided \mtbox{currentFile} reference, generating a corresponding file in the Portable Document Format. Please refer to the user manual for further details.

\begin{description}
\item[\rpsbox{make}] This option, as the name indicates, sets whether the engine should create context formats. Please refer to the user manual for further details on this option.

\item[\abox{ctx}] This option, as the name indicates, sets the \rbox{ctx} file for process management specification. Please refer to the user manual for further details on this option.

\item[\abox{interface}] This option, as the name indicates, sets the specified user interface. Please refer to the user manual for further details on this option.

\item[\rpsbox{autopdf}] This option, as the name indicates, closes the corresponding \rbox{pdf} file in viewer and then reopens it afterwards.

\item[\abox{purge}] This option, as the name indicates, purges files according to the provided rule. Possible values are: 

\begin{description}
\item[\povalue{partial}] This value, as the name suggests, purges files either or not after a proper run, defined through a pattern. Please refer to the user manual for further details.

\item[\povalue{all}] This value, as the name suggests, purges all files either or not after a proper run, defined through a pattern. Please refer to the user manual for further details.

\item[\povalue{result}] This value, as the name suggests, purges the resulting file before the actual run. Please refer to the user manual for further details.
\end{description}

\item[\abox{modules}] This option, as the name indicates, sets a list of modules and styles to be loaded, normally part of the distribution.

\item[\abox{environments}] This option, as the name indicates, sets a list of environment files (document styles) to load first.

\item[\abox{mode}] This option, as the name indicates, enables modes according to the provided list (conditional processing in styles).

\item[\abox{path}] This option, as the name indicates, consults the given paths in the provided list during a file lookup.

\item[\abox{arguments}] This option, as the name indicates, sets variables that can be consulted during a run. Such variables are defined as key/value pairs.

\item[\abox{trackers}]  This option, as the name indicates, sets the list of tracker variables. Please refer to the user manual for further details.

\item[\abox{directives}]  This option, as the name indicates, sets the list of directive variables. Please refer to the user manual for further details.

\item[\abox{silent}] This option, as the name indicates, disables the log categories based on the provided list. Please refer to the user manual for further details.

\item[\abox{errors}]  This option, as the name indicates, shows errors at the end of a run and quits when it the provided list. Please refer to the user manual for further details.

\item[\rpsbox{synctex}] This option sets whether \rbox{synctex}, an input and output synchronization feature that allows navigation from source to typeset material and vice versa, available in most \TeX\ engines, is activated.

\item[\abox{interaction}] This option alters the underlying engine behaviour. If this option is omitted, the engine will prompt the user for interaction in the event of an error. Possible values are, in order of increasing user interaction (courtesy of our master Enrico Gregorio):

\begin{description}
\item[\povalue{batchmode}] In this mode, nothing is printed on the terminal, and errors are scrolled as if the \rbox{return} key is hit at every error. Missing files that \TeX\ tries to input or request from keyboard input cause the job to abort.

\item[\povalue{nonstopmode}] In this mode, the diagnostic message will appear on the terminal, but there is no possibility of user interaction just like in batch mode, previously described.
\end{description}

\item[\abox{options}] This option, as the name indicates, takes a list of raw command line options and appends it to the actual system call. An error is thrown if any data structure other than a proper list is provided as the value.
\end{description}

\begin{codebox}{Example}{teal}{\icnote}{white}
% arara: context
\end{codebox}

\item[\rulebox{convert}]
This rule runs the \rbox{convert} program, a member of the ImageMagick suite of tools. This program is used to convert between image formats as well  as  resize  an  image, blur, crop, despeckle, dither, draw on, flip, join, resample, and more.

\begin{description}
\item[\rpbox{program}{convert}] This option specifies the command utility path as a means to avoid potential clashes with underlying operating system commands.

\item[\abox{options}~\rqbox] This option, as the name indicates, takes a list of raw command line options and appends it to the actual script call. An error is thrown if any data structure other than a proper list is provided as the value.
\end{description}

\begin{codebox}{Example}{teal}{\icnote}{white}
% arara: convert: { options: [ 'photo1.jpg', '--resize',
% arara: --> '50%', 'photo2.jpg' ] }
\end{codebox}

\item[\rulebox{copy}]
This rule copies the \rbox{File} reference to the provided target using the underlying operating system copy operation. The target is \emph{always} overwritten. However, be mindful that an error will be thrown if you try to overwrite the file referenced in \mtbox{getOriginalReference}.

\begin{description}
\item[\abox{target}~\rqbox] This option, as the name implies, specifies the target for the copy operation. Keep in mind that this option is required.
\end{description}

\begin{codebox}{Example}{teal}{\icnote}{white}
% arara: copy: { target: 'backup/thesis.tex' }
\end{codebox}

\item[\rulebox{csplain}]
This rule runs the \rbox{csplain} \TeX\ engine, a conservative extension of Knuth's plain \TeX\ with direct processing characters and hyphenation patterns for Czech and Slovak, on the provided \mtbox{currentFile} reference.

\begin{description}
\item[\abox{interaction}] This option alters the underlying engine behaviour. When such option is omitted, \TeX\ will prompt the user for interaction in the event of an error. Possible values are, in order of increasing user interaction (courtesy of our master Enrico Gregorio):

\begin{description}
\item[\povalue{batchmode}] In this mode, nothing is printed on the terminal, and errors are scrolled as if the \rbox{return} key is hit at every error. Missing files that \TeX\ tries to input or request from keyboard input cause the job to abort.

\item[\povalue{nonstopmode}] In this mode, the diagnostic message will appear on the terminal, but there is no possibility of user interaction just like in batch mode, previously described.

\item[\povalue{scrollmode}] In this mode, as the name indicates, \TeX\ will stop only for missing files to input or if proper keyboard input is necessary. \TeX\ fixes errors itself.

\item[\povalue{errorstopmode}] In this mode, \TeX\ will stop at each error, asking for proper user intervention. This is the most user interactive mode available.
\end{description}

\item[\rpsbox{shell}] This option sets whether the possibility of running underlying system commands from within \TeX\ is activated.

\item[\rpsbox{synctex}] This option sets whether \rbox{synctex}, an input and output synchronization feature that allows navigation from source to typeset material and vice versa, available in most \TeX\ engines, is activated.

\item[\rpsbox{draft}] This option sets whether the draft mode, i.e, a mode that produces no output, so the engine can check the syntax, is activated.

\item[\abox{options}] This option, as the name indicates, takes a list of raw command line options and appends it to the actual system call. An error is thrown if any data structure other than a proper list is provided as the value.
\end{description}

\begin{codebox}{Example}{teal}{\icnote}{white}
% arara: csplain: { interaction: batchmode, shell: yes }
\end{codebox}

\item[\rulebox{datatooltk}]
This rule runs \rbox{datatooltk}, an application that creates \rbox{datatool} databases in raw format from several structured data formats, in batch mode. This rule requires \abox{output} and one of the import options.

\begin{description}
\item[\abox{output}~\rqbox] This option provides the database name to be saved as output. To guard against accidentally overwriting a document file, \rbox{datatooltk} now forbids the \rbox{tex} extension for output files. This option is required.

\item[\abox{csv}] This option, as the name indicates, imports data from the \rbox{csv} file reference provided as a plain string value.

\item[\abox{sep}] This option specifies the character used to separate values in the \rbox{csv} file. Defaults to a comma.

\item[\abox{delim}] This option specifies the character used to delimit values in the \rbox{csv} file. Defaults to a double quote.

\item[\abox{name}] This option, as the name indicates, sets the label reference of the newly created database according to the provided value.

\item[\abox{sql}] This option imports data from an SQL database where the provided value refers to a proper \rbox{select} SQL statement.

\item[\abox{sqldb}] This option, as the name indicates, sets the name of the SQL database according to the provided value.

\item[\abox{sqluser}] This option, as the name indicates, sets the name of the SQL user according to the provided value.

\item[\rpbox{noconsole}{gui}] This action dictates the password request action if such information was not provided earlier. If there is no console available, the action is determined by the following values:

\begin{description}
\item[\povalue{error}] As the name indicates, this action issues an error when no password was previously provided through the proper option.

\item[\povalue{stdin}] This action requests the password via the standard input stream, which is less secure than using a console.

\item[\povalue{gui}] This action displays a dialog box in which the user can enter the password for the SQL database.
\end{description}

\item[\abox{probsoln}] This option, as the name indicates, imports data in the \rbox{probsoln} format from the file name provided as the value.

\item[\abox{input}] This option, as the name indicates, imports data in the \rbox{datatool} format from the file name provided as the value.

\item[\abox{sort}] This option, as the name indicates, sorts the database according to the column whose label is provided as the value. The value may be preceded by \rbox{+} or \rbox{-} to indicate ascending or descending order, respectively. If the sign is omitted, ascending is assumed.

\item[\abox{sortlocale}] This option, as the name indicates, sorts the database according to alphabetical order rules of the locale provided as the value. If the value is set to \rbox{none} strings are sorted according to non-locale letter order.

\item[\rpsbox{sortcase}] This option sets whether strings will be sorted using case-sensitive comparison for non-locale letter ordering. The default behaviour is case-insensitive.

\item[\abox{seed}] This option, as the name indicates, sets the random generator seed to the provided value. The seed is cleared if an empty value is provided.

\item[\rpsbox{shuffle}] This option sets whether the database will be properly shuffled. Shuffle is always performed after sort, regardless of the option order.

\item[\rpsbox{csvheader}] This option sets whether the \rbox{csv} file has a header row. The
spreadsheet import functions also use this setting.

\item[\rpsbox{debug}] This option, as the name indicates, sets whether the debug mode of \rbox{datatooltk} is activated. The debug mode is disabled by default.

\item[\rpsbox{owneronly}] This option sets whether read and write permissions when saving \rbox{dbtex} files should be defined for the owner only. This option has no effect on some operating systems.

\item[\rpsbox{maptex}] This option sets whether \TeX\ special characters will be properly mapped when importing data from \rbox{csv} files or SQL databases.

\item[\abox{xls}] This option, as the name indicates, imports data from a Microsoft Excel \rbox{xls} file reference provided as a plain string value.

\item[\abox{ods}] This option, as the name indicates, imports data from an Open Document Spreadsheet \rbox{ods} file reference provided as a plain string value.

\item[\abox{sheet}] This option specifies the sheet to select from the Excel workbook or Open Document Spreadsheet. This may either be an index or the name of the sheet.

\item[\abox{filterop}] This option specifies the logical operator to be associated with a given filter. Filtering is always performed after sorting and shuffling. Possible values are:

\begin{description}
\item[\povalue{or}\hfill\hphantom{w}] This value, as the name indicates, uses the logical \rbox{or} operator when filtering. This is the default behaviour. Note that this value has no effect if only one filter is supplied.

\item[\povalue{and}] This value, as the name indicates, uses the logical \rbox{and} operator when filtering. Note that this value has no effect if only one filter is supplied.
\end{description}

\item[\abox{filters}] This option takes a list and sets up a sequence of filters. Each element from the provided list must be another list of exactly three entries representing a key, an operator and a value, respectively.

\item[\abox{truncate}] This option truncates the database to the number of rows provided as the value. Truncation is always performed after any sorting, shuffling and filtering, but before column removal.
\end{description}

\begin{codebox}{Example}{teal}{\icnote}{white}
% arara: datatooltk: {
% arara: --> output: books.dbtex,
% arara: --> csv: booklist.csv }
\end{codebox}

\item[\rulebox{detex}]
This rule runs \rbox{detex}, a command line application that, as the name indicates, reads the provided \mtbox{currentFile} reference, removes all comments and \TeX\ control sequences and writes the remainder to the standard output or file.

\begin{description}
\item[\rpsbox{references}] This option defines whether the tool should echo values from page counters, references and citations.

\item[\rpsbox{follow}] This option defines whether the tool should follow files referenced through typical input mechanisms.

\item[\rpsbox{math}] This option defines whether the tool should replace math with nouns and verbs in order to preserve grammar.

\item[\rpsbox{spaces}] This option, as the name indicates, define whether the tool should replace control sequences with spaces.

\item[\rpsbox{words}] This option, as the name indicates, define whether the tool should output only a list of words.

\item[\abox{environments}] This option takes a list of environments to be ignored by the tool during the text transformation.

\item[\abox{mode}] This option, as the name indicates, defines the operation mode for the tool. Possible values are:

\begin{description}
\item[\povalue{latex}] This value, as the name suggests, enables and forces the \LaTeX\ mode for the tool.

\item[\povalue{tex}] This value, as the name suggests, enables and forces the plain \TeX\ mode for the tool.
\end{description}

\item[\abox{output}] This value, as the name indicates, sets the output file in which the contents will be redirected.
\end{description}

\begin{codebox}{Example}{teal}{\icnote}{white}
% arara: detex: { output: thesis.txt }
\end{codebox}

\item[\rulebox{dvipdfm}]
This rule runs \rbox{dvipdfm}, a command line utility for file format translation, on the corresponding base name of the \mtbox{currentFile} reference (i.e, the name without the associated extension) as a string concatenated with the \rbox{dvi} suffix, generating a Portable Document Format \rbox{pdf} file.

\begin{description}
\item[\abox{output}] This option, as the name indicates, sets the output name for the generated \rbox{pdf} file. There is no need to provide an extension, as the value is always normalized with \mtbox{getBasename} such that only the name without the associated extension is used. The base name of the current file reference is used as the default value.

\item[\abox{options}] This option, as the name indicates, takes a list of raw command line options and appends it to the actual system call. An error is thrown if any data structure other than a proper list is provided as the value.
\end{description}

\begin{codebox}{Example}{teal}{\icnote}{white}
% arara: dvipdfm: { output: thesis }
\end{codebox}

\item[\rulebox{dvipdfmx}]
This rule runs \rbox{dvipdfmx}, an extended version of \rbox{dvipdfm} created to support multibyte character encodings and large character sets for East Asian languages, on the corresponding base name of the \mtbox{currentFile} reference (i.e, the name without the associated extension) as a string concatenated with the \rbox{dvi} suffix, generating a Portable Document Format \rbox{pdf} file.

\begin{description}
\item[\abox{output}] This option, as the name indicates, sets the output name for the generated \rbox{pdf} file. There is no need to provide an extension, as the value is always normalized with \mtbox{getBasename} such that only the name without the associated extension is used. The base name of the current file reference is used as the default value.

\item[\abox{options}] This option, as the name indicates, takes a list of raw command line options and appends it to the actual system call. An error is thrown if any data structure other than a proper list is provided as the value.
\end{description}

\begin{codebox}{Example}{teal}{\icnote}{white}
% arara: dvipdfmx: { options: [ '-K', '40' ] }
\end{codebox}

\item[\rulebox{dvips}]
This rule runs \rbox{dvips} on the corresponding base name of the \mtbox{currentFile} reference (i.e, the name without the associated extension) as a string concatenated with the \rbox{dvi} suffix, generating a PostScript \rbox{ps} file.

\begin{description}
\item[\abox{output}] This option, as the name indicates, sets the output name for the generated \rbox{ps} file. There is no need to provide an extension, as the value is always normalized with \mtbox{getBasename} such that only the name without the associated extension is used. The base name of the current file reference is used as the default value.

\item[\abox{options}] This option, as the name indicates, takes a list of raw command line options and appends it to the actual system call. An error is thrown if any data structure other than a proper list is provided as the value.
\end{description}

\begin{codebox}{Example}{teal}{\icnote}{white}
% arara: dvips: { output: thesis }
\end{codebox}

\item[\rulebox{dvipspdf}]
This rule runs \rbox{dvips} in order to obtain a corresponding \rbox{ps} file from the initial \rbox{dvi} reference, and then runs \rbox{ps2pdf} on the previously generated \rbox{ps} file in order to obtain a \rbox{pdf} file. Note that all base names are acquired from the \mtbox{currentFile} reference (i.e, the name without the associated extension) and used to construct the resulting files.

\begin{description}
\item[\abox{output}] This option, as the name indicates, sets the output name for the generated \rbox{pdf} file. There is no need to provide an extension, as the value is always normalized with \mtbox{getBasename} such that only the name without the associated extension is used. The base name of the current file reference is used as the default value.

\item[\abox{options1}] This option, as the name indicates, takes a list of raw command line options and appends it to the \rbox{dvips} program call. An error is thrown if any data structure other than a proper list is provided as the value.

\item[\abox{options2}] This option, as the name indicates, takes a list of raw command line options and appends it to the \rbox{ps2pdf} program call. An error is thrown if any data structure other than a proper list is provided as the value.
\end{description}

\begin{codebox}{Example}{teal}{\icnote}{white}
% arara: dvipspdf: { output: article }
\end{codebox}

\item[\rulebox{dvisvgm}]
This rule runs \rbox{dvisvgm} in order to obtain a corresponding \rbox{svg} file, a vector graphics format based on XML, from the initial reference. It is important to observe that the base name is acquired from the \mtbox{currentFile} reference (i.e, the name without the associated extension) and used to construct the resulting file.

\begin{description}
\item[\abox{entry}] This option sets the extension to be used for the initial reference (i.e, the current file name) as input to the command line tool. The following values are available for this option:

\begin{description}
\item[\povalue{dvi}] This value sets the extension to be used for the initial reference as a device independent format. This is the default value when no value is provided.

\item[\povalue{xdv}] This value sets the extension to be used for the initial reference as an extended device independent format.

\item[\povalue{eps}] This value sets the extension to be used for the initial reference as an encapsulated PostScript graphics format.

\item[\povalue{pdf}] This value sets the extension to be used for the initial reference as a Portable Document Format.
\end{description}

\item[\abox{pages}] This value, as the name implies, takes a list of integers indicating the pages to be processed by the command line tool.

\item[\abox{options}] This option, as the name indicates, takes a list of raw command line options and appends it to the actual system call. An error is thrown if any data structure other than a proper list is provided as the value.
\end{description}

\begin{codebox}{Example}{teal}{\icnote}{white}
% arara: dvisvgm
\end{codebox}

\item[\rulebox{etex}]
This rule runs the \rbox{etex} extended (plain) \TeX\ engine on the provided \mtbox{currentFile} reference, generating a corresponding file in a device independent format.

\begin{description}
\item[\abox{interaction}] This option alters the underlying engine behaviour. If this option is omitted, \TeX\ will prompt the user for interaction in the event of an error. Possible values are, in order of increasing user interaction (courtesy of our master Enrico Gregorio):

\begin{description}
\item[\povalue{batchmode}] In this mode, nothing is printed on the terminal, and errors are scrolled as if the \rbox{return} key is hit at every error. Missing files that \TeX\ tries to input or request from keyboard input cause the job to abort.

\item[\povalue{nonstopmode}] In this mode, the diagnostic message will appear on the terminal, but there is no possibility of user interaction just like in batch mode, previously described.

\item[\povalue{scrollmode}] In this mode, as the name indicates, \TeX\ will stop only for missing files to input or if proper keyboard input is necessary. \TeX\ fixes errors itself.

\item[\povalue{errorstopmode}] In this mode, \TeX\ will stop at each error, asking for proper user intervention. This is the most user interactive mode available.
\end{description}

\item[\rpsbox{shell}] This option sets whether the possibility of running underlying system commands from within \TeX\ is activated.

\item[\abox{options}] This option, as the name indicates, takes a list of raw command line options and appends it to the actual system call. An error is thrown if any data structure other than a proper list is provided as the value.
\end{description}

\begin{codebox}{Example}{teal}{\icnote}{white}
% arara: etex: {
% arara: --> shell: yes,
% arara: --> interaction: batchmode
% arara: --> }
\end{codebox}

\item[\rulebox{fig2dev}]
This rule runs \rbox{fig2dev}, a command line application that translates \rbox{fig} code in the corresponding base name of the \mtbox{currentFile} reference (i.e, the name without the associated extension) as a string concatenated with the \rbox{fig} suffix into the specified graphics language and puts them in the specified output file.

\begin{description}
\item[\abox{language}~\rqbox] This option, as the name indicates, sets the output graphics language for proper transformation. Possible values are listed in the following table (you can infer their types as well, based on the values):

\bigskip

{\centering
\def\arraystretch{1.5}
\begin{tabularx}{0.8\textwidth}{YYYYYY}
\povalue{box} & \povalue{cgm} & \povalue{epic} &
\povalue{eepic} & \povalue{eepicemu} & \povalue{emf}\\
\povalue{eps} & \povalue{gif} & \povalue{ibmgl} &
\povalue{jpeg} & \povalue{latex} & \povalue{map}\\
\povalue{mf} & \povalue{mp} & \povalue{mmp} &
\povalue{pcx} & \povalue{pdf} & \povalue{pdftex}\\
\povalue{pdftex\_t} & \povalue{pic} & \povalue{pictex} &
\povalue{png} & \povalue{ppm} & \povalue{ps}\\
\povalue{pstex} & \povalue{pstex\_t} & \povalue{ptk} &
\povalue{shape} & \povalue{sld} & \povalue{svg}\\
\povalue{textyl} & \povalue{tiff} & \povalue{tk} &
\povalue{tpic} & \povalue{xbm} & \povalue{xpm}
\end{tabularx}\par}

\bigskip

\item[\abox{output}~\rqbox] This option, as the name indicates, sets the output file which will contain the translated code based on the provided file and language.

\item[\abox{magnification}] This option, as the name suggests, sets the magnification level at which the figure is rendered.

\item[\abox{font}] This option sets the default font used for text objects to the provided value. Keep in mind that the format of this option depends on the graphics language in use.

\item[\abox{size}] This option, as the name suggests, set the default font size (in points) for text objects to the provided value.

\item[\abox{options}] This option, as the name indicates, takes a list of raw command line options and appends it to the actual system call. An error is thrown if any data structure other than a proper list is provided as the value.
\end{description}

\begin{codebox}{Example}{teal}{\icnote}{white}
% arara: fig2dev: { language: mp, output: drawing.mp }
\end{codebox}

\item[\rulebox{frontespizio}]
This rule automates the steps required by the \rbox{frontespizio} package in order to help Italian users generate the frontispiece to their thesis. First and foremost, the frontispiece is generated. If \rbox{latex} is used as the underlying engine, there is an additional intermediate conversion step to a proper \rbox{eps} file. Finally, the final document is compiled.

\begin{description}
\item[\rpbox{engine}{pdflatex}] This option, as the name indicates, sets the underlying \TeX\ engine to be used for both compilations (the frontispiece and the document itself). Possible values are:

\begin{description}
\item[\povalue{latex}] This value, as the name indicates, sets the underlying \TeX\ engine to \rbox{latex} for both compilations (frontispiece and document).

\item[\povalue{pdflatex}] This value, as the name indicates, sets the underlying \TeX\ engine to \rbox{pdflatex} for both compilations (frontispiece and document).

\item[\povalue{xelatex}] This value, as the name indicates, sets the underlying \TeX\ engine to \rbox{xelatex} for both compilations (frontispiece and document).

\item[\povalue{lualatex}] This value, as the name indicates, sets the underlying \TeX\ engine to \rbox{lualatex} for both compilations (frontispiece and document).
\end{description}

\item[\rpsbox{shell}] This option sets whether the possibility of running underlying system commands from within the selected \TeX\ engine is activated.

\item[\abox{interaction}] This option alters the underlying engine behaviour. If this option is omitted, \TeX\ will prompt the user for interaction in the event of an error. Possible values are, in order of increasing user interaction (courtesy of our master Enrico Gregorio):

\begin{description}
\item[\povalue{batchmode}] In this mode, nothing is printed on the terminal, and errors are scrolled as if the \rbox{return} key is hit at every error. Missing files that \TeX\ tries to input or request from keyboard input cause the job to abort.

\item[\povalue{nonstopmode}] In this mode, the diagnostic message will appear on the terminal, but there is no possibility of user interaction just like in batch mode, previously described.

\item[\povalue{scrollmode}] In this mode, as the name indicates, \TeX\ will stop only for missing files to input or if proper keyboard input is necessary. \TeX\ fixes errors itself.

\item[\povalue{errorstopmode}] In this mode, \TeX\ will stop at each error, asking for proper user intervention. This is the most user interactive mode available.
\end{description}

\item[\abox{options}] This option, as the name indicates, takes a list of raw command line options and appends it to the actual \TeX\ engine call. An error is thrown if any data structure other than a proper list is provided as the value.
\end{description}

\begin{codebox}{Example}{teal}{\icnote}{white}
% arara: frontespizio: { engine: xelatex,
% arara: --> shell: yes, interaction: nonstopmode }
\end{codebox}

\item[\rulebox{ghostscript}]
This rule runs \rbox{ghostscript}, an interpreter for PostScript and Portable Document Format files, according to the provided parameters.

\begin{description}
\item[\rpbox{program}{gs}] This option specifies the command utility path as a means to avoid potential clashes with underlying operating system commands or specific Windows naming schemes.

\item[\abox{options}~\rqbox] This option, as the name indicates, takes a list of raw command line options and appends it to the actual \rbox{ghostscript} call. An error is thrown if any data structure other than a proper list is provided as the value. This option is required.

\item[\abox{device}] This option specifies which output device the tool should use. If this option is not given, the default device (usually a display device) is used.

\item[\abox{output}] This option, as the name indicates, specifies a file in which the tool should send the output. Please refer to the documentation for more details.
\end{description}

\begin{codebox}{Example}{teal}{\icnote}{white}
% arara: ghostscript: { options: [ '-dCompatibilityLevel=1.4',
% arara: --> '-dPDFSETTINGS=/printer', '-dNOPAUSE', '-dQUIET',
% arara: --> '-dBATCH', 'input.pdf' ],
% arara: --> output: output.pdf,
% arara: --> device: pdfwrite }
\end{codebox}

\item[\rulebox{gnuplot}]
This rule runs \rbox{gnuplot}, a command-driven plotting program that can generate plots of functions, data and data fits. The program also provides scripting capabilities, looping, functions, text processing, variables, macros, arbitrary pre-processing of input data (usually across columns), as well as the ability to perform non-linear multi-dimensional multi-set weighted data fitting.

\begin{description}
\item[\rpsbox{persist}] This option, as the name implies, sets whether the program should let plot windows survive after the main execution exits.

\item[\rpsbox{default}] this option, as the name suggests, sets whether the program should read the default settings from either \rbox{gnuplotrc} or \rbox{~/.gnuplot} on entry.

\item[\abox{commands}] This option, as the name implies, executes the requested commands before loading the next input file. Please refer to the user manual for further details.

\item[\abox{input}~\rqbox] This required option, as the name indicates, sets the list of input file names to be processed by the program. An error is thrown if any data structure other than a proper list is provided as the value.
\end{description}

\begin{codebox}{Example}{teal}{\icnote}{white}
% arara: gnuplot: { input: [ myplot.gnuplot ], default: yes }
\end{codebox}

\item[\rulebox{halt}]
This rule, as the name suggests, sets a \rbox{halt} flag, which stops the current interpretation workflow, such that subsequent directives are ignored. This rule contains no associated options. Please refer to Section~\ref{sec:others}, on page~\pageref{sec:others}, for more information on flags.

\begin{codebox}{Example}{teal}{\icnote}{white}
% arara: halt
\end{codebox}

\item[\rulebox{indent}]
This rule runs \rbox{latexindent}, a Perl script that indents \TeX\ files according to an indentation scheme, on the provided \mtbox{currentFile} reference. Environments, including those with alignment delimiters, and commands, including those that can split braces and brackets across lines, are usually handled correctly by the script.

\begin{description}
\item[\rpsbox{silent}] This option, as the name indicates, sets whether the script will operate in silent mode, in which no output is given to the terminal.

\item[\rpsbox{overwrite}] This option, as the name indicates, sets whether the \mtbox{currentFile} reference will be overwritten. If activated, a copy will be made before the actual indentation process.

\item[\abox{trace}] This option, as the name indicates, enables the script tracing mode, such that a verbose output will be given to the \rbox{indent.log} log file. Possible values are:

\begin{description}
\item[\povalue{default}] This value, as the name indicates, refers to the default tracing level. Note that, especially for large files, this value does affect performance of the script.

\item[\povalue{complete}] This value, as the name indicates, refers to the detailed, complete tracing level. Note that, especially for large files, performance of the script will be significantly affected when this value is used.
\end{description}

\item[\rpsbox{screenlog}] This option, as the name indicates, sets whether \rbox{latexindent} will output the log file to the screen, as well as to the specified log file.

\item[\rpsbox{modifylinebreaks}] This option, as the name indicates, sets whether the script will modify line breaks, according to specifications written in a configuration file.

\item[\abox{cruft}] This option sets the provided value as a cruft location in which the script will write backup and log files. The default behaviour sets the working directory as cruft location.

\item[\abox{logfile}] This option, as the name indicates, sets the name of the log file generated by \rbox{latexindent} according to the provided value.

\item[\abox{output}] This option, as the name indicates, sets the name of the output file. Please note that this option has higher priority over some switches, so options like \abox{overwrite} will be ignored by the underlying script.

\item[\abox{options}] This option, as the name indicates, takes a list of raw command line options and appends it to the actual script call. An error is thrown if any data structure other than a proper list is provided as the value.

\item[\abox{settings}] This option, as the name indicates, dictates the indentation settings to be applied in the current script execution. Two possible values are available:

\begin{description}
\item[\povalue{local}] This value, as the name implies, acts a switch to indicate a local configuration. In this scenario, the script will look for a proper settings file in the same directory as the \mtbox{currentFile} reference and add the corresponding content to the indentation scheme. Optionally, a file location can be specified as well. Please refer to the \abox{where} option for more details on such feature.

\item[\povalue{onlydefault}] This value, as the name indicates, ignores any local configuration, so the script will resort to the default indentation behaviour.
\end{description}

\item[\abox{where}] This option, as the name indicates, sets the file location containing the indentation settings according to the provided value. This option can only be used if, and only if, \rbox[cyan]{local} is set as the value for the \abox{settings} option, otherwise the rule will throw an error.

\item[\abox{replacement}] This option, as the name indicates, implements the replacement mode switches. Three possible values are available:

\begin{description}
\item[\povalue{full}] This value, as the name indicates, performs indentation and replacements, not respecting verbatim code blocks.

\item[\povalue{noverb}] This value, as the name indicates, performs indentation and replacements, and will respect verbatim code blocks.

\item[\povalue{noindent}] This value, as the name implies, will not perform indentation, and will perform replacements not respecting verbatim code blocks.
\end{description}

\item[\abox{check}] This option, as the name indicates, checks whether the text after indentation matches that given in the original file. Two possible values are available:

\begin{description}
\item[\povalue{standard}] This value, as the name indicates, refers to the standard behaviour. Please refer to the documentation for further details.

\item[\povalue{verbose}] This value, as the name indicates, refers to including more details to the check. The tool will print the result in the standard output, as well as in its own log file. Please refer to the documentation for further details.
\end{description}

\item[\abox{lines}] This option instructs the tool to operate only on specific line ranges within the file being inspected. Please refer to the documentation for further details.
\end{description}

\begin{codebox}{Example}{teal}{\icnote}{white}
% arara: indent: { overwrite: yes }
\end{codebox}

\item[\rulebox{knitr}]
This rule calls the \rbox{knitr} package, a transparent engine for dynamic report generation with R. It takes an \rbox{.Rnw} file as input, extracts the R code in it according to a list of patterns, evaluates the code and writes the output in another file. It can also tangle R source code from the input document.

\begin{description}
\item[\rpbox{output}{NULL}] This option sets the output file. when absent, \rbox{knitr} will try to guess a default, which will be under the current working directory.

\item[\rpsbox{tangle}] This option sets whether to tangle the R code from the input file. Note that, when used, this option requires \abox{output} to be specified as well, otherwise an error is thrown.

\item[\rpsbox{quiet}] This option, as the name indicates, sets whether the tool should suppress both progress bar and messages.

\item[\rpbox{envir}{parent.frame()}] This option sets the environment in which code chunks are to be evaluated. Please refer to the documentation for further details.

\item[\rpbox{encoding}{getOption("encoding")}] This option, as the name indicates, sets the encoding of the input file. Please refer to the documentation for further details.
\end{description}

\begin{codebox}{Example}{teal}{\icnote}{white}
% arara: knitr: { quiet: yes }
\end{codebox}

\item[\rulebox{latex}]
This rule runs the \rbox{latex} \TeX\ engine on the provided \mtbox{currentFile} reference, generating a corresponding file in a device independent format.

\begin{description}
\item[\rpbox{branch}{stable}] This option allows branching formats for the current engine, mainly focused on package development. Users of current \TeX\ distributions might benefit from format branching in order to easily test documents and code against the upcoming releases. Possible values are:

\begin{description}
\item[\povalue{stable}] This value, as the name implies, enables the stable engine format branch. Note that this is the default format.

\item[\povalue{developer}] For experienced users, this value enables the experimental, developer engine format branch.
\end{description}

\item[\abox{interaction}] This option alters the underlying engine behaviour. If this option is omitted, \TeX\ will prompt the user for interaction in the event of an error. Possible values are, in order of increasing user interaction (courtesy of our master Enrico Gregorio):

\begin{description}
\item[\povalue{batchmode}] In this mode, nothing is printed on the terminal, and errors are scrolled as if the \rbox{return} key is hit at every error. Missing files that \TeX\ tries to input or request from keyboard input cause the job to abort.

\item[\povalue{nonstopmode}] In this mode, the diagnostic message will appear on the terminal, but there is no possibility of user interaction just like in batch mode, previously described.

\item[\povalue{scrollmode}] In this mode, as the name indicates, \TeX\ will stop only for missing files to input or if proper keyboard input is necessary. \TeX\ fixes errors itself.

\item[\povalue{errorstopmode}] In this mode, \TeX\ will stop at each error, asking for proper user intervention. This is the most user interactive mode available.
\end{description}

\item[\rpsbox{shell}] This option sets whether the possibility of running underlying system commands from within \TeX\ is activated.

\item[\rpsbox{synctex}] This option sets whether \rbox{synctex}, an input and output synchronization feature that allows navigation from source to typeset material and vice versa, available in most \TeX\ engines, is activated.

\item[\rpsbox{draft}] This option sets whether the draft mode, i.e, a mode that produces no output, so the engine can check the syntax, is activated.

\item[\abox{options}] This option, as the name indicates, takes a list of raw command line options and appends it to the actual system call. An error is thrown if any data structure other than a proper list is provided as the value.
\end{description}

\begin{codebox}{Example}{teal}{\icnote}{white}
% arara: latex: { interaction: scrollmode, draft: yes }
\end{codebox}

\item[\rulebox{latexmk}]
This rule runs \rbox{latexmk}, a fantastic command line tool for fully automated \TeX\ document generation, on the provided \mtbox{currentFile} reference.

\begin{description}
\item[\abox{clean}] This option, as the name indicates, removes all temporary files generated after a sequence of intermediate calls for document generation. Two possible values are available:

\begin{description}
\item[\povalue{all}] This value, as the name indicates, removes all temporary, intermediate files, as well as resulting, final formats such as PostScript and Portable Document Format. Only relevant source files are kept.

\item[\povalue{partial}] This value, as the name indicates, removes all temporary, intermediate files and keeps the resulting, final formats such as PostScript and Portable Document Format.
\end{description}

\item[\abox{engine}] This option, as the name indicates, sets the underlying \TeX\ engine of \rbox{latexmk} to be used for the compilation sequence. Possible values are:

\begin{description}
\item[\povalue{latex}] This value, as the name indicates, sets the underlying \TeX\ engine of the script to \rbox{latex} for the compilation sequence.

\item[\povalue{latex-dev}] This value, as the name indicates, sets the underlying \TeX\ engine of the script to \rbox{latex-dev} (the development branch) for the compilation sequence.

\item[\povalue{pdflatex}] This value, as the name indicates, sets the underlying \TeX\ engine of the script to \rbox{pdflatex} for the compilation sequence.

\item[\povalue{pdflatex-dev}] This value, as the name indicates, sets the underlying \TeX\ engine of the script to \rbox{pdflatex-dev} (the development branch) for the compilation sequence.

\item[\povalue{xelatex}] This value, as the name indicates, sets the underlying \TeX\ engine of the script to \rbox{xelatex} for the compilation sequence.

\item[\povalue{xelatex-dev}] This value, as the name indicates, sets the underlying \TeX\ engine of the script to \rbox{xelatex-dev} (the development branch) for the compilation sequence.

\item[\povalue{lualatex}] This value, as the name indicates, sets the underlying \TeX\ engine of the script to \rbox{lualatex} for the compilation sequence.

\item[\povalue{lualatex-dev}] This value, as the name indicates, sets the underlying \TeX\ engine of the script to \rbox{lualatex-dev} (the development branch) for the compilation sequence.
\end{description}

\item[\abox{program}] This option, as the name suggests, sets the \TeX\ engine according to the provided value. It is important to note that this option has higher priority over \abox{engine} values, so the latter will be discarded.

\item[\abox{options}] This option, as the name indicates, takes a list of raw command line options and appends it to the actual script call. An error is thrown if any data structure other than a proper list is provided as the value.
\end{description}

\begin{codebox}{Example}{teal}{\icnote}{white}
% arara: latexmk: { engine: pdflatex }
\end{codebox}

\item[\rulebox{llmk}]
This rule runs \rbox{llmk}, a command line tool specific for building \LaTeX\ documents. The tool's aim is to provide a simple way to specify a workflow of processing documents and encourage people to always explicitly show the right workflow for each document.

\begin{description}
\item[\abox{clean}] This option, as the name indicates, removes all temporary files generated after a sequence of intermediate calls for document generation. Two possible values are available:

\begin{description}
\item[\povalue{all}] This value, as the name indicates, removes all temporary, intermediate files, as well as resulting, final formats such as PostScript and Portable Document Format. Only relevant source files are kept.

\item[\povalue{partial}] This value, as the name indicates, removes all temporary, intermediate files and keeps the resulting, final formats such as PostScript and Portable Document Format.
\end{description}

\item[\abox{debug}] This option activates the specified debug category, so debugging messages related to the activated category will be shown. Please refer to the documentation for more details.

\item[\rpsbox{dry}] This option sets whether the tool should display a list of commands to be executed without actually invoking them.

\item[\abox{mode}]
This option sets the verbosity level of messages to be displayed during a run. Three possible values are available:

\begin{description}
\item[\povalue{quiet}] This value, as the name indicates, suppresses most of the messages from the program during execution.

\item[\povalue{silent}] This value, as the name indicates, silences messages from invoked programs by redirecting both standard output and standard error streams to the null device.

\item[\povalue{verbose}] This value, as the name indicates, displays additional information such as invoked commands with options and arguments by the program.
\end{description}
\end{description}

\begin{codebox}{Example}{teal}{\icnote}{white}
% arara: llmk: { mode: verbose }
\end{codebox}

\item[\rulebox{ltx2any}]
This rule runs \rbox{ltx2any}, a command line tool written in Ruby that acts as a \LaTeX\ build wrapper, on the provided \mtbox{currentFile} reference.

\begin{description}
\item[\rpsbox{clean}] This option, as the name indicates, sets whether all intermediate results generated during the compilation to be deleted.

\item[\abox{engine}] This option, as the name indicates, sets the engine to be using during the current execution.

\item[\abox{parameters}] This option, as the name indicates, takes a list of parameters to be passed to the engine. An error is thrown if any data structure other than a proper list is provided as the value.

\item[\abox{tikzimages}] This option takes a list of externalised Ti\textit{k}Z images to rebuild. An error is thrown if any data structure other than a proper list is provided as the value.

\item[\abox{jobname}] This option, as the name indicates, sets the job name to be used in the resulting file.

\item[\abox{logname}] This option, as the name indicates, sets the log file name to be used during the current execution.

\item[\abox{logformat}] This option, as the name indicates, sets the log format to be used during the current execution. Three possible values are available:

\begin{description}
\item[\povalue{raw}] This value, as the name indicates, sets the log format to be raw, i.e, as generated by the underlying engines.

\item[\povalue{markdown}] This value, as the name indicates, sets the log format to be displayed in Markdown.

\item[\povalue{pdf}] This value, as the name indicates, sets the log format to be displayed in the Portable Document Format.
\end{description}

\item[\abox{loglevel}] This option, as the name indicates, sets the log level to be used during the current execution. Three possible values are available:

\begin{description}
\item[\povalue{error}] This value, as the name indicates, sets the base log level to report errors only. No other information is appended.

\item[\povalue{warning}] This value, as the name indicates, sets the base log level to report warnings and errors. No other information is appended.

\item[\povalue{info}] This value, as the name indicates, set the base log level to report all information available, regardless of message categories.
\end{description}

\item[\abox{frequency}] This option, as the name indicates, sets how often the engine runs. Values smaller than one will cause it to run until the resulting file no longer changes.

\item[\abox{directory}] This option, as the name indicates, sets the directory to hold intermediate files during the compilation.

\item[\rpsbox{synctex}] This option sets whether \rbox{synctex}, an input and output synchronization feature that allows navigation from source to typeset material and vice versa, available in most \TeX\ engines, is activated.

\item[\abox{options}] This option, as the name indicates, takes a list of raw command line options and appends it to the actual system call. An error is thrown if any data structure other than a proper list is provided as the value.
\end{description}

\begin{codebox}{Example}{teal}{\icnote}{white}
% arara: ltx2any: { synctex: yes }
\end{codebox}

\item[\rulebox{luahbtex}]
This rule runs the \rbox{luahbtex} \TeX\ engine on the provided \mtbox{currentFile} reference, generating a corresponding file in the Portable Document Format, as expected.

\begin{description}
\item[\abox{interaction}] This option alters the underlying engine behaviour. If this option is omitted, \TeX\ will prompt the user for interaction in the event of an error. Possible values are, in order of increasing user interaction (courtesy of our master Enrico Gregorio):

\begin{description}
\item[\povalue{batchmode}] In this mode, nothing is printed on the terminal, and errors are scrolled as if the \rbox{return} key is hit at every error. Missing files that \TeX\ tries to input or request from keyboard input cause the job to abort.

\item[\povalue{nonstopmode}] In this mode, the diagnostic message will appear on the terminal, but there is no possibility of user interaction just like in batch mode, previously described.

\item[\povalue{scrollmode}] In this mode, as the name indicates, \TeX\ will stop only for missing files to input or if proper keyboard input is necessary. \TeX\ fixes errors itself.

\item[\povalue{errorstopmode}] In this mode, \TeX\ will stop at each error, asking for proper user intervention. This is the most user interactive mode available.
\end{description}

\item[\rpsbox{shell}] This option sets whether the possibility of running underlying system commands from within \TeX\ is activated.

\item[\rpsbox{synctex}] This option sets whether \rbox{synctex}, an input and output synchronization feature that allows navigation from source to typeset material and vice versa, available in most \TeX\ engines, is activated.

\item[\rpsbox{draft}] This option sets whether the draft mode, i.e, a mode that produces no output, so the engine can check the syntax, is activated.

\item[\abox{options}] This option, as the name indicates, takes a list of raw command line options and appends it to the actual system call. An error is thrown if any data structure other than a proper list is provided as the value.
\end{description}

\begin{codebox}{Example}{teal}{\icnote}{white}
% arara: luahbtex: { interaction: batchmode,
% arara: --> shell: yes, draft: yes }
\end{codebox}

\item[\rulebox{lualatex}]
This rule runs the new \rbox{lualatex} \TeX\ engine on the provided \mtbox{currentFile} reference, generating a corresponding file in the Portable Document Format, as expected.

\begin{description}
\item[\rpbox{branch}{stable}] This option allows branching formats for the current engine, mainly focused on package development. Users of current \TeX\ distributions might benefit from format branching in order to easily test documents and code against the upcoming releases. Possible values are:

\begin{description}
\item[\povalue{stable}] This value, as the name implies, enables the stable engine format branch. Note that this is the default format.

\item[\povalue{developer}] For experienced users, this value enables the experimental, developer engine format branch.
\end{description}

\item[\abox{interaction}] This option alters the underlying engine behaviour. If this option is omitted, \TeX\ will prompt the user for interaction in the event of an error. Possible values are, in order of increasing user interaction (courtesy of our master Enrico Gregorio):

\begin{description}
\item[\povalue{batchmode}] In this mode, nothing is printed on the terminal, and errors are scrolled as if the \rbox{return} key is hit at every error. Missing files that \TeX\ tries to input or request from keyboard input cause the job to abort.

\item[\povalue{nonstopmode}] In this mode, the diagnostic message will appear on the terminal, but there is no possibility of user interaction just like in batch mode, previously described.

\item[\povalue{scrollmode}] In this mode, as the name indicates, \TeX\ will stop only for missing files to input or if proper keyboard input is necessary. \TeX\ fixes errors itself.

\item[\povalue{errorstopmode}] In this mode, \TeX\ will stop at each error, asking for proper user intervention. This is the most user interactive mode available.
\end{description}

\item[\rpsbox{shell}] This option sets whether the possibility of running underlying system commands from within \TeX\ is activated.

\item[\rpsbox{synctex}] This option sets whether \rbox{synctex}, an input and output synchronization feature that allows navigation from source to typeset material and vice versa, available in most \TeX\ engines, is activated.

\item[\rpsbox{draft}] This option sets whether the draft mode, i.e, a mode that produces no output, so the engine can check the syntax, is activated.

\item[\abox{options}] This option, as the name indicates, takes a list of raw command line options and appends it to the actual system call. An error is thrown if any data structure other than a proper list is provided as the value.
\end{description}

\begin{codebox}{Example}{teal}{\icnote}{white}
% arara: lualatex: { interaction: errorstopmode,
% arara: --> synctex: yes }
\end{codebox}

\item[\rulebox{luatex}]
This rule runs the \rbox{luatex} \TeX\ engine on the provided \mtbox{currentFile} reference, generating a corresponding file in the Portable Document Format, as expected.

\begin{description}
\item[\abox{interaction}] This option alters the underlying engine behaviour. If this option is omitted, \TeX\ will prompt the user for interaction in the event of an error. Possible values are, in order of increasing user interaction (courtesy of our master Enrico Gregorio):

\begin{description}
\item[\povalue{batchmode}] In this mode, nothing is printed on the terminal, and errors are scrolled as if the \rbox{return} key is hit at every error. Missing files that \TeX\ tries to input or request from keyboard input cause the job to abort.

\item[\povalue{nonstopmode}] In this mode, the diagnostic message will appear on the terminal, but there is no possibility of user interaction just like in batch mode, previously described.

\item[\povalue{scrollmode}] In this mode, as the name indicates, \TeX\ will stop only for missing files to input or if proper keyboard input is necessary. \TeX\ fixes errors itself.

\item[\povalue{errorstopmode}] In this mode, \TeX\ will stop at each error, asking for proper user intervention. This is the most user interactive mode available.
\end{description}

\item[\rpsbox{shell}] This option sets whether the possibility of running underlying system commands from within \TeX\ is activated.

\item[\rpsbox{synctex}] This option sets whether \rbox{synctex}, an input and output synchronization feature that allows navigation from source to typeset material and vice versa, available in most \TeX\ engines, is activated.

\item[\rpsbox{draft}] This option sets whether the draft mode, i.e, a mode that produces no output, so the engine can check the syntax, is activated.

\item[\abox{options}] This option, as the name indicates, takes a list of raw command line options and appends it to the actual system call. An error is thrown if any data structure other than a proper list is provided as the value.
\end{description}

\begin{codebox}{Example}{teal}{\icnote}{white}
% arara: luatex: { interaction: batchmode,
% arara: --> shell: yes, draft: yes }
\end{codebox}

\item[\rulebox{make}]
This rule runs \rbox{make}, a build automation tool that automatically builds executable programs and libraries from source code, according to a special file which specifies how to derive the target program.

\begin{description}
\item[\abox{targets}] This option takes a list of targets. Note that \rbox{make} updates a target if it depends on files that have been modified since the target was last modified, or if the target does not exist.

\item[\abox{options}] This option, as the name indicates, takes a list of raw command line options and appends it to the actual system call. An error is thrown if any data structure other than a proper list is provided as the value.
\end{description}

\begin{codebox}{Example}{teal}{\icnote}{white}
% arara: make: { targets: [ compile, package ] }
\end{codebox}

\item[\rulebox{makeglossaries}]
This rule runs \rbox{makeglossaries}, an efficient Perl script designed for use with \TeX\ documents that work with the \rbox{glossaries} package. All the information required to be passed to the relevant indexing application should also be contained in the auxiliary file. The script takes the corresponding base name of the \mtbox{currentFile} reference (i.e, the name without the associated extension) as the mandatory argument.

\begin{description}
\item[\abox{options}] This option, as the name indicates, takes a list of raw command line options and appends it to the actual script call. An error is thrown if any data structure other than a proper list is provided as the value.
\end{description}

\begin{codebox}{Example}{teal}{\icnote}{white}
% arara: makeglossaries if found('aux', '@istfilename')
\end{codebox}

\item[\rulebox{makeglossarieslite}]
This rule runs \rbox{makeglossaries-lite}, a lightweight Lua script designed for use with \TeX\ documents that work with the \rbox{glossaries} package. All the information required to be passed to the relevant indexing application should also be contained in the auxiliary file. The script takes the corresponding base name of the \mtbox{currentFile} reference (i.e, the name without the associated extension) as the mandatory argument.

\begin{description}
\item[\abox{options}] This option, as the name indicates, takes a list of raw command line options and appends it to the actual script call. An error is thrown if any data structure other than a proper list is provided as the value.
\end{description}

\begin{codebox}{Example}{teal}{\icnote}{white}
% arara: makeglossarieslite if found('aux', '@istfilename')
\end{codebox}

\item[\rulebox{makeindex}]
This rule runs \rbox{makeindex}, a general purpose hierarchical index generator, on the corresponding base name of the \mtbox{currentFile} reference (i.e, the name without the associated extension) as a string concatenated with the \rbox{idx} suffix, generating an index as a special \rbox{ind} file.

\begin{description}
\item[\abox{style}] This option, as the name indicates, sets the underlying index style file. Make sure to provide a valid \rbox{ist} file when using this option.

\item[\rpsbox{german}] This option, as the name indicates, sets whether German word ordering should be used when generating the index, according to the rules set forth in DIN 5007.

\item[\abox{order}] This option, as the name indicates, sets the default ordering scheme for the \rbox{makeindex} program. Two possible values are available:

\begin{description}
\item[\povalue{letter}] This value, as the name indicates, activates the letter ordering scheme. In such scheme, a blank space does not precede any letter in the alphabet.

\item[\povalue{word}] This value, as the name indicates, activates the word ordering scheme. In such scheme, a blank space precedes any letter in the alphabet.
\end{description}

\item[\rpbox{input}{idx}] This option, as the name indicates, sets the default extension for the input file, according to the provided value. Later, this value will be concatenated as a suffix for the base name of the \mtbox{currentFile} reference (i.e, the name without the associated extension).

\item[\rpbox{output}{ind}] This option, as the name indicates, sets the default extension for the output file, according to the provided value. Later, this value will be concatenated as a suffix for the base name of the \mtbox{currentFile} reference (i.e, the name without the associated extension).

\item[\rpbox{log}{ilg}] This option, as the name indicates, sets the default extension for the log file, according to the provided value. Later, this value will be concatenated as a suffix for the base name of the \mtbox{currentFile} reference (i.e, the name without the associated extension).

\item[\abox{options}] This option, as the name indicates, takes a list of raw command line options and appends it to the actual system call. An error is thrown if any data structure other than a proper list is provided as the value.
\end{description}

\begin{codebox}{Example}{teal}{\icnote}{white}
% arara: makeindex: { style: book.ist }
\end{codebox}

\item[\rulebox{metapost}]
This rule runs \rbox{metapost}, a tool to compile the Metapost graphics programming language. Please note that you will have to make the \rbox{.mp} extension known to \arara\ in order to compile Metapost files. Furthermore, it is advised to use this in your regular \TeX\ document specifying the \abox{files} parameter to include all graphics you want to compile for inclusion in your document.

\begin{description}
\item[\abox{interaction}] This option alters the underlying engine behaviour. If this option is omitted, \TeX\ will prompt the user for interaction in the event of an error. Possible values are, in order of increasing user interaction (courtesy of our master Enrico Gregorio):

\begin{description}
\item[\povalue{batchmode}] In this mode, nothing is printed on the terminal, and errors are scrolled as if the \rbox{return} key is hit at every error. Missing files that \TeX\ tries to input or request from keyboard input cause the job to abort.

\item[\povalue{nonstopmode}] In this mode, the diagnostic message will appear on the terminal, but there is no possibility of user interaction just like in batch mode, previously described.

\item[\povalue{scrollmode}] In this mode, as the name indicates, \TeX\ will stop only for missing files to input or if proper keyboard input is necessary. \TeX\ fixes errors itself.

\item[\povalue{errorstopmode}] In this mode, \TeX\ will stop at each error, asking for proper user intervention. This is the most user interactive mode available.
\end{description}

\item[\abox{numbersystem}] This option sets the number system Metapost will use for calculations.

\begin{description}
\item[\povalue{scaled}] In this mode, 32-bit fixed-point arithmetics is used.

\item[\povalue{double}] In this mode, IEEE floating-point arithmetics with 64 bits is used.
\item[\povalue{binary}] This mode is similary to \povalue{double} but without a fixed-length mantissa.
\item[\povalue{decimal}] In this mode, arbitrary precision arithmetics is used and numbers are internally represented in base 10.
\end{description}

\item[\abox{options}] This option, as the name indicates, takes a list of raw command line options and appends it to the actual system call. An error is thrown if any data structure other than a proper list is provided as the value.
\end{description}

\begin{codebox}{Example}{teal}{\icnote}{white}
% arara: metapost: { files: [ graphics.mp ] }
\end{codebox}

\item[\rulebox{move}]
This rule moves the \rbox{File} reference to the provided target using the underlying operating system move operation. The target is \emph{always} overwritten. However, be mindful that an error will be thrown if you try to move or overwrite the file referenced in \mtbox{getOriginalReference}.

\begin{description}
\item[\abox{target}~\rqbox] This option, as the name implies, specifies the target for the move operation. Keep in mind that this option is required.
\end{description}

\begin{codebox}{Example}{teal}{\icnote}{white}
% arara: move: { files: [ 'thesis.pdf' ],
% arara: --> target: 'backup/thesis.pdf' }
\end{codebox}

\item[\rulebox{nomencl}]
This rule runs \rbox{makeindex} in order to automatically generate a nomenclature list from \TeX\ documents that work with the \rbox{nomencl} package. The program itself is a general purpose hierarchical index generator and takes the corresponding base name of the \mtbox{currentFile} reference (i.e, the name without the associated extension) as a string concatenated with the \rbox{nlo} suffix and a special style file in order to generate the nomenclature list as a special \rbox{nls} file.

\begin{description}
\item[\rpbox{style}{nomencl.ist}] This option, as the name indicates, sets the underlying index style file. The default value is set to the one automatically provided by the \rbox{nomencl} package, so it is highly recommended to not override it.

\item[\abox{options}] This option, as the name indicates, takes a list of raw command line options and appends it to the actual system call. An error is thrown if any data structure other than a proper list is provided as the value.
\end{description}

\begin{codebox}{Example}{teal}{\icnote}{white}
% arara: nomencl
\end{codebox}

\item[\rulebox{pbibtex}]
This rule runs the \rbox{pbibtex} program, a reference management software, on the corresponding base name of the \mtbox{currentFile} reference (i.e, the name without the associated extension) as a string.

\begin{description}
\item[\abox{options}] This option, as the name indicates, takes a list of raw command line options and appends it to the actual system call. An error is thrown if any data structure other than a proper list is provided as the value.
\end{description}

\begin{codebox}{Example}{teal}{\icnote}{white}
% arara: pbibtex
% arara: --> if exists(toFile('references.bib'))
\end{codebox}

\item[\rulebox{pdfcrop}]
This rule runs \rbox{pdfcrop}, a command line utility to calculate and remove empty margins, on corresponding base name of the \mtbox{currentFile} reference (i.e, the name without the associated extension) as a string concatenated with the \rbox{pdf} suffix.

\begin{description}
\item[\abox{output}] This option, as the name indicates, sets the output file. When omitted, the tool uses the input base name with the \rbox{-crop.pdf} suffix.

\item[\rpsbox{verbose}] This option, as the name indicates, sets whether the command line tool will be executed in verbose mode.

\item[\rpsbox{debug}] This option, as the name indicates, sets whether the command line tool will be executed in debug mode.

\item[\abox{engine}] This option, as the name indicates, sets the underlying \TeX\ engine to be used during the run. Three possible values are available:

\begin{description}
\item[\povalue{pdftex}] This value, as the name indicates, sets \rbox{pdftex} as the underlying \TeX\ engine to be used during the run.

\item[\povalue{xetex}] This value, as the name indicates, sets \rbox{xetex} as the underlying \TeX\ engine to be used during the run.

\item[\povalue{luatex}] This value, as the name indicates, sets \rbox{luatex} as the underlying \TeX\ engine to be used during the run.
\end{description}

\item[\abox{margins}] This option, as the name indicates, takes a list of four elements denoting left, top, right and bottom margins, respectivelly. An error will be thrown if no list is provided or if the list does not contain exactly four elements.

\item[\rpsbox{clip}] This option, as the name indicates, sets whether the command line tool should include clipping support, if margins are set.

\item[\rpsbox{hires}] This option, as the name indicates, sets whether the command line tool should use a high resolution bounding box feature.

\item[\rpsbox{ini}] This option, as the name indicates, sets whether the \rbox{initex} variant of the underlying \TeX\ engine is used.

\item[\rpsbox{restricted}] This option, as the name indicates, sets whether the command line tool should run on restricted mode.

\item[\abox{papersize}] This option, as the name indicates, sets the paper size. According to the documentation, this option should only be used with older versions of \rbox{ghostscript}.

\item[\abox{resolution}] This option, as the name indicates, sets the resolution by forwarding the value to the underlying \rbox{ghostscript} call.

\item[\abox{bbox}] This option, as the name indicates, takes a list of four elements denoting left, bottom, right and top margins, respectivelly, to override bounding box values found by \rbox{ghostscript}. An error will be thrown if no list is provided or if the list does not contain exactly four elements.

\item[\rpsbox{uncompress}] This option, as the name indicates, sets whether the tool should generate an uncompressed Portable Document Format file, useful for debugging.

\item[\abox{options}] This option, as the name indicates, takes a list of raw command line options and appends it to the actual system call. An error is thrown if any data structure other than a proper list is provided as the value.
\end{description}

\begin{codebox}{Example}{teal}{\icnote}{white}
% arara: pdfcrop
\end{codebox}

\item[\rulebox{pdfcsplain}]
This rule runs the \rbox{pdfcsplain} \TeX\ engine, a conservative extension of Knuth's plain \TeX\ with direct processing characters and hyphenation patterns for Czech and Slovak, on the provided \mtbox{currentFile} reference.

\begin{description}
\item[\abox{interaction}] This option alters the underlying engine behaviour. If this option is omitted, \TeX\ will prompt the user for interaction in the event of an error. Possible values are, in order of increasing user interaction (courtesy of our master Enrico Gregorio):

\begin{description}
\item[\povalue{batchmode}] In this mode, nothing is printed on the terminal, and errors are scrolled as if the \rbox{return} key is hit at every error. Missing files that \TeX\ tries to input or request from keyboard input cause the job to abort.

\item[\povalue{nonstopmode}] In this mode, the diagnostic message will appear on the terminal, but there is no possibility of user interaction just like in batch mode, previously described.

\item[\povalue{scrollmode}] In this mode, as the name indicates, \TeX\ will stop only for missing files to input or if proper keyboard input is necessary. \TeX\ fixes errors itself.

\item[\povalue{errorstopmode}] In this mode, \TeX\ will stop at each error, asking for proper user intervention. This is the most user interactive mode available.
\end{description}

\item[\rpsbox{shell}] This option sets whether the possibility of running underlying system commands from within \TeX\ is activated.

\item[\rpsbox{synctex}] This option sets whether \rbox{synctex}, an input and output synchronization feature that allows navigation from source to typeset material and vice versa, available in most \TeX\ engines, is activated.

\item[\rpsbox{draft}] This option sets whether the draft mode, i.e, a mode that produces no output, so the engine can check the syntax, is activated.

\item[\abox{options}] This option, as the name indicates, takes a list of raw command line options and appends it to the actual system call. An error is thrown if any data structure other than a proper list is provided as the value.
\end{description}

\begin{codebox}{Example}{teal}{\icnote}{white}
% arara: pdfcsplain: { shell: yes, synctex: yes }
\end{codebox}

\item[\rulebox{pdflatex}]
This rule runs the \rbox{pdflatex} \TeX\ engine on the provided \mtbox{currentFile} reference, generating a corresponding file in the Portable Document Format, as expected.

\begin{description}
\item[\rpbox{branch}{stable}] This option allows branching formats for the current engine, mainly focused on package development. Users of current \TeX\ distributions might benefit from format branching in order to easily test documents and code against the upcoming releases. Possible values are:

\begin{description}
\item[\povalue{stable}] This value, as the name implies, enables the stable engine format branch. Note that this is the default format.

\item[\povalue{developer}] For experienced users, this value enables the experimental, developer engine format branch.
\end{description}

\item[\abox{interaction}] This option alters the underlying engine behaviour. If this option is omitted, \TeX\ will prompt the user for interaction in the event of an error. Possible values are, in order of increasing user interaction (courtesy of our master Enrico Gregorio):

\begin{description}
\item[\povalue{batchmode}] In this mode, nothing is printed on the terminal, and errors are scrolled as if the \rbox{return} key is hit at every error. Missing files that \TeX\ tries to input or request from keyboard input cause the job to abort.

\item[\povalue{nonstopmode}] In this mode, the diagnostic message will appear on the terminal, but there is no possibility of user interaction just like in batch mode, previously described.

\item[\povalue{scrollmode}] In this mode, as the name indicates, \TeX\ will stop only for missing files to input or if proper keyboard input is necessary. \TeX\ fixes errors itself.

\item[\povalue{errorstopmode}] In this mode, \TeX\ will stop at each error, asking for proper user intervention. This is the most user interactive mode available.
\end{description}

\item[\rpsbox{shell}] This option sets whether the possibility of running underlying system commands from within \TeX\ is activated.

\item[\rpsbox{synctex}] This option sets whether \rbox{synctex}, an input and output synchronization feature that allows navigation from source to typeset material and vice versa, available in most \TeX\ engines, is activated.

\item[\rpsbox{draft}] This option sets whether the draft mode, i.e, a mode that produces no output, so the engine can check the syntax, is activated.

\item[\abox{options}] This option, as the name indicates, takes a list of raw command line options and appends it to the actual system call. An error is thrown if any data structure other than a proper list is provided as the value.
\end{description}

\begin{codebox}{Example}{teal}{\icnote}{white}
% arara: pdflatex: { interaction: batchmode }
% arara: --> if missing('pdf') || changed('tex')
\end{codebox}

\item[\rulebox{pdftex}]
This rule runs the \rbox{pdftex} \TeX\ engine on the provided \mtbox{currentFile} reference, generating a corresponding file in the Portable Document Format, as expected.

\begin{description}
\item[\abox{interaction}] This option alters the underlying engine behaviour. If this option is omitted, \TeX\ will prompt the user for interaction in the event of an error. Possible values are, in order of increasing user interaction (courtesy of our master Enrico Gregorio):

\begin{description}
\item[\povalue{batchmode}] In this mode, nothing is printed on the terminal, and errors are scrolled as if the \rbox{return} key is hit at every error. Missing files that \TeX\ tries to input or request from keyboard input cause the job to abort.

\item[\povalue{nonstopmode}] In this mode, the diagnostic message will appear on the terminal, but there is no possibility of user interaction just like in batch mode, previously described.

\item[\povalue{scrollmode}] In this mode, as the name indicates, \TeX\ will stop only for missing files to input or if proper keyboard input is necessary. \TeX\ fixes errors itself.

\item[\povalue{errorstopmode}] In this mode, \TeX\ will stop at each error, asking for proper user intervention. This is the most user interactive mode available.
\end{description}

\item[\rpsbox{shell}] This option sets whether the possibility of running underlying system commands from within \TeX\ is activated.

\item[\rpsbox{synctex}] This option sets whether \rbox{synctex}, an input and output synchronization feature that allows navigation from source to typeset material and vice versa, available in most \TeX\ engines, is activated.

\item[\rpsbox{draft}] This option sets whether the draft mode, i.e, a mode that produces no output, so the engine can check the syntax, is activated.

\item[\abox{options}] This option, as the name indicates, takes a list of raw command line options and appends it to the actual system call. An error is thrown if any data structure other than a proper list is provided as the value.
\end{description}

\begin{codebox}{Example}{teal}{\icnote}{white}
% arara: pdftex: { draft: yes }
\end{codebox}

\item[\rulebox{pdftk}]
This rule runs \rbox{pdftk}, a command line tool for manipulating Portable Document Format documents, on the corresponding base name of the \mtbox{currentFile} reference (i.e, the name without the associated extension) as a string concatenated with the \rbox{pdf} suffix.

\begin{description}
\item[\abox{options}] This option, as the name indicates, takes a list of raw command line options and appends it to the actual system call. An error is thrown if any data structure other than a proper list is provided as the value.
\end{description}

\begin{codebox}{Example}{teal}{\icnote}{white}
% arara: pdftk: { options: [ burst ] }
\end{codebox}

\item[\rulebox{perltex}]
This rule runs \rbox{perltex}, a wrapper that enables a symbiosis between Perl, a popular general purpose programming language, and a \TeX\ engine, on the provided \mtbox{currentFile} reference.

\begin{description}
\item[\rpbox{engine}{latex}] This option, as the name indicates, sets the underlying \TeX\ engine to be used for the current compilation. Make sure to take a look at the manual for further details on this option. Possible values are:

\begin{description}
\item[\povalue{latex}] This value, as the name suggests, sets the underlying \TeX\ engine to \rbox{latex} for the current compilation. Note that the engine might play a major role in the generated code.

\item[\povalue{pdflatex}] This value, as the name indicates, sets the underlying \TeX\ engine to \rbox{pdflatex} for the current compilation. Note that the engine might play a major role in the generated code.

\item[\povalue{xelatex}] This value, as the name suggests, sets the underlying \TeX\ engine to \rbox{xelatex} for the current compilation. Note that the engine might play a major role in the generated code.

\item[\povalue{lualatex}] This value, as the name indicates, sets the underlying \TeX\ engine to \rbox{lualatex} for the current compilation. Note that the engine might play a major role in the generated code.

\item[\povalue{tex}] This value, as the name suggests, sets the underlying \TeX\ engine to \rbox{tex} for the current compilation. Note that the engine might play a major role in the generated code.

\item[\povalue{pdftex}] This value, as the name indicates, sets the underlying \TeX\ engine to \rbox{pdftex} for the current compilation. Note that the engine might play a major role in the generated code.

\item[\povalue{luatex}] This value, as the name suggests, sets the underlying \TeX\ engine to \rbox{luatex} for the current compilation. Note that the engine might play a major role in the generated code.

\item[\povalue{context}] This value, as the name indicates, sets the underlying \TeX\ engine to \rbox{context} for the current compilation. Note that the engine might play a major role in the generated code.
\end{description}

\item[\rpsbox{safe}] This option sets whether the wrapper should enable sandboxing. When explicitly disabled, the wrapper might execute any arbitrary Perl code, including that which can harm files.

\item[\abox{permit}] This option takes a list of values in which indicate particular Perl operations to be performed, enabling finer-grained control over the wrapper sandbox.

\item[\rpsbox{standalone}]  This option generates a specific style file to make the document suitable for distribution to users who do not have the wrapper installed. Please refer to the manual for further details on this option.

\item[\abox{interaction}] This option alters the underlying engine behaviour. If this option is omitted, \TeX\ will prompt the user for interaction in the event of an error. Possible values are, in order of increasing user interaction (courtesy of our master Enrico Gregorio):

\begin{description}
\item[\povalue{batchmode}] In this mode, nothing is printed on the terminal, and errors are scrolled as if the \rbox{return} key is hit at every error. Missing files that \TeX\ tries to input or request from keyboard input cause the job to abort.

\item[\povalue{nonstopmode}] In this mode, the diagnostic message will appear on the terminal, but there is no possibility of user interaction just like in batch mode, previously described.

\item[\povalue{scrollmode}] In this mode, as the name indicates, \TeX\ will stop only for missing files to input or if proper keyboard input is necessary. \TeX\ fixes errors itself.

\item[\povalue{errorstopmode}] In this mode, \TeX\ will stop at each error, asking for proper user intervention. This is the most user interactive mode available.
\end{description}

\item[\rpsbox{shell}] This option sets whether the possibility of running underlying system commands from within \TeX\ is activated.

\item[\rpsbox{synctex}] This option sets whether \rbox{synctex}, an input and output synchronization feature that allows navigation from source to typeset material and vice versa, available in most \TeX\ engines, is activated.

\item[\abox{options}] This option, as the name indicates, takes a list of raw command line options and appends it to the actual system call. An error is thrown if any data structure other than a proper list is provided as the value.
\end{description}

\begin{codebox}{Example}{teal}{\icnote}{white}
% arara: perltex: { safe: no, standalone: yes }
\end{codebox}

\item[\rulebox{platex}]
This rule runs the \rbox{platex} \TeX\ engine on the provided \mtbox{currentFile} reference, generating a corresponding file in a device independent format.

\begin{description}
\item[\rpbox{branch}{stable}] This option allows branching formats for the current engine, mainly focused on package development. Users of current \TeX\ distributions might benefit from format branching in order to easily test documents and code against the upcoming releases. Possible values are:

\begin{description}
\item[\povalue{stable}] This value, as the name implies, enables the stable engine format branch. Note that this is the default format.

\item[\povalue{developer}] For experienced users, this value enables the experimental, developer engine format branch.
\end{description}

\item[\abox{interaction}] This option alters the underlying engine behaviour. If this option is omitted, \TeX\ will prompt the user for interaction in the event of an error. Possible values are, in order of increasing user interaction (courtesy of our master Enrico Gregorio):

\begin{description}
\item[\povalue{batchmode}] In this mode, nothing is printed on the terminal, and errors are scrolled as if the \rbox{return} key is hit at every error. Missing files that \TeX\ tries to input or request from keyboard input cause the job to abort.

\item[\povalue{nonstopmode}] In this mode, the diagnostic message will appear on the terminal, but there is no possibility of user interaction just like in batch mode, previously described.

\item[\povalue{scrollmode}] In this mode, as the name indicates, \TeX\ will stop only for missing files to input or if proper keyboard input is necessary. \TeX\ fixes errors itself.

\item[\povalue{errorstopmode}] In this mode, \TeX\ will stop at each error, asking for proper user intervention. This is the most user interactive mode available.
\end{description}

\item[\rpsbox{shell}] This option sets whether the possibility of running underlying system commands from within \TeX\ is activated.

\item[\rpsbox{synctex}] This option sets whether \rbox{synctex}, an input and output synchronization feature that allows navigation from source to typeset material and vice versa, available in most \TeX\ engines, is activated.

\item[\abox{options}] This option, as the name indicates, takes a list of raw command line options and appends it to the actual system call. An error is thrown if any data structure other than a proper list is provided as the value.
\end{description}

\begin{codebox}{Example}{teal}{\icnote}{white}
% arara: platex: { interaction: scrollmode, shell: yes }
\end{codebox}

\item[\rulebox{ps2pdf}]
This rule runs \rbox{ps2pdf}, a tool that converts PostScript to Portable Document Format, on the corresponding base name of the \mtbox{currentFile} reference (i.e, the name without the associated extension) as a string concatenated with the \rbox{ps} suffix.

\begin{description}
\item[\abox{output}] This option, as the name indicates, sets the output name for the generated \rbox{pdf} file. There is no need to provide an extension, as the value is always normalized with \mtbox{getBasename} such that only the name without the associated extension is used. The base name of the current file reference is used as the default value.

\item[\abox{options}] This option, as the name indicates, takes a list of raw command line options and appends it to the actual system call. An error is thrown if any data structure other than a proper list is provided as the value.
\end{description}

\begin{codebox}{Example}{teal}{\icnote}{white}
% arara: ps2pdf: { output: article }
\end{codebox}

\item[\rulebox{pythontex}]
This rule runs \rbox{pythontex}, a wrapper that provides access to Python from within typical \TeX\ documents, on the provided \mtbox{currentFile} reference. Make sure to take a look at the documentation for further details.

\begin{description}
\item[\abox{encoding}] This option sets the encoding of the underlying \TeX\ document and all related files. If an encoding is not specified, Unicode is assumed.

\item[\rpsbox{errorcode}] This option determines whether an exit code of 1 is returned if there were errors. On by default, but can be turned off since it is undesirable when working with some editors.

\item[\rpsbox{runall}] This option sets whether all code to be executed, regardless of modification. It is useful when code has not been modified, but a dependency such as a library or external data has changed.

\item[\abox{rerun}] This option, as the name indicates, sets the underlying threshold for reexecuting code. By default, the wrapper will rerun code that has been modified or that produced errors  on the last run. Possible values are:

\begin{description}
\item[\povalue{never}] When this value is used, the wrapper never executes code. In this scenario, a warning is issued if there is modified code. Please refer to the documentation for further details.

\item[\povalue{modified}] When this value is used, as the name indicates, the wrapper only executes code that has been modified or that has modified dependencies.

\item[\povalue{errors}] When this value is used, as the name indicates, the wrapper executes code that has been modified as well as code that produced errors on the last run.

\item[\povalue{warnings}] When this value is used, as the name indicates, the wrapper executes code that has been modified as well as code that produced errors or warnings on the last run.

\item[\povalue{always}] When this value is used, as the name indicates, the wrapper executes all code, regardless of modification or errors and warnings. It is useful when code has not been modified, but a dependency such as a library or external data has changed.
\end{description}

\item[\rpsbox{hashdependencies}] This option, as the name suggests, determines whether dependencies are checked for changes via their hashes or modification times. 

\item[\abox{jobs}] This option, as the name suggests, takes an integer value denoting the maximum number of concurrent processes. By default, the wrapper relies on the number of CPUs in the system.

\item[\rpsbox{verbose}] This option sets whether the wrapper should be executed in verbose mode, providing more output information, including a list of all processes that are launched.

\item[\abox{options}] This option, as the name indicates, takes a list of raw command line options and appends it to the actual system call. An error is thrown if any data structure other than a proper list is provided as the value.
\end{description}

\begin{codebox}{Example}{teal}{\icnote}{white}
% arara: pythontex: { jobs: 2, verbose: yes }
\end{codebox}

\item[\rulebox{qpdf}]
This rule runs \rbox{qpdf}, a command line application that does structural, content-preserving transformations of Portable Document Format files, as well as providing capabilities to developers.

\begin{description}
\item[\abox{options}~\rqbox] This option, as the name indicates, takes a list of raw command line options and appends it to the actual script call. An error is thrown if any data structure other than a proper list is provided as the value.
\end{description}

\begin{codebox}{Example}{teal}{\icnote}{white}
% arara: qpdf: { options: [ '--linearize', 'input.pdf',
% arara: --> 'output.pdf' ] }
\end{codebox}

\item[\rulebox{sage}]
This rule runs \rbox{sage}, a free open source mathematics software system, on the corresponding base name of the \mtbox{currentFile} reference (i.e, the name without the associated extension) as a string concatenated with the \rbox{sage} extension (which can be overriden).

\begin{description}
\item[\rpbox{program}{sage}] This option, as the name indicates, sets the program name. If the tool is not directly available in your system path, make sure to use the full path to the installed \rbox{sage} binary.

\item[\rpbox{extension}{sage}] This option, as the name indicates, sets the default extension to the input file to be processed by \rbox{sage}. Three possible values are available:

\begin{description}
\item[\povalue{sage}] This value, as the name indicates, sets the extension to refer to the Sage format, the default one used by the software system.

\item[\povalue{py}] This value, as the name indicates, sets the extension to refer to a typical Python source code.

\item[\povalue{spyx}] This value, as the name indicates, sets the extension to refer to the SPYX format, a specific Sage compiled source code.
\end{description}

\item[\abox{command}] This option, as the name indicates, forwards the provided value to the command line utility to be evaluated as a Sage code.

\item[\rpsbox{dotsage}] This option, as the name indicates, sets whether the tool should consider using the \rbox{.sage} directory in the user home directory or a temporary one.

\item[\abox{options}] This option, as the name indicates, takes a list of raw command line options and appends it to the actual system call. An error is thrown if any data structure other than a proper list is provided as the value.
\end{description}

\begin{codebox}{Example}{teal}{\icnote}{white}
% arara: sage
\end{codebox}

\item[\rulebox{sketch}]
This rule runs \rbox{sketch}, a system for producing line drawings of solid objects and scenes, on the corresponding base name of the \mtbox{currentFile} reference (i.e, the name without the associated extension) as a string concatenated with the \rbox{sk} suffix. Note that one needs to add support for this particular file type, as seen in Section~\ref{sec:basicstructure}, on page~\pageref{sec:basicstructure}.

\begin{description}
\item[\abox{options}] This option, as the name indicates, takes a list of raw command line options and appends it to the actual system call. An error is thrown if any data structure other than a proper list is provided as the value.
\end{description}

\begin{codebox}{Example}{teal}{\icnote}{white}
% arara: sketch
\end{codebox}

\item[\rulebox{songidx}]
This rule runs \rbox{songidx}, a song index generation script for the \rbox{songs} package, on the file reference provided as parameter, generating a proper index as a special \rbox{sbx} file. It is very important to observe that, at the time of writing, this script is not available off the shelf in \TeX\ Live or MiK\TeX\ distributions, so a manual deployment is required. The script execution is performed by the underlying \rbox{texlua} interpreter.

\begin{description}
\item[\abox{input}~\rqbox] This required option, as the name indicates, sets the input name for the song index file specified within the \TeX\ document. There is no need to provide an extension, as the value is always normalized with \mtbox{getBasename} such that only the name without the associated extension is used.

\item[\rpbox{script}{songidx.lua}] This option, as the name indicates, sets the script path. The default value is set to the script name, so either make sure \rbox{songidx.lua} is located in the same directory of your \TeX\ document or provide the correct location (preferably a full path).

\item[\abox{options}] This option, as the name indicates, takes a list of raw command line options and appends it to the actual script call. An error is thrown if any data structure other than a proper list is provided as the value.
\end{description}

\begin{codebox}{Example}{teal}{\icnote}{white}
% arara: songidx: { input: songs }
\end{codebox}

\item[\rulebox{spix}]
This rule runs \rbox{spix}, an interesting command line \TeX\ automation tool written in Python, on the provided \mtbox{currentFile} reference.

\begin{description}
\item[\abox{dry}] This option sets whether the tool should display a list of commands to be executed without actually invoking them.

\item[\abox{options}] This option, as the name indicates, takes a list of raw command line options and appends it to the actual script call. An error is thrown if any data structure other than a proper list is provided as the value.
\end{description}

\begin{codebox}{Example}{teal}{\icnote}{white}
% arara: spix
\end{codebox}

\item[\rulebox{tex}]
This rule runs the \rbox{tex} \TeX\ engine on the provided \mtbox{currentFile} reference, generating a corresponding file in a device independent format.

\begin{description}
\item[\abox{interaction}] This option alters the underlying engine behaviour. If this option is omitted, \TeX\ will prompt the user for interaction in the event of an error. Possible values are, in order of increasing user interaction (courtesy of our master Enrico Gregorio):

\begin{description}
\item[\povalue{batchmode}] In this mode, nothing is printed on the terminal, and errors are scrolled as if the \rbox{return} key is hit at every error. Missing files that \TeX\ tries to input or request from keyboard input cause the job to abort.

\item[\povalue{nonstopmode}] In this mode, the diagnostic message will appear on the terminal, but there is no possibility of user interaction just like in batch mode, previously described.

\item[\povalue{scrollmode}] In this mode, as the name indicates, \TeX\ will stop only for missing files to input or if proper keyboard input is necessary. \TeX\ fixes errors itself.

\item[\povalue{errorstopmode}] In this mode, \TeX\ will stop at each error, asking for proper user intervention. This is the most user interactive mode available.
\end{description}

\item[\rpsbox{shell}] This option sets whether the possibility of running underlying system commands from within \TeX\ is activated.

\item[\abox{options}] This option, as the name indicates, takes a list of raw command line options and appends it to the actual system call. An error is thrown if any data structure other than a proper list is provided as the value.
\end{description}

\begin{codebox}{Example}{teal}{\icnote}{white}
% arara: tex: { shell: yes }
\end{codebox}

\item[\rulebox{texcount}]
This rule runs \rbox{texcount}, a Perl script designed to count words in \TeX\ and \LaTeX\ files, ignoring macros, tables, formulae and more on the provided \mtbox{currentFile} reference. The script is highly configurable, so make sure to check the manual for further information. 

\begin{description}
\item[\abox{rules}] This option, as the name suggests, sets the rules which dictate how the script should work regarding word counting and option handling. Possible values are:

\begin{description}
\item[\povalue{relaxed}] This value, as the name indicates, sets a relaxed set of rules, allowing more general cases to be counted as either words and macros.

\item[\povalue{restricted}] This value, as the name indicates, sets a more restricted set of rules for word counting and option handling.
\end{description}

\item[\abox{verbosity}] This option, as the name suggests, sets the verbosity level of the script according to the provided integer value. Possible values are:

\begin{description}
\item[\povalue{\hphantom{x}0\hphantom{x}}] This value sets the lowest verbosity level of all, such that the script does not present parsing details.

\item[\povalue{\hphantom{x}1\hphantom{x}}] This value raises the details a bit and sets the verbosity level to include parsed words and marked formulae.

\item[\povalue{\hphantom{x}2\hphantom{x}}] This value adds more details from the previous verbosity level by including ignored text as well.

\item[\povalue{\hphantom{x}3\hphantom{x}}] This value adds more details from the previous verbosity level by including comments and options.

\item[\povalue{\hphantom{x}4\hphantom{x}}] This value sets the highest verbosity level of all, such that the script includes parsed worded, marked formulae, ignored text, comments, options and internal states.
\end{description}

\item[\rpsbox{strict}] This option sets whether the tool should enable strict mode, so certain groups for which rules are not defined raise warnings.

\item[\rpsbox{html}] This option, as the name suggests, defines whether the tool should output the report in the HTML format.

\item[\rpsbox{total}] This option, as the name suggests, defines whether the tool should provide a total sum instead of partial sums (per file).

\item[\rpsbox{unicode}] This option, as the name indicates, defines whether the tool should select Unicode as encoding for both input and output.

\item[\abox{output}] This option, as the name suggests, sets the output file name in which the report will be written.

\item[\abox{options}] This option, as the name indicates, takes a list of raw command line options and appends it to the actual system call. An error is thrown if any data structure other than a proper list is provided as the value.
\end{description}

\begin{codebox}{Example}{teal}{\icnote}{white}
% arara: texcount: { output: report.txt }
\end{codebox}

\item[\rulebox{texindy}]
This rule runs \rbox{texindy}, a variant of the \rbox{xindy} indexing system focused on \LaTeX\ documents, on the corresponding base name of the \mtbox{currentFile} reference (i.e, the name without the associated extension) as a string concatenated with the \rbox{idx} suffix, generating an index as a special \rbox{ind} file.

\begin{description}
\item[\rpsbox{quiet}] This option, as the name indicates, sets whether the tool will output progress messages. It is important to observe that \rbox{texindy} always outputs error messages, regardless of this option.

\item[\abox{codepage}] This option, as the name indicates, specifies the encoding to be used for letter group headings. Additionally, it specifies the encoding used internally for sorting, but that does not matter for the final result.

\item[\abox{language}] This option, as the name indicates, specifies the language that dictates the rules for index sorting. These rules are encoded in a module.

\item[\abox{markup}] This option, as the name indicates, specifies the input markup for the raw index. The following values are available:

\begin{description}
\item[\povalue{latex}] This value, as the name implies, is emitted by default from the \LaTeX\ kernel, and the raw input is encoded in the \LaTeX\ Internal Character Representation format.

\item[\povalue{xelatex}] This value, as the name implies, acts like the previous \rbox[cyan]{latex} markup option, but without \rbox{inputenc} usage. Raw input is encoded in the UTF-8 format.

\item[\povalue{omega}] This value, as the name implies, acts like the previous \rbox[cyan]{latex} markup option, but with Omega's special notation as encoding for characters not in the ASCII set.
\end{description}

\item[\abox{modules}] This option, as the name indicates, takes a list of module names. Modules are searched in the usual application path. An error is thrown if any data structure other than a proper list is provided as the value.

\item[\rpbox{input}{idx}] This option, as the name indicates, sets the default extension for the input file, according to the provided value. Later, this value will be concatenated as a suffix for the base name of the \mtbox{currentFile} reference (i.e, the name without the associated extension).

\item[\rpbox{output}{ind}] This option, as the name indicates, sets the default extension for the output file, according to the provided value. Later, this value will be concatenated as a suffix for the base name of the \mtbox{currentFile} reference (i.e, the name without the associated extension).

\item[\rpbox{log}{ilg}] This option, as the name indicates, sets the default extension for the log file, according to the provided value. Later, this value will be concatenated as a suffix for the base name of the \mtbox{currentFile} reference (i.e, the name without the associated extension).

\item[\abox{options}] This option, as the name indicates, takes a list of raw command line options and appends it to the actual system call. An error is thrown if any data structure other than a proper list is provided as the value.
\end{description}

\begin{codebox}{Example}{teal}{\icnote}{white}
% arara: texindy: { markup: latex }
\end{codebox}

\item[\rulebox{tikzmake}]
This rule runs \rbox{make} on a very specific build file generated by the \rbox{tikzmake} package, as a means to simplify the externalization of Ti{\itshape k}Z pictures. This build file corresponds to the base name of the \mtbox{currentFile} reference (i.e, the name without the associated extension) as a string concatenated with the \rbox{makefile} suffix.

\begin{description}
\item[\rpsbox{force}] This option, as the name indicates, sets whether all targets specified in the corresponding build file should be unconditionally made.

\item[\abox{jobs}] This option, as the name indicates, specifies the number of jobs (commands) to run simultaneously. Note that the provided value must be a positive integer. The default number of job slots is one, which means serial execution.

\item[\abox{options}] This option, as the name indicates, takes a list of raw command line options and appends it to the actual system call. An error is thrown if any data structure other than a proper list is provided as the value.
\end{description}

\begin{codebox}{Example}{teal}{\icnote}{white}
% arara: tikzmake: { force: yes, jobs: 2 }
\end{codebox}

\item[\rulebox{upbibtex}]
This rule runs the \rbox{upbibtex} program, a reference management software, on the corresponding base name of the \mtbox{currentFile} reference (i.e, the name without the associated extension) as a string.

\begin{description}
\item[\abox{options}] This option, as the name indicates, takes a list of raw command line options and appends it to the actual system call. An error is thrown if any data structure other than a proper list is provided as the value.
\end{description}

\begin{codebox}{Example}{teal}{\icnote}{white}
% arara: pbibtex
% arara: --> if exists(toFile('references.bib'))
\end{codebox}

\item[\rulebox{uplatex}]
This rule runs the \rbox{uplatex} \TeX\ engine on the provided \mtbox{currentFile} reference, generating a corresponding file in a device independent format.

\begin{description}
\item[\rpbox{branch}{stable}] This option allows branching formats for the current engine, mainly focused on package development. Users of current \TeX\ distributions might benefit from format branching in order to easily test documents and code against the upcoming releases. Possible values are:

\begin{description}
\item[\povalue{stable}] This value, as the name implies, enables the stable engine format branch. Note that this is the default format.

\item[\povalue{developer}] For experienced users, this value enables the experimental, developer engine format branch.
\end{description}

\item[\abox{interaction}] This option alters the underlying engine behaviour. If this option is omitted, \TeX\ will prompt the user for interaction in the event of an error. Possible values are, in order of increasing user interaction (courtesy of our master Enrico Gregorio):

\begin{description}
\item[\povalue{batchmode}] In this mode, nothing is printed on the terminal, and errors are scrolled as if the \rbox{return} key is hit at every error. Missing files that \TeX\ tries to input or request from keyboard input cause the job to abort.

\item[\povalue{nonstopmode}] In this mode, the diagnostic message will appear on the terminal, but there is no possibility of user interaction just like in batch mode, previously described.

\item[\povalue{scrollmode}] In this mode, as the name indicates, \TeX\ will stop only for missing files to input or if proper keyboard input is necessary. \TeX\ fixes errors itself.

\item[\povalue{errorstopmode}] In this mode, \TeX\ will stop at each error, asking for proper user intervention. This is the most user interactive mode available.
\end{description}

\item[\rpsbox{shell}] This option sets whether the possibility of running underlying system commands from within \TeX\ is activated.

\item[\rpsbox{synctex}] This option sets whether \rbox{synctex}, an input and output synchronization feature that allows navigation from source to typeset material and vice versa, available in most \TeX\ engines, is activated.

\item[\abox{options}] This option, as the name indicates, takes a list of raw command line options and appends it to the actual system call. An error is thrown if any data structure other than a proper list is provided as the value.
\end{description}

\begin{codebox}{Example}{teal}{\icnote}{white}
% arara: uplatex: { interaction: scrollmode, shell: yes }
\end{codebox}

\item[\rulebox{uptex}]
This rule runs the \rbox{uptex} \TeX\ engine on the provided \mtbox{currentFile} reference, generating a corresponding file in a device independent format.

\begin{description}
\item[\abox{interaction}] This option alters the underlying engine behaviour. If this option is omitted, \TeX\ will prompt the user for interaction in the event of an error. Possible values are, in order of increasing user interaction (courtesy of our master Enrico Gregorio):

\begin{description}
\item[\povalue{batchmode}] In this mode, nothing is printed on the terminal, and errors are scrolled as if the \rbox{return} key is hit at every error. Missing files that \TeX\ tries to input or request from keyboard input cause the job to abort.

\item[\povalue{nonstopmode}] In this mode, the diagnostic message will appear on the terminal, but there is no possibility of user interaction just like in batch mode, previously described.

\item[\povalue{scrollmode}] In this mode, as the name indicates, \TeX\ will stop only for missing files to input or if proper keyboard input is necessary. \TeX\ fixes errors itself.

\item[\povalue{errorstopmode}] In this mode, \TeX\ will stop at each error, asking for proper user intervention. This is the most user interactive mode available.
\end{description}

\item[\rpsbox{shell}] This option sets whether the possibility of running underlying system commands from within \TeX\ is activated.

\item[\rpsbox{synctex}] This option sets whether \rbox{synctex}, an input and output synchronization feature that allows navigation from source to typeset material and vice versa, available in most \TeX\ engines, is activated.

\item[\abox{options}] This option, as the name indicates, takes a list of raw command line options and appends it to the actual system call. An error is thrown if any data structure other than a proper list is provided as the value.
\end{description}

\begin{codebox}{Example}{teal}{\icnote}{white}
% arara: uptex
\end{codebox}

\item[\rulebox{xdvipdfmx}]
This rule runs \rbox{xdvipdfmx}, the back end for the \rbox{xetex} \TeX\ engine (and not intended to be invoked directly), on the corresponding base name of the \mtbox{currentFile} reference (i.e, the name without the associated extension) as a string concatenated with a certain suffix, generating a Portable Document Format \rbox{pdf} file.

\item[\abox{entry}] This option sets the extension to be used for the initial reference (i.e, the current file name) as input to the command line tool. The following values are available for this option:

\begin{description}
\item[\povalue{dvi}] This value sets the extension to be used for the initial reference as a device independent format. This is the default value when no value is provided.

\item[\povalue{xdv}] This value sets the extension to be used for the initial reference as an extended device independent format.
\end{description}

\begin{description}
\item[\abox{output}] This option, as the name indicates, sets the output name for the generated \rbox{pdf} file. There is no need to provide an extension, as the value is always normalized with \mtbox{getBasename} such that only the name without the associated extension is used. The base name of the current file reference is used as the default value.

\item[\abox{options}] This option, as the name indicates, takes a list of raw command line options and appends it to the actual system call. An error is thrown if any data structure other than a proper list is provided as the value.
\end{description}

\begin{codebox}{Example}{teal}{\icnote}{white}
% arara: xdvipdfmx: { output: thesis }
\end{codebox}

\item[\rulebox{xelatex}]
This rule runs the new \rbox{xelatex} \TeX\ engine on the provided \mtbox{currentFile} reference, generating a corresponding file in the Portable Document Format, as expected.

\begin{description}
\item[\rpbox{branch}{stable}] This option allows branching formats for the current engine, mainly focused on package development. Users of current \TeX\ distributions might benefit from format branching in order to easily test documents and code against the upcoming releases. Possible values are:

\begin{description}
\item[\povalue{stable}] This value, as the name implies, enables the stable engine format branch. Note that this is the default format.

\item[\povalue{developer}] For experienced users, this value enables the experimental, developer engine format branch.

\item[\povalue{unsafe}] This value enables the unsafe engine format branch. According to the documentation, at all costs, avoid using this, or any, unsafe invocation with documents off the net or that are otherwise untrusted in any way.
\end{description}

\item[\abox{interaction}] This option alters the underlying engine behaviour. If this option is omitted, \TeX\ will prompt the user for interaction in the event of an error. Possible values are, in order of increasing user interaction (courtesy of our master Enrico Gregorio):

\begin{description}
\item[\povalue{batchmode}] In this mode, nothing is printed on the terminal, and errors are scrolled as if the \rbox{return} key is hit at every error. Missing files that \TeX\ tries to input or request from keyboard input cause the job to abort.

\item[\povalue{nonstopmode}] In this mode, the diagnostic message will appear on the terminal, but there is no possibility of user interaction just like in batch mode, previously described.

\item[\povalue{scrollmode}] In this mode, as the name indicates, \TeX\ will stop only for missing files to input or if proper keyboard input is necessary. \TeX\ fixes errors itself.

\item[\povalue{errorstopmode}] In this mode, \TeX\ will stop at each error, asking for proper user intervention. This is the most user interactive mode available.
\end{description}

\item[\rpsbox{shell}] This option sets whether the possibility of running underlying system commands from within \TeX\ is activated.

\item[\rpsbox{synctex}] This option sets whether \rbox{synctex}, an input and output synchronization feature that allows navigation from source to typeset material and vice versa, available in most \TeX\ engines, is activated.

\item[\abox{options}] This option, as the name indicates, takes a list of raw command line options and appends it to the actual system call. An error is thrown if any data structure other than a proper list is provided as the value.
\end{description}

\begin{codebox}{Example}{teal}{\icnote}{white}
% arara: xelatex: { shell: yes, synctex: yes }
\end{codebox}

\item[\rulebox{xetex}]
This rule runs the \rbox{xetex} \TeX\ engine on the provided \mtbox{currentFile} reference, generating a corresponding file in the Portable Document Format, as expected.

\begin{description}
\item[\rpbox{branch}{stable}] This option allows branching formats for the current engine, mainly focused on package development. Users of current \TeX\ distributions might benefit from format branching in order to easily test documents and code against the upcoming releases. Possible values are:

\begin{description}
\item[\povalue{stable}] This value, as the name implies, enables the stable engine format branch. Note that this is the default format.

\item[\povalue{unsafe}] This value enables the unsafe engine format branch. According to the documentation, at all costs, avoid using this, or any, unsafe invocation with documents off the net or that are otherwise untrusted in any way.
\end{description}

\item[\abox{interaction}] This option alters the underlying engine behaviour. If this option is omitted, \TeX\ will prompt the user for interaction in the event of an error. Possible values are, in order of increasing user interaction (courtesy of our master Enrico Gregorio):

\begin{description}
\item[\povalue{batchmode}] In this mode, nothing is printed on the terminal, and errors are scrolled as if the \rbox{return} key is hit at every error. Missing files that \TeX\ tries to input or request from keyboard input cause the job to abort.

\item[\povalue{nonstopmode}] In this mode, the diagnostic message will appear on the terminal, but there is no possibility of user interaction just like in batch mode, previously described.

\item[\povalue{scrollmode}] In this mode, as the name indicates, \TeX\ will stop only for missing files to input or if proper keyboard input is necessary. \TeX\ fixes errors itself.

\item[\povalue{errorstopmode}] In this mode, \TeX\ will stop at each error, asking for proper user intervention. This is the most user interactive mode available.
\end{description}

\item[\rpsbox{shell}] This option sets whether the possibility of running underlying system commands from within \TeX\ is activated.

\item[\rpsbox{synctex}] This option sets whether \rbox{synctex}, an input and output synchronization feature that allows navigation from source to typeset material and vice versa, available in most \TeX\ engines, is activated.

\item[\abox{options}] This option, as the name indicates, takes a list of raw command line options and appends it to the actual system call. An error is thrown if any data structure other than a proper list is provided as the value.
\end{description}

\begin{codebox}{Example}{teal}{\icnote}{white}
% arara: xetex: { interaction: scrollmode, synctex: yes }
\end{codebox}

\item[\rulebox{xindex}]
This rule runs \rbox{xindex}, a flexible and powerful indexing system, on a provided \rbox{idx} input. This tool is completely with the \rbox{makeindex} program.

\begin{description}
\item[\abox{input}~\rqbox] This option, as the name indicates, corresponds to the \rbox{idx} reference to be processed by the indexing system. Note that this option is required.

\item[\rpbox{config}{cfg}] This option specifies a configuration extension. Make sure to take a look at the documentation for further details.

\item[\rpbox{language}{en}] This option, as the name suggests, specifies the language. Make sure to take a look at the documentation for further details.

\item[\abox{options}] This option, as the name indicates, takes a list of raw command line options and appends it to the actual system call. An error is thrown if any data structure other than a proper list is provided as the value.
\end{description}

\begin{codebox}{Example}{teal}{\icnote}{white}
% arara: xindex: { input: mydoc.idx }
\end{codebox}

\item[\rulebox{xindy}]
This rule runs \rbox{xindy}, a flexible and powerful indexing system, on the corresponding base name of the \mtbox{currentFile} reference (i.e, the name without the associated extension) as a string concatenated with the \rbox{idx} suffix, generating an index as a special \rbox{ind} file.

\begin{description}
\item[\rpsbox{quiet}] This option, as the name indicates, sets whether the tool will output progress messages. It is important to observe that \rbox{xindy} always outputs error messages, regardless of this option.

\item[\abox{codepage}] This option, as the name indicates, specifies the encoding to be used for letter group headings. Additionally, it specifies the encoding used internally for sorting, but that does not matter for the final result.

\item[\abox{language}] This option, as the name indicates, specifies the language that dictates the rules for index sorting. These rules are encoded in a module.

\item[\abox{markup}] This option, as the name indicates, specifies the input markup for the raw index. The following values are available:

\begin{description}
\item[\povalue{latex}] This value, as the name implies, is emitted by default from the \LaTeX\ kernel, and the raw input is encoded in the \LaTeX\ Internal Character Representation format.

\item[\povalue{xelatex}] This value, as the name implies, acts like the previous \rbox[cyan]{latex} markup option, but without \rbox{inputenc} usage. Raw input is encoded in the UTF-8 format.

\item[\povalue{omega}] This value, as the name implies, acts like the previous \rbox[cyan]{latex} markup option, but with Omega's special notation as encoding for characters not in the ASCII set.

\item[\povalue{xindy}] This value, as the name implies, uses the \rbox{xindy} input markup as specified in the \rbox{xindy} manual.
\end{description}

\item[\abox{modules}] This option, as the name indicates, takes a list of module names. Modules are searched in the usual application path. An error is thrown if any data structure other than a proper list is provided as the value.

\item[\rpbox{input}{idx}] This option, as the name indicates, sets the default extension for the input file, according to the provided value. Later, this value will be concatenated as a suffix for the base name of the \mtbox{currentFile} reference (i.e, the name without the associated extension).

\item[\rpbox{output}{ind}] This option, as the name indicates, sets the default extension for the output file, according to the provided value. Later, this value will be concatenated as a suffix for the base name of the \mtbox{currentFile} reference (i.e, the name without the associated extension).

\item[\rpbox{log}{ilg}] This option, as the name indicates, sets the default extension for the log file, according to the provided value. Later, this value will be concatenated as a suffix for the base name of the \mtbox{currentFile} reference (i.e, the name without the associated extension).

\item[\abox{options}] This option, as the name indicates, takes a list of raw command line options and appends it to the actual system call. An error is thrown if any data structure other than a proper list is provided as the value.
\end{description}

\begin{codebox}{Example}{teal}{\icnote}{white}
% arara: xindy: { markup: xelatex }
\end{codebox}
\end{description}

It is highly advisable to browse the relevant documentation about packages and tools described in this chapter as a means to learn more about features and corresponding advanced usage. For \TeX\ Live users, we recommend the use of \rbox{texdoc}, a command line program to find and view documentation. For example, this manual can be viewed through the following command:

\begin{codebox}{Terminal}{teal}{\icnote}{white}
$ texdoc arara
\end{codebox}

The primary function of the handy \rbox{texdoc} tool is to locate relevant documentation for a given keyword (typically, a package name) on your disk, and open it in an appropriate viewer. For MiK\TeX\ users, the distribution provides a similar tool named \rbox{mthelp} to find and view documentation. Make sure to use these tools whenever needed!
